\documentclass[10pt,letterpaper]{article}


\usepackage{bm}
\usepackage{geometry}
\usepackage{natbib,graphics,epsfig,rotate,lscape,graphicx,amsmath,amsthm,amssymb,float,amsfonts,amsbsy,hyperref,delarray,sectsty,amsfonts,amscd,pifont}
\usepackage{color,multirow}
\usepackage{algorithmic}
\usepackage{algorithm}



\geometry{letterpaper,left=1in,right=1in,top=1.2in,bottom=1.1in}
\bibliographystyle{plain}
\allowdisplaybreaks
%\def\references{\bibliography{C:/hanstex/macros/bib/11-1-11}}

\newtheorem{theorem}{Theorem}
\newtheorem{lemma}{Lemma}
\newtheorem{corollary}{Corollary}
\newtheorem{Prop}{Proposition}
\newtheorem{aside}{Aside}
\newtheorem{claim}{Claim}
\newtheorem{conjecture}{Conjecture}
\newtheorem{definition}{Definition}
\newtheorem{proposition}{Proposition}

\newcommand{\bea}{\begin{eqnarray*}}
\newcommand{\eea}{\end{eqnarray*}}
\newcommand{\ed}{\end{document}}
\newcommand{\no}{\noindent}
\newcommand{\et}{\textit{et al. }}
\newcommand{\btab}{\begin{tabular}}
\newcommand{\etab}{\end{tabular}}
\newcommand{\bc}{\begin{center}}
\newcommand{\ec}{\end{center}}
\newcommand{\np}{\newpage}
\newcommand{\la}{\label}
\newcommand{\bi}{\begin{itemize}}
\newcommand{\ei}{\end{itemize}}
\newcommand{\bfi}{\begin{figure}}
\newcommand{\efi}{\end{figure}}
\newcommand{\ben}{\begin{enumerate}}
\newcommand{\een}{\end{enumerate}}
\newcommand{\bdes}{\begin{description}}
\newcommand{\edes}{\end{description}}
\newcommand{\bay}{\begin{array}}
\newcommand{\eay}{\end{array}}
\newcommand{\bs}{\boldsymbol}
\newcommand{\mb}{\boldsymbol}
\newcommand{\nn}{\nonumber}
\newcommand{\sm}{\vspace{.2cm}}
\newcommand{\bla}{\textcolor{black}}
\newcommand{\blu}{\textcolor{blue}}
\newcommand{\red}{\textcolor{red}}

\def\stackunder#1#2{\mathrel{\mathop{#2}\limits_{#1}}}
\renewcommand{\labelenumi}{(\roman{enumi})}
\newcommand{\Comment}[1]{\textcolor{blue}{\textsc{#1}}}


\def\Ver{1}
\def\LongVer{1}
%%------------------------------ begin long version
%\if\Ver\LongVer{ 
%{\flushleft\textcolor{blue}{$\downarrow$---------begin long version---------}}\newline
%
%{\flushleft\textcolor{blue}{$\uparrow$------------end long version---------}}\newline
%} \fi
%%------------------------------ end long version




\newcommand{\be}{\begin{eqnarray}}
\newcommand{\ee}{\end{eqnarray}}
\renewcommand{\baselinestretch}{1}%{1.7}

%====================================================================================
\begin{document}
\begin{center}
\Large \bf
Lemmas for the Carath\'eodory Extension Theorem.
\end{center}

\paragraph{Definitions}
\begin{enumerate}
\item $P_0$ is a probability measure on the field $\mathcal F_0$ of subsets in $\Omega$.
\item  $\mathcal F^\uparrow := \{\lim_k\!\!\uparrow A_k : A_k\in \mathcal F_0  \}$ and  $P^\uparrow(\lim_k\!\!\uparrow A_k) := \lim_k P_0(A_k)$ 
\item  $\mathcal F^\downarrow := \{\lim_k\!\!\downarrow A_k : A_k\in \mathcal F_0  \}$ and  $P^\downarrow(\lim_k\!\!\downarrow A_k) := \lim_k P_0(A_k)$ 
\item  $P^*(A)=\inf\{P^\uparrow(B): A\subset B\in \mathcal F^\uparrow  \}$
\item $P_*(A)=\sup\{P^\downarrow(B): A\supset B\in \mathcal F^\downarrow  \}$
\item  $\overline{\mathcal F}:=\{ A\in 2^\Omega : P^*(A) = P_*(A)\}$ and $\overline P(A):=P^*(A)=P_*(A)$ for $A\in \overline{\mathcal F}$
\item $\mathcal F :=\sigma\langle \mathcal F_0 \rangle$. Note: we will show $\mathcal F\subset \overline{\mathcal F}$.
\item $P(A):= \overline P(A)$ for all $A\in \mathcal F$.
\end{enumerate}


%%%%%%%%%%%%%%%%%%%
\paragraph{Lemma 1.}
\begin{enumerate}
\item   $\lim_k\!\!\uparrow (A_k \cap B) = (\lim_k\!\!\uparrow A_k) \cap B$
\item $\lim_k\!\!\uparrow (A_k \cup B) = (\lim_k\!\!\uparrow A_k) \cup B$
\end{enumerate}


%%%%%%%%%%%%%%%%%%%
\paragraph{Lemma 2. Properties of $\mathcal F^\uparrow$ and $\mathcal F^\downarrow$.}
\begin{enumerate}
\item  $A\in \mathcal F^\uparrow \Leftrightarrow A^c \in \mathcal F^\downarrow$
\item $\mathcal F^{\uparrow}= \left\{ \bigcup_{k=1}^\infty A_k \colon A_k\in \mathcal F_0 \right\}$
\item $\mathcal F^{\downarrow}= \left\{ \bigcap_{k=1}^\infty A_k \colon A_k\in \mathcal F_0 \right\}$ 
\item If $A, B\in \mathcal F^{\uparrow}$ then $A\cap B\in \mathcal  F^{\uparrow}$ and $A\cup B\in \mathcal  F^{\uparrow}$.
\item $\mathcal F^{\uparrow}$ is closed under countable unions and increasing limits of $\mathcal F^\uparrow$ sets.
%\item  $\mathcal F^{\downarrow}$ is closed under countable intersections and decreasing limits of sets.
\end{enumerate}


%%%%%%%%%%%%%%%%%%%
\paragraph{Lemma 3. Properties of $P^\uparrow$ and $P^\downarrow$.}
\begin{enumerate}
\item If $A\in \mathcal F^{\uparrow}$ then $P^\uparrow (A) + P^\downarrow(A^c) = 1$.
\item  If $A, B\in \mathcal F^\uparrow$ then $P^\uparrow(A\cup B)= P^\uparrow(A) + P^\uparrow(B) - P^\uparrow(A\cap B)$.
\item If $A\subset B$ and $A, B\in \mathcal F^\uparrow$ then $P^\downarrow (A)\leq P^\uparrow (A)\leq P^\uparrow(B)$. 
%\item If $A \in \mathcal F^\uparrow$ then $P^\downarrow (A)\leq P\uparrow\leq P^\uparrow(A)$.
\item If $\lim_n\!\!\uparrow A_n= A$ and $A_n\in \mathcal F^\uparrow$ then $P^\uparrow(A_n)\nearrow P^\uparrow(A)$.
\end{enumerate}


%%%%%%%%%%%%%%%%%%%%%
\paragraph{Lemma 4. Properties of $P^*$ and $P_*$.}
\begin{enumerate}
\item[Fact 1:] If $A\in 2^\Omega$ then $P^* (A) + P_*(A^c) = 1$. Almost the complement rule.
\item[Fact 2:]  If $A\subset B\subset C$ and $A, B, C\in2^\Omega$ then 
$P_* (A)\leq P_*(B)\leq P^*(B)\leq P^*(C)$.
\item[Fact 3:]  If $A, B\in 2^\Omega$ then $P^*(A\cup B)\leq P^*(A) + P^*(B) - P^*(A\cap B)$. Almost inclusion-exclusion.
\item[Fact 4:]  If $A, B\in 2^\Omega$ then $P_*(A\cup B)\geq P_*(A) + P_*(B) - P_*(A\cap B)$.  Almost inclusion-exclusion.
\item If $A_k\subset B_k$ then $P^*(\cup_{k=1}^n B_k) - P^*(\cup_{k=1}^n 
A_k)\leq \sum_{k=1}^n \bigl[P^*(B_k) - P^*(A_k)\bigr]$. Approximating unions term-by-term.
\item[Fact 5:] If $\lim_n\!\!\uparrow A_n= A$ then $P^*(A_n)\nearrow P^*(A)$. Continuous from below.
\end{enumerate}

\newpage
%%%%%%%%%%%%%%%%%%
\paragraph{Proof of Lemma 1.}
\begin{enumerate}
\item
 Notice that $A_k \subset A_{k+1}$ implies $ A_k\cap B \subset A_{k+1}\cap B$ for all $k=1,2,\ldots$. Therefore
\begin{align*}
 \lim_k\!\!\uparrow (A_k\cap B) &= \bigcup_{k=1}^\infty (A_k\cap B) = B\cap \bigcup_{k=1}^\infty A_k = B\cap\lim_k\!\!\uparrow A_k.
\end{align*} 
\item Similar to (i). 
\end{enumerate}

%%%%%%%%%%%%%%%%%%%%%%%%
\paragraph{Proof of Lemma 2.}
\begin{enumerate}
\item 
\begin{align*}
A\in \mathcal F^\uparrow &\Longleftrightarrow \exists A_k \text{  s.t. } A_k\subset A_{k+1}\text{ and  } \bigcup_{k=1}^\infty A_k = A \\
&\Longleftrightarrow \exists A^c_k \text{  s.t. } A^c_k\supset A^c_{k+1}\text{ and } \bigcap_{k=1}^\infty A_k^c = A^c \\
&\Longleftrightarrow A^c\in \mathcal F^\downarrow.
\end{align*}
\item The fact that $\mathcal F^\uparrow \subset \left\{ \bigcup_{k=1}^\infty A_k \colon A_k\in \mathcal F_0 \right\}$ is trival since $\lim_k\!\! \uparrow A_k := \bigcup_{k=1}^\infty A_k$. For the other inclusion let $ \bigcup_{k=1}^\infty A_k\in \left\{ \bigcup_{k=1}^\infty A_k \colon A_k\in \mathcal F_0 \right\}$. Then 
\[\underbrace{\bigcup_{k=1}^n A_k}_{\text{ in $\mathcal F_0$}} \,\,\uparrow  \underbrace{\bigcup_{k=1}^\infty A_k}_{\therefore \text{ this is in }\mathcal F^\uparrow}\]
Therefore $\left\{ \bigcup_{k=1}^\infty A_k \colon A_k\in \mathcal F_0 \right\}\subset \mathcal F^\uparrow$.
\item This proof is similar to (ii).
\item  Let $A, B\in \mathcal F^\uparrow$. Then there increasing exists $\mathcal F_0$ sets $A_k$ and $B_k$ such that $A= \bigcup_{k=1}^\infty A_k$ and $B= \bigcup_{k=1}^\infty B_k$. Notice that $A_k\cap B_k$ and $A_k\cup B_k$ are both increasing sets in $k$. To see why $\lim_k\!\!\uparrow (A_k\cap B_k) = A\cap B$ notice
\begin{align}
w\in \lim_k\!\!\uparrow(A_k\cap B_k) 
&\Longleftrightarrow w \in \bigcup_{k=1}^\infty (A_k\cap B_k)\nonumber\\
&\Longleftrightarrow \text{there exists a $k_0$ such that }w \in A_{k_0}\text{ and }w\in B_{k_0}\nonumber\\
&\Longrightarrow w \in \bigcup_{k=1}^\infty A_k=A \text{ and }w \in \bigcup_{k=1}^\infty B_k=B\label{firstR}\\
&\Longleftrightarrow w \in A\cap B\nonumber
\end{align}
Therefore  $\lim_k\!\!\uparrow (A_k\cap B_k) \subset A\cap B$. To see the other inclusion notice that the `$\Longrightarrow$' in conditional (\ref{firstR}) can be turned into  `$\Longleftrightarrow$'. This follows since $w \in \bigcup_{k=1}^\infty A_k$ implies there exists a $k_1$ such that  $w\in A_{k_1}\subset A_{k_1+1}\subset \cdots$ by monotonicity and  similarly $w \in \bigcup_{k=1}^\infty B_k$ implies there exists a $k_2$ such that  $w\in B_{k_2}\subset B_{k_2+1}\subset \cdots$. Now taking $k_0:=max(k_1,k_2)$ shows that $w \in A_{k_0}$ and $w\in B_{k_0}$. Therefore 
\[ A,B \in \mathcal F^\uparrow \Longrightarrow A\cap B= \lim_k\!\!\uparrow \underbrace{(A_k\cap B_k)}_{\in \mathcal F_0} \in \mathcal F^\uparrow\]
The proof that $A\cup B \in \mathcal F^\uparrow$ is similar.
\item Let $A_k\subset A_{k+1}$ be $\mathcal F^\uparrow$ sets for $k\in \Bbb N:=\{ 1,2,3,\ldots\}$. We show $\lim_k\!\!\uparrow A_k := \bigcup_{k=1}^\infty A_k\in \mathcal F^\uparrow $. First write 
\[ \bigcup_{k=1}^\infty A_k =  \bigcup_{k=1}^\infty  \bigcup_{m=1}^\infty A_{k,m}\]
 where for each $k\in \Bbb N$,  $A_{k,m}\in \mathcal F_0$ and $\lim_m\!\!\uparrow A_{k,m} = A_k$. We show $\lim_N\!\!\uparrow \bigcup^{N}_{k=1}\bigcup^N_{m=1}A_{k,m} = A$ which would show that $A\in \mathcal F^\uparrow$ since $\bigcup^{N}_{k=1}\bigcup^N_{m=1}A_{k,m} \in \mathcal F_0$. Clearly $\bigcup^{N}_{k=1}\bigcup^N_{m=1}A_{k,m}$ increases in $N$.
Notice also that for any $M\leq N$
\begin{equation}
\label{monotone}
 A_{M,N} \subset \bigcup^{N}_{k=1}\bigcup^N_{m=1}A_{k,m}\subset \bigcup^{N}_{k=1}\bigcup^\infty_{m=1}A_{k,m} =\bigcup^{N}_{k=1}A_{k}=A_N.\end{equation}
Taking limits as $N\rightarrow \infty$ gives
 \[ A_{M} \subset\lim_N\!\!\uparrow \bigcup^{N}_{k=1}\bigcup^N_{m=1}A_{k,m} \subset \lim_N\!\!\uparrow A_N=A.
  \]
Taking limits as $M\rightarrow\infty$ gives
\begin{equation}
\label{tt}
A = \lim_N\!\!\uparrow\underbrace{ \bigcup^{N}_{k=1}\bigcup^N_{m=1}A_{k,m}}_{\in \mathcal F_0} \in \mathcal F^\uparrow.\end{equation}
Now it's easy to also show that $\mathcal F^\uparrow$ is also closed under countable unions of $\mathcal F^\uparrow$ sets (since partial unions increase up to infinite unions and since partial unions of $\mathcal F^\uparrow$ sets are also  $\mathcal F^\uparrow$ sets by Lemma 2(iv) ).
 \end{enumerate} 


%%%%%%%%%%
\paragraph{Proof of Lemma 3.}
\begin{enumerate}
\item Suppose $A\in \mathcal F^{\uparrow}$. Let $A_k\in \mathcal F_0$ such that $A_k\uparrow A$. Then $A^c_k\downarrow A^c$ and 
\begin{align}
P^\uparrow(A)+P^\downarrow(A^c)= \lim_k \underbrace{[P_0(A_k)+ P_0(A_k^c)]}_{=1}=1.
\end{align}

\item Suppose $A, B\in \mathcal F^\uparrow$. Let $A_k, B_k\in \mathcal F_0$ such that  $A_k\uparrow A$ and $B_k\uparrow B$. Notice that the proof of  Lemma 2(iv) shows that $A_k\cup B_k\uparrow A\cup B\in \mathcal F^\uparrow$ and $A_k\cap B_k \uparrow A\cap B\in\mathcal F^\uparrow$. Therefore
\begin{align*}
P^\uparrow(A\cup B)&=\lim_k\!\!\uparrow P_0(A_k\cup B_k) \\
&=\lim_k\!\!\uparrow [ P_0(A_k) + P_0( B_k)- P_0(A_k\cap B_k) ] \\
&=P^\uparrow(A) + P^\uparrow( B)- P^\uparrow(A\cap B).
\end{align*}
\item Suppose $A,B \in \mathcal F^\uparrow$ and $A\subset B$. To see why $P^\downarrow(A)\leq P^\uparrow(A)$ notice that $A\cup A^c = \Omega$ implies
\begin{align*}
1&=P^\uparrow(\Omega) = P^\uparrow (A\cup A^c)\\
&\leq  P^\uparrow (A) + P^\uparrow( A^c)\quad\text{by (ii)} \\
&\leq  P^\uparrow (A) + 1- P^\downarrow( A)\quad\text{by (i)} 
\end{align*}


Next, to see why $P^\uparrow(A)\leq P^\uparrow(B)$ notice that $A\subset B$ impiles that $B=A\cup (B- A)$ is a disjoint decomposition of $B$ (i.e. $B$ = hole + ring). Therefore
\begin{align*}
P^\uparrow(B) &= P^\uparrow(A) + P^\uparrow(B- A)-0, \quad\text{by (ii)}\\
&\geq  P^\uparrow(A).
\end{align*}


\item  Suppose $\lim_n\!\!\uparrow A_n= A$ and $A_n\in \mathcal F^\uparrow$. Then from (iii)  $P^\uparrow(A_n)$ is monotonically increasing and bounded about by $P^\uparrow(A)$. We just need to show the limit is $P^\uparrow(A)$. By equation (\ref{monotone}) and (iii)
\begin{equation}
\label{ttt}
 P^\uparrow\Bigl(\underbrace{\bigcup_{k=1}^N \bigcup_{m=1}^N A_{k,m}}_{\in \mathcal F_0}\Bigr)\leq P^\uparrow(A_N)\leq  P^\uparrow(A). \end{equation}
Notice that
\begin{align*}
 \lim_N P^\uparrow\Bigl(\bigcup_{k=1}^N \bigcup_{m=1}^N A_{k,m}\Bigr) &= \lim_N P_0\Bigl(\bigcup_{k=1}^N \bigcup_{m=1}^N A_{k,m}\Bigr),\quad \text{since $\bigcup_{k=1}^N \bigcup_{m=1}^N A_{k,m}\in \mathcal F_0$} \\
 &= P^\uparrow\Bigl( \underbrace{\lim_N\!\!\uparrow \bigcup_{k=1}^N \bigcup_{m=1}^N A_{k,m}}_{=A,\text{ by (\ref{tt})}}\Bigr),\quad \text{by definition of $P^\uparrow$.}
 \end{align*}
 Therefore taking limits in $N$ in (\ref{ttt}) one obtains $ P^\uparrow(A)\leq \lim_N P^\uparrow(A_N)  \leq P^\uparrow(A)$.
\end{enumerate}



%%%%%%%%%%%%%%%%%%%%%
\paragraph{Lemma 4.}
\begin{enumerate}
\item We show this in two stages. 

First we show that for any $A, B, C\in 2^\Omega$ such that $A\subset B$ then 
\begin{equation}
\label{s1}
P^*(B\cup C)- P^*(A\cup C) \leq P^*(B)-P^*(A).
\end{equation}
To see why notice first that (\ref{s1}) is equivalent to $ P^*(B)+ P^*(A\cup C)\geq P^*(B\cup C)+P^*(A) $ which is true because 
\begin{align*}
P^*(B)+ P^*(A\cup C)&\geq P^*[B\cap (A\cup C)] + P^*[B\cup (A\cup C)], \quad\text{by fact 3}\\
&= P^*[(B\cap A)\cup (B\cap C)] + P^*[A\cup B\cup C] \\
&\geq P^*[A\cup (B\cap C)] + P^*[B\cup C],\quad\text{since $A\subset B$} \\
&\geq P^*[A] + P^*[B\cup C].
\end{align*}


Secondly we use (\ref{s1}) to show that for any $A_1\subset B_1$ and $A_2\subset B_2$
\begin{equation}
\label{s2}
P^*(B_1\cup B_2)- P^*(A_1\cup A_1)\leq  \bigl[ P^*(B_1) -  P^*(A_1)\bigr] +  \bigl[ P^*(B_2) -  P^*(A_2)\bigr].
\end{equation}
Notice the two following inequalities which follow directly from (\ref{s1})
\begin{align*}
P^*(B_1\cup A_2) - P^*(A_1\cup A_2)&\leq \bigl[P^*(B_1) - P^*(A_1)\bigr]\\
P^*(B_1\cup B_2) - P^*(B_1\cup A_2)&\leq \bigl[P^*(B_2) - P^*(A_2)\bigr].
\end{align*}
Adding the above two inequalities gives (\ref{s2}). Now induction proves the claim.


\end{enumerate}

\end{document}



















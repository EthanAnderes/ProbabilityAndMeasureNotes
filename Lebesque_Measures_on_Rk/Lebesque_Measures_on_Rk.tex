\documentclass[10pt,letterpaper,twocolumn]{article}


\usepackage{bm}
\usepackage{geometry}
\usepackage{natbib,graphics,epsfig,rotate,lscape,graphicx,amsmath,amsthm,amssymb,float,amsfonts,amsbsy,hyperref,delarray,sectsty,amsfonts,amscd,pifont}
\usepackage{color,multirow}
\usepackage{algorithmic}
\usepackage{algorithm}



\geometry{letterpaper,left=.4in,right=.4in,top=.8in,bottom=1in}
\bibliographystyle{plain}
\allowdisplaybreaks
%\def\references{\bibliography{C:/hanstex/macros/bib/11-1-11}}

\newtheorem{theorem}{Theorem}
\newtheorem{example}{Example}
\newtheorem{lemma}{Lemma}
\newtheorem{corollary}{Corollary}
\newtheorem{Prop}{Proposition}
\newtheorem{aside}{Aside}
\newtheorem{claim}{Claim}
\newtheorem{conjecture}{Conjecture}
\newtheorem{definition}{Definition}
\newtheorem{proposition}{Proposition}
\newtheorem{exercise}{Exercise}

\newcommand{\bea}{\begin{eqnarray*}}
\newcommand{\eea}{\end{eqnarray*}}
\newcommand{\ed}{\end{document}}
\newcommand{\no}{\noindent}
\newcommand{\et}{\textit{et al. }}
\newcommand{\btab}{\begin{tabular}}
\newcommand{\etab}{\end{tabular}}
\newcommand{\bc}{\begin{center}}
\newcommand{\ec}{\end{center}}
\newcommand{\np}{\newpage}
\newcommand{\la}{\label}
\newcommand{\bi}{\begin{itemize}}
\newcommand{\ei}{\end{itemize}}
\newcommand{\bfi}{\begin{figure}}
\newcommand{\efi}{\end{figure}}
\newcommand{\ben}{\begin{enumerate}}
\newcommand{\een}{\end{enumerate}}
\newcommand{\bdes}{\begin{description}}
\newcommand{\edes}{\end{description}}
\newcommand{\bay}{\begin{array}}
\newcommand{\eay}{\end{array}}
\newcommand{\bs}{\boldsymbol}
\newcommand{\mb}{\boldsymbol}
\newcommand{\nn}{\nonumber}
\newcommand{\sm}{\vspace{.2cm}}
\newcommand{\bla}{\textcolor{black}}
\newcommand{\blu}{\textcolor{blue}}
\newcommand{\red}{\textcolor{red}}

\def\stackunder#1#2{\mathrel{\mathop{#2}\limits_{#1}}}
\renewcommand{\labelenumi}{(\roman{enumi})}
\newcommand{\Comment}[1]{\textcolor{blue}{\textsc{#1}}}

\def\Ver{1}
\def\LongVer{1}

\newcommand{\question}[1]{ {\vspace{.5cm}\noindent \bf Question:}\\  #1} 
\newcommand{\comment}[1]{ {\vspace{.2cm}\noindent\it Comments:}\\  #1} 
\newcommand{\answer}[1]{ \if\Ver\LongVer{ {\vspace{.2cm}\noindent \it Answer:} \footnotesize \\ #1} \fi }


%%------------------------------ begin long version
%\if\Ver\LongVer{ 
%{\flushleft\textcolor{blue}{$\downarrow$---------begin long version---------}}\newline
%
%{\flushleft\textcolor{blue}{$\uparrow$------------end long version---------}}\newline
%} \fi
%%------------------------------ end long version




\newcommand{\be}{\begin{eqnarray}}
\newcommand{\ee}{\end{eqnarray}}
\renewcommand{\baselinestretch}{1}%{1.7}

%====================================================================================
\begin{document}

\title{Lebesque measure on $\Bbb R^d$}
\date{}
\maketitle

%\abstract{In this note we collect some facts about Lebesque measure on $\Bbb R^d$. We focus on the following objects: The Borel  $\sigma$-field  $\mathcal R^d$;  Lebesque measure $\mathcal L^d$; The completion $\bar{\mathcal R}^d$ and $\bar{\mathcal L}^d$; Hasusdorf measure $\mathcal H^d$; and finally how to specify measures on $(\Bbb R^d,\mathcal R^d)$ using distributions functions.
%We also include a construction of non-measureable sets. 
%}
%
\tableofcontents

%\section{The Borel $\sigma$-field $\mathcal R^d$}

%%%%%%%%%%%%%%%%%%
%%%%%%%%%%%%%%%%%
\section{The Borel field $\mathcal B_0^{(0,1]^d}$}

\begin{definition} The Borel field on $(0,1]^d$, denoted $\mathcal B_0^{(0,1]^d}$, is defined as the field generated by the rectangles in $(0,1]^d$ as follows
\[
 \mathcal B_0^{(0,1]^d}:=f\bigl \langle (a_1,b_1]\times\cdots\times (a_d,b_d] : 0\leq a_k < b_k\leq1 \bigr\rangle.
\]
\end{definition}
The most important fact about this field is the following characterization of a general set in $\mathcal B_0^{(0,1]^d}$.
\begin{claim} \label{bfield}
Any set in  $\mathcal B_0^{(0,1]^d}$ is a  finite (possibly empty) disjoint union of rectangles from $\{(a_1,b_1]\times\cdots\times (a_d,b_d] : 0\leq a_k < b_k\leq1  \}$.
\end{claim}
%
%\begin{proof}[Proof of Claim \ref{bfield}]
%Let $\mathcal F_0$ denote the class of set which are a  finite disjoint union of rectangles from $\{(a_1,b_1]\times\cdots\times (a_d,b_d] : 0\leq a_k\leq b_k\leq1  \}$. One can use ``good sets"  to show that $\mathcal B_0^d\subset \mathcal F_0$ and then to notice that the field properties of $\mathcal B_0^d$ immediately gives  $\mathcal F_0\subset\mathcal B_0^d$.
%\end{proof}

The reason this claim is so important is that it allows us to easily define the notion of $d$-dimensional volume  on sets in $\mathcal B_0^{(0,1]}$.  Then, by showing that this $d$-dimensional volume satisfies the axioms of a probability measure on $\mathcal B_0^{(0,1]}$  we can invoke the Carath\'eodory extension theorem to get a $d$-dimensional volume measurement over all subsets in the Borel $\sigma$-field $\mathcal B^{(0,1]^d} :=\sigma \langle \mathcal B_0^{(0,1]^d}\rangle$ which will result in Lebesque measure on $(0,1]^d$.


%%%%%%%%%%%%%%%%%%
%%%%%%%%%%%%%%%%%
\section{The Borel $\sigma$-field $\mathcal B^{(0,1]^d}$}

\begin{definition} The Borel $\sigma$-field on $(0,1]^d$, denoted $\mathcal B^{(0,1]^d}$, is defined as the field generated by the rectangles in $(0,1]^d$ as follows
\[
 \sigma\bigl \langle (a_1,b_1]\times\cdots\times (a_d,b_d] : 0\leq a_k < b_k\leq 1 \bigr\rangle.
\]
\end{definition}

This $\sigma$-field was the main object  we studied in the first third of the class. In particular, we proved  the SLLN and the law of the iterated logarithm using the uniform probability measure (i.e.\! $1$-d volume) which we proved existed by defining it on $\mathcal B_0^{(0,1]}$ then extending with Carath\'eodory.  
The Borel $\sigma$-field is very rich and can be generated by many different sub-classes. In particular 
\begin{align*}
\mathcal B^{(0,1]} 
&=\sigma\langle \mathcal B_0^{(0,1]}\rangle \\
&= \sigma\bigl\langle (a,b]: 0\leq a \leq b \leq 1 \bigr\rangle \\
  & =\sigma\bigl\langle (a,b): 0< a < b <1  \bigr\rangle \\
  & =\sigma\bigl\langle [a,b]: 0< a < b <1  \bigr\rangle \\
    & =\sigma\bigl\langle (0,a]: 0< a  <1  \bigr\rangle \\
& = \sigma\bigl\langle \text{open subsets of $(0,1]$}  \bigr\rangle  \\
& = \sigma\bigl\langle \text{closed subsets of $(0,1]$}  \bigr\rangle.
\end{align*}
All of the above equalities are shown using the good sets principle. In particular, to show that $\sigma\langle \mathcal A_1\rangle=\sigma\langle\mathcal A_2 \rangle$ one simply needs to establish that $ \mathcal A_1 \subset \sigma\langle \mathcal A_2\rangle$ (which implies that $ \sigma\langle \mathcal A_1\rangle \subset \sigma\langle \mathcal A_2\rangle$ by ``good sets") and $ \mathcal A_2 \subset \sigma\langle \mathcal A_1\rangle$ (which implies that $ \sigma\langle \mathcal A_2\rangle \subset \sigma\langle \mathcal A_1\rangle$ by ``good sets").

\begin{example}
To see an example lets show that 
\[ \sigma\bigl\langle (a,b]: 0< a < b < 1 \bigr\rangle =\sigma\bigl\langle (a,b): 0< a < b <1  \bigr\rangle.\] 
It will be sufficient to show the following two statements (i) and (ii) for any arbitrary $0<a_0<b_0<1$.

\noindent
\underline{(i) $(a_0,b_0]\in \sigma\langle (a,b):0<a<b<1\rangle$:}
This is follows from the identity
\[(a_0,b_0] = \bigcap_{n=1}^\infty (a_0,b_0+n^{-1} ).  \]


\noindent
\underline{(ii) $(a_0,b_0)\in \sigma\langle (a,b]:0<a<b<1\rangle$:}
This is follows from the identity
\[(a_0,b_0) = \bigcup_{n=1}^\infty (a_0,b_0-n^{-1}  ].  \]

\end{example}


%----------------------------
\subsection{Some interesting sets in $\mathcal B^{(0,1]}$}

\begin{example}
The set of normal and abnormal numbers are in $\mathcal B^{(0,1]}$.
\end{example}

\begin{example}
All countable subsets of $(0,1]$ and  all co-countable subsets of $(0,1]$ (i.e. the complements of countable sets) are in $\mathcal B^{(0,1]}$. In particular, the collection of  irrational numbers in $(0,1]$ is a Borel set.
\end{example}


\begin{example}
Under the spinner model we have
 \begin{align*}
 &\{ \limsup_n \frac{s_n}{\ell_n}=1  \}\\
 &\qquad=\bigcap_{\epsilon\in (0,1)\cap \Bbb Q} \{ {s_n/\ell_n}> (1-\epsilon) \text{  i.o.} \}\cap \{ {s_n/\ell_n}< (1+\epsilon) \text{  a.a.} \}
 \end{align*}
 where $\ell_n:=\sqrt{2n\log\log n}$.
 Therefore the law of the iterated logarithm event $\{w: \limsup_n \frac{s_n(w)}{\sqrt{2n\log\log n}}=1  \}$ is in $\mathcal B^{(0,1]}$.
\end{example}


\begin{example}
The Cantor set is in $\mathcal B^{(0,1]}$.  It is an instructive exercise to show that: the Cantor set is in $\mathcal B^{(0,1]}$;  that it has Lebesque measure $0$; and that it is uncountable (a nice way to see that it is uncountable is to work with a base-3 digit characterization of the Cantor set).
\end{example}





%%%%%%%%%%%%%%%%%%
%%%%%%%%%%%%%%%%%
\section{The Borel $\sigma$-field $\mathcal B^{\Bbb R^d}$}

Now we construct the Borel $\sigma$-field on the whole Euclidean space $\Bbb R^d$ which will be used to define Lebesque measure which generalizes the $d$-dimensional volume measure on $\mathcal B^{(0,1]^d}$.

\begin{definition} The Borel $\sigma$-field of $\Bbb R^d$, denoted $\mathcal B^{\Bbb R^d}$, is defined as the $\sigma$-field generated by the class of all finite rectangles in $\Bbb R^d$ as follows
\[ \sigma\bigl\langle (a_1,b_1]\times \cdots \times (a_d,b_d]: -\infty< a_k < b_k<\infty \bigr\rangle. \]
\end{definition}



%----------------------------
\subsection{How  $\mathcal B^{\Bbb R^d} $ is related to $\mathcal B^{(0,1]^d}$}
Since both classes of sets  $\mathcal B^{\Bbb R^d} $ and $\mathcal B^{(0,1]^d}$ contain subsets of $(0,1]^d$ it is natural to ask about  the nature of the overlap. In particular, if a set $A\in \mathcal B^{\Bbb R^d}$ and $A$ is also a subset of $(0,1]^d$ is it necessary that $A\in\mathcal B^{(0,1]^d}$ (and vise versa)? Another equality interesting question asks the following: if I take a set $A\in \mathcal B^{\Bbb R^d}$ and intersect it with $(0,1]^d$ is the result in $\mathcal B^{(0,1]^d}$ (and, if so, are all $\mathcal B^{(0,1]^d}$ sets structured in this manner)? It turns out the both answers are yes. To prove this we establish two  more general statements which yield our previous questions as corollaries.


\begin{claim}
\label{ccc}
Suppose $\mathcal F$ is a $\sigma$-field in $\Omega$ and let $\Omega_0$ be any subset of $\Omega$ (not necessarily in $\mathcal F$). Then 
\[\mathcal F\cap \Omega_0 := \{ F\cap \Omega_0: F\in \mathcal F\}\]
 is a $\sigma$-field of $\Omega_0$.  
\end{claim}


\begin{proof} It will be sufficient to show the following three facts (i), (ii), (iii):\\
\underline{(i) $\Omega_0\in \mathcal F\cap \Omega_0$:} 
To see why notice that $\Omega \in \mathcal F$. Therefore $\Omega\cap \Omega_0\in \mathcal F\cap \Omega_0$. Now since $\Omega_0\subset \Omega$ implies $\Omega\cap\Omega_0= \Omega_0$ we have that $\Omega_0\in \mathcal F\cap \Omega_0$ as was to be shown.

\noindent
\underline{(ii) $A\in \mathcal F\cap \Omega_0\Rightarrow (\Omega_0 - A)\in \mathcal F\cap \Omega_0$:} Let $A\in \mathcal F\cap \Omega_0$. Then $A = B\cap \Omega_0$ for some $B\in\mathcal F$. Therefore letting $A^c$ denote complementation within the larger space $\Omega$ we have
\begin{align*}
\Omega_0-A & = \Omega_0 \cap A^c\\
& =\Omega_0 \cap (B^c \cup \Omega_0^c)\\
& =\underbrace{\Omega_0 \cap B^c.}_{\in\mathcal F\cap \Omega_0}
\end{align*}

\noindent
\underline{(iii) $A_1,A_2,\ldots \in \mathcal F\cap \Omega_0\Rightarrow \cup_{k=1}^\infty A_k\in \mathcal F\cap \Omega_0$:} Notice that each $A_k = B_k\cap \Omega_0$ for some $B_k\in\mathcal F$. Therefore
\begin{align*}
\bigcup_{k=1}^\infty A_k &=\bigcup_{k=1}^\infty (B_k\cap \Omega_0)\\
&=\Omega_0\cap \underbrace{\bigcup_{k=1}^\infty B_k}_{\in \mathcal F}
\end{align*}
Therefore $\bigcup_{k=1}^\infty A_k \in \mathcal F\cap \Omega_0$.
\end{proof}


\begin{claim} 
\label{restricTHM}
Let $\Omega$ be a sample space and  $\mathcal A$ be a class of subsets of $\Omega$.  If $\Omega_0\subset \Omega$ then
\[
\underbrace{\sigma\bigl\langle  \mathcal A \cap \Omega_0 \bigr\rangle}_{\shortstack{ \text{\small $\sigma$-field} \\ \text{\small on $\Omega_0$}}}= \underbrace{\sigma\bigl\langle \mathcal A \bigr\rangle}_{\shortstack{ \text{\small $\sigma$-field} \\ \text{\small on $\Omega$}}} \cap\, \Omega_0.
\]
\begin{proof}
We first show that $\sigma \langle  \mathcal A \cap \Omega_0 \rangle \subset \sigma \langle \mathcal A  \rangle \cap\, \Omega_0.$ This easily follows by ``good sets"  since clearly $\mathcal A\cap \Omega_0\subset \sigma\langle \mathcal A\rangle\cap \,\Omega_0$ and  Claim \ref{ccc} shows that $\sigma \langle \mathcal A  \rangle \cap\, \Omega_0$ is a $\sigma$-field.

Now we show  $\sigma\langle \mathcal A \rangle \cap\, \Omega_0 \subset \sigma \langle  \mathcal A \cap \Omega_0 \rangle $.  Notice that this inclusion is 
equivalent to the statement that for every $A\in \sigma\langle\mathcal A\rangle$, $A\cap \Omega_0\in \sigma\langle\mathcal A\cap \Omega_0\rangle.$  To show this  let 
\[ \mathcal G:=\{ A\subset \Omega: A\cap \Omega_0 \in \sigma\langle \mathcal A \cap \Omega_0 \rangle \}. \]
It will then be sufficient to show (i)-(iv) below and then use good sets to conclude $\sigma\langle\mathcal A \rangle\subset \mathcal G$.
\\
\noindent
\underline{(i)  $\mathcal A\subset \mathcal G$}: This follows since for any $A\in \mathcal A$ one has that 
\[A\cap \Omega_0\in \mathcal A \cap \Omega_0 \subset \sigma\langle \mathcal A \cap \Omega_0 \rangle.  \]
 
  \vspace{.2cm}
\noindent
\underline{(ii) $\Omega \in \mathcal G$ }: This follows since $\Omega_0\subset \Omega$ implies that 
\[\Omega \cap \Omega_0 = \Omega_0 \in \sigma\langle \mathcal A\cap \Omega_0\rangle\]
where the last inclusion follows since $\sigma\langle \mathcal A\cap \Omega_0\rangle$ is a $\sigma$-field of $\Omega_0$ so it must contain $\Omega_0$.

  \vspace{.2cm}
\noindent
\underline{(iii) $A\in \mathcal G\Rightarrow A^c \in \mathcal G$ }: Notice that $A^c$ here denotes complementation within $\Omega$. This axiom follows since 
\begin{align*}
A\in\mathcal G&\Rightarrow A\cap \Omega_0\in \sigma\langle \mathcal A\cap \Omega_0\rangle \\
&\Rightarrow \underbrace{\Omega_0-A\cap \Omega_0}_\text{\footnotesize complement in $\Omega_0$}\in \sigma\langle \mathcal A\cap \Omega_0\rangle \\
&\Rightarrow \underbrace{\Omega_0\cap (A^c\cup \Omega_0^c)}_{= A^c\cap\Omega_0}\in \sigma\langle \mathcal A\cap \Omega_0\rangle \\
&\Rightarrow A^c\in\mathcal G.
\end{align*}


  \vspace{.2cm}
\noindent
\underline{(iv) $A_1,A_2,\ldots \in \mathcal G \Rightarrow \bigcup_k A_k\in \mathcal G$ }:
\begin{align*}
A_1,A_2,\ldots \in \mathcal G&\Rightarrow A_k\cap \Omega_0 \in \sigma\langle \mathcal A\cap \Omega_0\rangle,\,\forall k \\
&\Rightarrow  \bigcup_k (A_k\cap \Omega_0) \in\sigma\langle \mathcal A\cap \Omega_0\rangle \\
 &\Rightarrow  \Bigl(\bigcup_k A_k\Bigr)\cap \Omega_0 \in\sigma\langle \mathcal A\cap \Omega_0\rangle \\
&\Rightarrow  \bigcup_k A_k\in \mathcal G.
\end{align*}


\end{proof}



The above two claims immediately yield the desired corollary. 
\begin{corollary} $\mathcal B^{\Bbb R^d}\cap (0,1]^d =\mathcal B^{(0,1]^d} = \{ A\in\mathcal B^{\Bbb R^d}: A\subset (0,1]^d\}$
\end{corollary}
\begin{proof}
The first equality follows since 
\begin{align*}
\mathcal B^{\Bbb R^d}\cap (0,1]^d &= \sigma\langle \textit{\small finite rectangles in $\Bbb R^d$}\rangle \cap (0,1]^d \\
&= \sigma\langle \textit{\small finite rectangles in $\Bbb R^d \cap (0,1]^d$}\rangle  \\
&=\mathcal B^{(0,1]^d} .
\end{align*}


The second equality follows by noticing \mbox{ $ \{ A\in\mathcal B^{\Bbb R^d}: A\subset (0,1]^d\}=\mathcal B^{\Bbb R^d}\cap (0,1]^d$}. The inclusion `$\subset$' is obvious. The other inclusion is also almost obvious since
\begin{align*}
 A\in \mathcal B^{\Bbb R^d}\cap (0,1]^d&\Rightarrow \text{$A=B\cap (0,1]^d$ for some $B\in\mathcal B^{\Bbb R^d}$}\\
 &\Rightarrow \text{$A\in \mathcal B^{\Bbb R^d}$  (since $(0,1]^d$ and $ B$ are in $\mathcal B^{\Bbb R^d}$)}\\
 &\qquad\text{ and $A\subset (0,1]^d$  }\\
  &\Rightarrow  A \in \{ A\in\mathcal B^{\Bbb R^d}: A\subset (0,1]^d\}.
\end{align*}
\end{proof}


\end{claim}






%----------------------------
\subsection{Equivalent generators of $\mathcal B^{\Bbb R^d}$}
\label{equiv}

\begin{claim}
\begin{align}
\mathcal B^{\Bbb R^d}
& = \sigma\bigl\langle (-\infty,c_1]\times \cdots \times (-\infty, c_d]: -\infty < c_k< \infty \bigr\rangle \label{rrr}   \\
& = \sigma\bigl\langle \text{open subsets of $\Bbb R^d$}  \bigr\rangle  \\
& = \sigma\bigl\langle \text{closed subsets of $\Bbb R^d$}  \bigr\rangle\\
& = \sigma\bigl\langle \text{compact subsets of $\Bbb R^d$}  \bigr\rangle\\
&= \sigma \Bigl\langle f\langle \textit{finite rectangles of $\Bbb R^d$} \rangle\Bigr\rangle.
\end{align}
\end{claim}
\begin{proof}
 We only show that $\mathcal B^{\Bbb R^d} =  \sigma\bigl\langle \text{open subsets}\bigr\rangle$ and simply remark that the other proofs are similar. It will be sufficient to show the following two statements (i) and (ii):
 
 \vspace{.2cm}
\noindent
\underline{(i) the open sets are in $\mathcal B^{\Bbb R^d}$:} 
Let $G$ be an open subset of $\Bbb R^d$. Now for each element $y\in G$ there exists a finite rectangle within $G$, call it $R_y$, with rational edges which covers $y$ (i.e. $y\in R_y$ and $R_y\subset G$). Then clearly $G= \cup_{y\in G} R_y$. Notice that since there are only countably many rational rectangles the union $ \cup_{y\in G} R_y$ must be a countable union of the generators of $\Bbb R^d$ and therefore $G\in \mathcal B^{\Bbb R^d}$.

\vspace{.2cm}
\noindent
\underline{(ii) the finite rectangles are in $\sigma\bigl\langle \text{open sets}  \bigr\rangle$:} 
 This follows easily since 
\[ (a_1,b_1]\times \cdots \times (a_d,b_d] = \bigcap_{n=1}^\infty  (a_1,b_1+n^{-1})\times \cdots \times (a_d,b_d+n^{-1}). \]
\end{proof}

It is worth while mentioning that while some of these characterization of $\mathcal B^{\Bbb R^d}$ are useful (for example the left infinite generators in (\ref{rrr}) form a $\pi$-system so any $\sigma$-finite measure is uniquely specified on these generators) I would suggest that the most important characterization is $\mathcal B^{\Bbb R^d}=\sigma \langle f\langle \textit{finite rectangles} \rangle\rangle$. The reason being is that one can use the extension theorem to uniquely extend a $\sigma$-finite measure on $f\langle  \textit{finite rectangles} \rangle$ to one on $\sigma \langle f\langle  \textit{finite rectangles} \rangle\rangle$. Moreover, we have an explicit understanding of the elements in $f\langle  \textit{finite rectangles} \rangle$ which makes it relatively easy to specify a measure.  Be careful, though, the structure of $f\langle \textit{finite rectangles in $\Bbb R^d$}\rangle$ is a bit different than $f\langle \textit{finite rectangles in $(0,1]^d$}\rangle$
%
%\begin{exercise}
%\label{ex1}
%Show that $R_0^1:=f\langle \textit{finite rectangles in $\Bbb R$}\rangle = f\langle (a,b]: -\infty< a< b<\infty\rangle$  is the the set of finite disjoint unions of intervals of the form $(-\infty, b]$, $(a,\infty)$ and $(a,b]$ for finite $a\leq b$. (Hint: in a previous exercise you showed that for any class $\mathcal A$ in $\Omega$, then when defining
%\begin{align}
%\mathcal C &:= \textit{the collection of $\mathcal A$ sets and their complements}\\
%\mathcal I  &:= \textit{the collection of finite intersections of $\mathcal C$ sets} \\
%\mathcal U &:=\textit{the  collection of finite unions of $\mathcal U$ sets}.
%\end{align}
%one has that $f\langle\mathcal A\rangle = \mathcal U$. Use this to prove the exercise.
%\end{exercise}
%
%----------------------------
%\subsection{Examples of sets in $\mathcal R^d$}
%
%\begin{example}
%As with the ... the one point subsets, the countable, and co-countable sets are in here. 
%Also all perfect sets.
%\end{example}
%
%\begin{example}
%There are also some very weird sets in here... For example... the cantor set is in here.
%\end{example}
%
%\begin{example}
%All of the above examples are shown to be elements of $\mathcal R^d$ by applying a countable number of the intersections, unions or complelents of the generators of $\mathcal R^d$. However, as we should be familar with  by not, there is more in $\mathcal R^d$. Here we give an example of a set in $\mathcal R^d$, which can't be obtained by a countable sequence of set operations of the generators. See page 30 of Billingsly.
%\end{example}
%
%

%----------------------------
%\subsection{Other miscellaneous facts}


%%%%%%%%%%%%%%%%%%
%%%%%%%%%%%%%%%%%
\section{Lebesque measure $\mathcal L^d$}
%Lebesque measure on $(\Bbb R^d, \mathcal B^{\Bbb R^d})$ , denoted  $\mathcal L^d$, can be considered a generalization of the uniform probability measure (i.e. $d$) which we constructed on $(0,1]^d$. Indeed, one method of  constructing  $\mathcal L^d$  is to stitch together uniform probability measures on translated unit cubes in $\Bbb R^d$. In this section we give the details of this construction and prove some useful facts about $\mathcal L^d$.


For any $\bs i=(i_1,\ldots, i_d)\in\Bbb Z^d$ let $(\bs i,\bs i+1]$ be the unit cube in $\Bbb R^d$ translated up by $\bs i$ so that
\[(\bs i,\bs i+1] \equiv (i_1, i_1+1]\times \cdots \times (i_d,i_d + 1].  \] 
Notice that these sets give a checker board decomposition, $\Bbb R^d=\bigcup_{\bs i\in\Bbb Z^d} (\bs i,\bs i+1] $, so that $\Bbb R^d$ is expressed as a countable disjoint union of the translated unit cubes. Let $\mathcal B_{0}^{(\bs i,\bs i+1] }$ denote the field of finite disjoint unions of rectangles in $(\bs i,\bs i+1] $ and let $\mathcal B^{(\bs i,\bs i+1] }\equiv \sigma \langle \mathcal B_{0}^{(\bs i,\bs i+1] }\rangle $ denote the Borel $\sigma$-field of  $(\bs i,\bs i+1]$. Finally let $P_{\bs i}$ denote the unique uniform probability measure on $\mathcal B^{(\bs i,\bs i+1] }$ which assigns Euclidean volume  to the rectangles in  $(\bs i,\bs i+1] $, i.e.
\[ P_{\bs i}\bigl( (a_1,b_1]\times \cdots \times (a_d,b_d]\bigr)=\prod_{k=1}^d (b_k - a_k) \]
whenever $(a_1,b_1]\times \cdots \times (a_d,b_d]\subset (\bs i,\bs i+1]$. The construction of $P_{\bs i}$ is done in exactly the same way as the uniform probability measure was constructed on $(0,1]$ in the beginning of the class. Lets recall how this is done. One first shows that for any $A\in \mathcal \mathcal B_{0}^{(\bs i,\bs i+1] }$ one can define $P_{\bs i}(A)$ to be the sum  of the disjoint rectangle volumes which make up $A$ (this is not trivial since there are different decompositions of $A$ into disjoint rectangles, but one can use a result similar to Theorem 1.3 of Billingsley to prove that $P_{\bs i}$ is well defined).  Secondly, one shows that $P_{\bs i}$ is a probability measure on $((\bs i, \bs i +1],\mathcal B_{0}^{(\bs i,\bs i+1] })$. The hard part of this step is to  show the countable additivity. For $(0,1]$ we used the equivalent condition that  $P_{\bs i}$ is continuous from above at $\varnothing$. This argument carries over to $((\bs i, \bs i +1],\mathcal B_{0}^{(\bs i,\bs i+1] }, P_{\bs i})$. Finally one invokes the Carath\'eodory Extension theorem to get a uniform probability measure $((\bs i, \bs i +1],\mathcal B^{(\bs i,\bs i+1] }, P_{\bs i})$ (uniqueness follows by the fact that rectangles, including the empty ones, form a $\pi$-system).

Now, using the uniform probability measures $((\bs i, \bs i +1],\mathcal B^{(\bs i,\bs i+1] }, P_{\bs i})$ we can define Lebesque measure $\mathcal L^d$ on sets $A\in\mathcal B^{(\bs i,\bs i+1] }$ by stitching these $P_{\bs i}$ together as follows
\begin{equation}
\label{Lm}
 \mathcal L^d(A):= \sum_{\bs i\in\Bbb Z^d} P_{\bs i}\bigl( (\bs i, \bs i+1]\cap A\bigr). 
 \end{equation}
Notice that each $ (\bs i, \bs i+1]\cap A$ is in the Borel $\sigma$-field $\mathcal B^{(\bs i,\bs i+1] }$ by Claim \ref{restricTHM} so that 
$P_{\bs i}\bigl( (\bs i, \bs i+1]\cap A\bigr)$ is defined.
Lets see that $\mathcal L^d$ is indeed a measure on $(\Bbb R^d,\mathcal B^{\Bbb R^d })$.
\begin{claim}
 $\mathcal L^d$  is a measure on $(\Bbb R^d,\mathcal B^{\Bbb R^d })$.
\end{claim}
\begin{proof}
We show the following three axioms (i), (ii) and (iii):\\
\noindent
\underline{(i)  $\mathcal L^d(A)\in [0,\infty]$}: Trivial.
 
 
  \vspace{.2cm}
\noindent
\underline{(ii)  $\mathcal L^d(\varnothing)=0$}: This is also easy since $P_{\bs i}\bigl( (\bs i, \bs i+1]\cap \varnothing\bigr)=0$.


  \vspace{.2cm}
\noindent
\underline{(iii) Countable additivity}: Suppose $A_1,A_2,\ldots\in\mathcal B^{\Bbb R^d }$ are disjoint. Then 
\begin{align}
\mathcal L^d\Bigl (\bigcup_{k=1}^\infty A_k\Bigr)&=\sum_{\bs i\in\Bbb Z^d} P_{\bs i}\Bigl((\bs i, \bs i+1] \cap \bigcup_{k=1}^\infty A_k \Bigr) \nonumber\\
&=\sum_{\bs i\in\Bbb Z^d} P_{\bs i}\Bigl( \bigcup_{k=1}^\infty (\bs i, \bs i+1] \cap A_k \Bigr) \nonumber\\
&=\sum_{\bs i\in\Bbb Z^d} \sum_{k=1}^\infty P_{\bs i} \Bigl((\bs i, \bs i+1] \cap A_k \Bigr)\label{se} \\
&=\sum_{k=1}^\infty  \sum_{\bs i\in\Bbb Z^d} P_{\bs i} \Bigl((\bs i, \bs i+1] \cap A_k \Bigr)\label{se2} \\
&=\sum_{k=1}^\infty  \mathcal L^d\bigl (A_k \bigr) \nonumber
\end{align}
where (\ref{se}) follows since $P_{\bs i}$ is countably additive and the $ (\bs i, \bs i+1]\cap A_k $'s are disjoint; and (\ref{se2}) follows from general results about positive iterated sums.
\end{proof}


%----------------------------
\subsection{Uniqueness}
Recall the uniqueness theorem for $\sigma$-finite measures
\begin{claim}
\label{ui}
If $\mu_1$ and $\mu_2$ are measures on $(\Omega, \sigma\langle \mathcal P\rangle)$ such that 
\begin{itemize}
\item[(a)] $\mu_1$ and $\mu_2$ agree on $\mathcal P$;
\item[(b)] $\mathcal P$ is a $\pi$-system;
\item[(c)] $\mu_1$ and $\mu_2$ are $\sigma$-finite on $\mathcal P$,
\end{itemize}
then $\mu_1$ and $\mu_2$ agree on all of $ \sigma\langle \mathcal P\rangle$.
\end{claim}
\begin{proof}
See the class notes for a proof.
\end{proof}

To apply this to Lebesque measure define $\mathcal P$ to be the $\pi$-system composed of the finite rectangles $\{ (a_1,b_1]\times \cdots \times (a_d,b_d]: -\infty < a_k < b_k <\infty\}$ and the empty set $\varnothing$. Notice that using methods discussion in Section \ref{equiv} one can easily establish that $\mathcal B^{\Bbb R^d}=\sigma\langle\mathcal P\rangle$.  
 Also notice that $\mathcal L^d$ is $\sigma$-finite on $\mathcal P$ since  $\mathcal L^d\bigl((\bs i,\bs i+1])=1$, $\Bbb R^d = \cup_{\bs i\in\Bbb Z^d} (\bs i,\bs i+1]$ and each $(\bs i,\bs i+1]\in \mathcal P$.
 Therefore Claim \ref{ui} establishes the following claim
 
 \begin{corollary}
 \label{ui2}
  $\mathcal L^d$ is the only measure on $(\Bbb R^d,\mathcal B^{(0,1]^d})$ which assigns standard Euclidean volume to the finite rectangles as follows
 \begin{equation}
\mathcal L^d \bigl( (a_1,b_1]\times \cdots \times (a_d,b_d]\bigr)=\prod_{k=1}^d (b_k - a_k)
\end{equation}
for $-\infty < a_k < b_k <\infty$.
\end{corollary}
 

%----------------------------
\subsection{Behavior under linear transformations}

Lebesque measure $\mathcal L^d$ can be thought of as a uniform measure on $\Bbb R^d$. In this section we show some facts about $\mathcal L^d$ which one would expect from a uniform measure.
\begin{claim}
For any $A\in\mathcal B^{\Bbb R^d }$ and $x\in \Bbb R^d$, the set $A+x:= \{ a+x: a\in A\}$ is in $\mathcal B^{\Bbb R^d }$ and
\begin{equation}
 \label{translate}
 \mathcal L^d(A+x) =  \mathcal L^d(A) 
 \end{equation}
\end{claim}
\begin{proof}
To show $A+x \in\mathcal B^{\Bbb R^d }$ use the good sets principle. Fix $x\in \Bbb R^d$ and set   $\mathcal G_x:=\{ A\subset \mathcal B^{\Bbb R^d }: A\in \mathcal B^{\Bbb R^d } \text{  and } A+x \in \mathcal B^{\Bbb R^d }\}$. 
It is easy to see that $\mathcal G_x$ is a $\sigma$-field since complementation and union is preserved under translation by $x$. For example,
\begin{align*}
A\in\mathcal G_x &\Rightarrow  A\in \mathcal B^{\Bbb R^d } \text{  and } A+x \in \mathcal B^{\Bbb R^d } \\
&\Rightarrow  A^c\in \mathcal B^{\Bbb R^d } \text{  and } (A+x)^c \in \mathcal B^{\Bbb R^d } \\
&\Rightarrow  A^c\in \mathcal B^{\Bbb R^d } \text{  and } A^c+x \in \mathcal B^{\Bbb R^d } \\
&\Rightarrow A^c\in\mathcal G_x.
\end{align*}
The other axioms are established in a similar fashion.
Moreover, clearly all the finite rectangles are in $\mathcal G_x$. Therefore good sets implies $\mathcal B^{\Bbb R^d } \subset \mathcal G_x$  which implies $A\in \mathcal B^{\Bbb R^d }\rightarrow A+x \in \mathcal B^{\Bbb R^d }$, as was to be shown.

Now to show (\ref{translate}) one can simply use the same arguments used in the Claim \ref{ui2} on the uniqueness of $\mathcal L^d$. In particular, fix $x$ and define $\mu_x(A):= \mathcal L^d(A+x)$. It is easy to show that $\mu_x$ is a measure on $(\Bbb R^d, \mathcal B^{\Bbb R^d })$. Moreover, since the volume of any rectangle in $\Bbb R^d$ is invariant under translation by $x$, the measures $\mu_x$ and $\mathcal L^d$ both agree on the $\pi$-system of finite, possibly empty, rectangles in $\Bbb R^d$. Since they are also both $\sigma$-finite on these rectangles one must have, by Claim \ref{ui}, $\mathcal L^d(A) =\mu_x(A):= \mathcal L^d(A+x) $ for all $A\in\mathcal B^{\Bbb R^d }$, as was to be shown.
\end{proof}

The above claim illustrates how $\mathcal L^d$ behaves under translation of sets. The following theorem shows how $\mathcal L^d$ behaves under nonsingular linear transformations
\begin{claim}
If $T:\Bbb R^d\rightarrow \Bbb R^d$ is linear and nonsingular, then $A\in\mathcal B^{\Bbb R^d }$ implies that $T\!A:=\{ T(a): a\in A\}\in \mathcal B^{\Bbb R^d }$ and 
\[ \mathcal L^d(T\!A):=|\det T |\mathcal L^d(A). \]
\end{claim}
\begin{proof}
This proof is similar to the previous proof, albeit a bit more tedious. See Theorem 12.2 in Billingsley (page 173) the details.
\end{proof}
It is interesting to know that the assumption that $T:\Bbb R^d\rightarrow \Bbb R^d$ be nonsingular is necessary for the Borel measurability of $T\!A$. In particular, there exists a linear singular map $T:\Bbb R^2\rightarrow \Bbb R^2$ and a Borel measurable set $A\in \mathcal B^{\Bbb R^2}$ such that $T\!A\notin \mathcal B^{\Bbb R}$!

%------------
\subsection{Lower dimensional subsets}
In this section we show that $\mathcal L^d$ assigns zero measure to low dimensional hyperplanes. In fact, this will be a consequence of the following more general theorem in combination with the translation invariance of $\mathcal L^d$. 

\begin{claim}
\label{cc}
Let $(\Omega, \mathcal F,\mu)$ be a $\sigma$-finite measure space. Then $\mathcal F$ cannot contain an uncountable, disjoint collection of sets of positive $\mu$-measure
\end{claim}
\begin{proof}
Let $\{B_i: i\in \mathcal I\}$ a disjoint collection of sets of such that $\mu(B_i)>0$ for each $i\in\mathcal I$. 
We show $\mathcal I$ must be countable.


Since $\mu$ is $\sigma$-finite there exists $A_1,A_2,\ldots \in\mathcal F$ such that $\mu(A_k)<\infty$ and $\Omega = \cup_k A_k$. 

We show the following three facts.\\
\noindent
\underline{(i) $\{ i\in \mathcal I: \mu(A_k\cap B_i)>\epsilon\}$ is finite for all $k$}:
Let $\epsilon>0$ and suppose by contradiction one can find a countably infinite set  $\mathcal I_c\subset \mathcal I$ such that $\mu(A_k\cap B_i)>\epsilon $ for all $i\in \mathcal I_c$ and for this set of indices one has 
\begin{align*}
\mu(A_k)&\geq \mu(A_k\cap (\cup_{i\in\mathcal I_c}B_i ))= \sum_{i\in\mathcal I_c}\mu(A_k\cap B_i) > \sum_{i\in\mathcal I_c}\epsilon =\infty
\end{align*}
which gives a contradiction. 


  \vspace{.2cm}
\noindent
\underline{(ii) $\{ i\in \mathcal I: \mu(A_k\cap B_i)>0\}$ is countable for all $k$}:
This follows from the  identity
\[\{ i\in \mathcal I: \mu(A_k\cap B_i)>0\}=\bigcup_\text{\small rational 
$\epsilon$} \underbrace{\{ i\in \mathcal I: \mu(A\cap B_i)>\epsilon\}.} _\text{\small finite by (i)}   \]


 \vspace{.2cm}
\noindent
\underline{(iii) $\mathcal I = \bigcup_k \{ i\in \mathcal I: \mu(A_k\cap B_i)>0\}$}:
 To show $\mathcal I \cup \bigcup_k \{ i\in \mathcal I: \mu(A_k\cap B_i)>0\}$ notice that  if $i\in\mathcal I$ then $\mu(B_i)>0$. Now  $\Omega =\cup_k A_k$ so there must exist a $k$ such that $\mu(A_k\cap B_i)>0$.  Therefore $i\in  \bigcup_k \{ i\in \mathcal I: \mu(A_k\cap B_i)>0\}$. The other inclusion is obvious.
 
 To finish the proof simply notice that (ii) and (iii) imply $\mathcal I$ is countable. 
\end{proof}

\begin{corollary}If $k<d$ then
$\mathcal L^d(A)=0$ for any $k$-dimensional hyperplane $A\subset \Bbb R^d$ where $k<d$.
\end{corollary}
\begin{proof} 
Let $A$ be a $k$-dimensional hyperplane where $k<d$. Let $x$ be a point in $\Bbb R^d$ which is not  contained in $A$.
Then $\{A+xt: t\in\Bbb R\}$ is an uncountable class of disjoint subsets of $\mathcal B^{\Bbb R^d}$. Since $\mathcal L^d$ is translation invariance $\mathcal L^d(A) = \mathcal L^d(A+xt)$ for each $t\in\Bbb R$. Now by Claim \ref{cc}, $\mathcal L^d(A)=0$, for otherwise there would exists a uncountable, disjoint collection of sets of positive $\mathcal L^d$-measure.
\end{proof}


%----------------------------
\subsection{Regularity and Approximation}

%Remember, by the definition of the outer measure and by results ... we have that 
%\begin{align*}
%\mathcal L^d(B) &= \inf \{ \mathcal L^d(A):  B\subset A\in\mathcal F^\uparrow\}\\
%&= \inf \{ \sum_{k=1}^\infty \mathcal L^d(A_k):  B\subset \cup_{k=1}^\infty A_k, A_k\in \mathcal F_0\}\\
%&=\inf \{ \mathcal L^d(A):  B\subset A\in\mathcal F\} 
%\end{align*}
If $B\in\mathcal B^{\Bbb R^d}$  then
\begin{align*}
\mathcal L^d(B)  &= \sup \{ \mathcal L^d(C):  C\subset B,\, \text{$C$ closed}\}\\
  &=\, \inf \{ \mathcal L^d(O):  B\subset O,\, \text{$O$ open}\}
\end{align*}
%If, in addition $\mathcal L^d(B)<\infty$ then 
%\begin{align*}
%\mathcal L^d(B)  &= \sup \{ \mathcal L^d(K):  K\subset B,\, \text{$K$ compact}\} 
%\end{align*}



To prove these claims we need some results that hold for general measure.


\begin{lemma}
\label{l1}
 Suppose  $\mathcal F_0$ is a field, $\mu$ is a measure on $\mathcal F:=\sigma\langle \mathcal F_0\rangle$ and $\mu$ is $\sigma$-finite on $\mathcal F_0$. For all $B\in \mathcal F$ and $\epsilon>0$ there exists a disjoint sequence of $\mathcal F_0$-sets $A_1, A_2,\ldots$ such that $B\subset \cup_{n=1}^\infty A_n$ and $ \mu\bigl( \cup_{n=1}^\infty A_n - 
 B\bigr)\leq \epsilon$.
\end{lemma}


%%------------------------------ begin long version
\if\Ver\LongVer{ 
{\flushleft\textcolor{blue}{$\downarrow$---------begin long version---------}}\newline

\begin{proof} 

Since $\mu$ is $\sigma$-finite on $\mathcal F_0$ and $\mathcal F_0$ is a field, one can find disjoint $\mathcal F_0$-sets $F_1, F_2,\ldots$ such that $\Omega = \cup_{n=1}^\infty F_n$ and $\mu(F_n)<\infty$. We start by supposing $\mu(F_n)>0$ for all $n$ and show at the end of the proof to remove this assumption. 


Fix $B\in \mathcal F$.
Define  $ \mu_n(\cdot):= \frac{\mu(\cdot\cap F_n)}{\mu(F_n)} $ on $\mathcal F$. Notice that $\mu_n$ is a probability measure on $\mathcal F$ so that
\[ \mu_n(B)=\inf\{ \mu_n(D): B\subset D \in \mathcal F^{\uparrow}\}\leq 1. \]
Therefore one can find a $D_n\in \mathcal F^{\uparrow}$, covering $B$, such that
$B\subset D_n$ and
\begin{align}
\label{aaa}
\underbrace{\mu_n(D_n)-\mu_n(B)}_{=\mu_n(D_n - B)} &\leq \frac{\epsilon}{2^n \mu(F_n)}.
\end{align}
 Now define  $D :=\bigcup_{n=1}^\infty D_n\cap F_n$
and notice that 
\begin{align*}
 B\subset D_n, \,\forall n&\Rightarrow (B\cap F_n) \subset (D_n\cap F_n), \,\forall n \\
 &\Rightarrow \bigcup_{n=1}^\infty  (B\cap  F_n) \subset  \bigcup_{n=1}^\infty (D_n\cap F_n) \\
 &\Rightarrow B\subset D.
 \end{align*}
Notice also that each  $D_n\in\mathcal F^\uparrow$ can be written as a disjoint union of $\mathcal F_0$-sets (exercise). Let $D_n = \cup_{m=1}^\infty A_{n,m}$ be such a decomposition (i.e. the $A_{n,m}$'s are  disjoint across $m$ and $A_{n,m}\in\mathcal F_0$). Then 
\[D =  \bigcup_{(n,m)\in \Bbb N^+\times \Bbb N^+}  A_{n,m}\cap F_n\]
where the sets $A_{n,m}\cap F_n$ are disjoint (the $F_n$'s are disjoint for different $n$'s and  the $A_{n,m}$'s are disjoint for different $m$'s) and are $\mathcal F_0$-sets. Now
\begin{align}
\mu(D - B) &= \mu(\bigcup_{n=1}^\infty D_n\cap F_n \cap B^c)\nonumber\\
 &= \sum_{n=1}^\infty \mu(D_n\cap F_n \cap B^c)\nonumber\\
 &=  \sum_{n=1}^\infty \mu(F_n)\mu_n(D_n \cap B^c)\nonumber\\
  &=  \sum_{n=1}^\infty \mu(F_n)\underbrace{\mu_n(D_n - B)}_{\leq \epsilon/(2^n\mu(F_n))}\nonumber\\
 &\leq \epsilon.\label{if}
\end{align}
Therefore the class $\{ A_{n,m}\cap F_n\}_{(n,m)\in \Bbb N^+\times \Bbb N^+ }$ gives a countable, disjoint $\mathcal F_0$-set covering of $B$ such that $\mu(\bigcup_{(n,m)\in \Bbb N^+\times \Bbb N^+}  A_{n,m}\cap F_n - B )\leq \epsilon$. 

It's easy to extend to the case when some of the $\mu(F_n)=0$ by defining $\mu_n (\cdot):=0$ and $D_n:=F_n$ for these $n$. Then (\ref{if}) still follows.
\end{proof}

%
{\flushleft\textcolor{blue}{$\uparrow$------------end long version---------}}\newline
} \fi
%%------------------------------ end long version




\begin{exercise} 
\label{l2}
Let $\mu$ be any measure on $(\Bbb R^d,\mathcal B^{\Bbb R^d})$ which assigns finite measure to bounded sets in $\mathcal B^{\Bbb R^d}$.
Define the Borel field of $\Bbb R^d$ as follows.
\begin{align*} 
\mathcal B_0^{\Bbb R^d}:= f\bigl\langle (a_1,b_1]\times \cdots \times (a_d,b_d]: -\infty< a_k < b_k<\infty \bigr\rangle.
 \end{align*}
Show that for any  $A\in \mathcal B_0^{\Bbb R^d}$ and any $\epsilon>0$ there exists an open set $G$ such that $A\subset G$ and $\mu(G-A)\leq \epsilon$ 
(Hint:  Use a characterization of fields and simply make sure the boundaries of $A$ are finite).
\end{exercise}





\begin{claim} 
\label{oir}
Let $\mu$ be any measure on $(\Bbb R^d,\mathcal B^{\Bbb R^d})$ which assigns finite measure to bounded sets in $\mathcal B^{\Bbb R^d}$.
For any  $B\in\mathcal B^{\Bbb R^d}$ and $\epsilon>0$ there exists a closed set $C$ and an open set $O$ such that $C\subset B\subset O$ and 
\[\mathcal \mu(O-C)<\epsilon.\]
\end{claim}


%%------------------------------ begin long version
\if\Ver\LongVer{ 
{\flushleft\textcolor{blue}{$\downarrow$---------begin long version---------}}\newline



\begin{proof} This is a relatively easy consequence of Lemma \ref{l1} and Exercise \ref{l2}. 


First notice the following fact: if $A,B,C$ are $\mathcal F$-sets then
\begin{equation}
\label{trans}
 A\subset B\subset C \Rightarrow \mu(C-A)=\mu(C-B)+\mu(B-A). \end{equation}
Once consequence of this fact is that it will be sufficient to approximate $B$ by an open covering $O$ such that $\mathcal \mu(O-B)<\epsilon/2$ and by a closed subset $C$ such that $\mathcal \mu(B-C)<\epsilon/2$.

Now notice that $\mu$ is $\sigma$-finite on $\mathcal R_0^d$ 
since $\mu$ assigns finite measure to bounded sets in $\mathcal R^d$ (just use the covers $(\bs i,\bs i+1]$). Therefore we can use Lemma \ref{l1} to get a disjoint sets $A_k\in\mathcal R^d_0$ such that
\[ B\subset \bigcup_{k=1}^\infty A_k \text{ and } \mu\bigl( \bigcup_{k=1}^\infty A_k-B \bigr)\leq \epsilon/4. \]
Now, use Lemma \ref{l2} to expand $A_k$ to an open cover $G_k$ in such a way that $\mu(G_k-A_k)\leq \epsilon /(2^k 4)$. Now clearly $B\subset \bigcup_{k=1}^\infty A_k\subset  \bigcup_{k=1}^\infty G_k$ and 
\begin{align*}
\mu\bigl( \bigcup_{k=1}^\infty G_k-B \bigr) &= \mu\bigl(\bigcup_{k=1}^\infty G_k - \bigcup_{k=1}^\infty A_k \bigr)+ \mu\bigl( \bigcup_{k=1}^\infty A_k-B \bigr) \\
&\leq \sum_{k=1}^\infty \mu(G_k - A_k) + \epsilon/4\\
&\leq \epsilon/2.
\end{align*}
Since  $\bigcup_{k=1}^\infty G_k$ is open have constructed the desired open set $O$.

To show the existence of the closed set we simply take complements. In particular, let $O$ be an open set such that 
\[ B^c \subset O\text{ and } \mu(O - B^c)\leq \epsilon/2.\]
Notice that $O-B^c = O\cap B = B - O^c$  and that $O^c$ is a closed set. Therefore
\[ O^c \subset B\text{ and } \mu(B - O^c)\leq \epsilon/2\]
as was to be shown.
\end{proof}

%
{\flushleft\textcolor{blue}{$\uparrow$------------end long version---------}}\newline
} \fi
%%------------------------------ end long version




The following claim now gives our desired results for Lebesque measure.
\begin{corollary}
\label{ir}
Let $\mu$ be any measure on $(\Bbb R^d,\mathcal B^{\Bbb R^d})$ which assigns finite measure to bounded sets in $\mathcal B^{\Bbb R^d}$. Then
\begin{align*}
\mathcal \mu(B)  &= \sup \{ \mathcal \mu(C):  C\subset B,\, \text{$C$ closed}\}\\
  &=\, \inf \{ \mu(O):  B\subset O,\, \text{$O$ open}\}
\end{align*}
%If, in addition,  $\mathcal \mu(B)<\infty$ then 
%\[\mathcal \mu(B) = \sup \{ \mathcal \mu(K):  K\subset B,\, \text{$K$ compact}\} .\]
\end{corollary}


%%------------------------------ begin long version
\if\Ver\LongVer{ 
{\flushleft\textcolor{blue}{$\downarrow$---------begin long version---------}}\newline



\begin{proof}\textcolor{blue}{This would be a good exercise too!!}
We first show that $\mathcal \mu(B) = \inf \{ \mu(O):  B\subset O,\, \text{$O$ open}\}
$. Notice that this equality is immediately true when $\mu(B)=\infty$, since it implies $\mu(O)=\infty$ for any open cover of $B$.  Now suppose $\mu(B)<\infty$. 
Clearly 
\[\mathcal \mu(B) \leq \inf \{ \mu(O):  B\subset O,\, \text{$O$ open}\}\]
 since $\mu(B)\leq \mu(O)$ for each open cover $O$ of $B$. Conversely, Claim \ref{oir} shows that there exists open sets $O_n$ such that $B\subset O_n$ and $\mu(O_n-B) \leq n^{-1}$. Since $\mu(B)<\infty$ this implies that $0\leq \mu(O_n)-\mu(B)=\mu(O_n-B) \leq n^{-1}$ which implies that $\lim_{n\rightarrow \infty}\mu(O_n)= \mu(B)$ and therefore
 \[\mathcal \mu(B) \geq \inf \{ \mu(O):  B\subset O,\, \text{$O$ open}\}.\]

Now we show that $\mathcal \mu(B) = \sup \{ \mu(C): C\subset B,\, \text{$C$ closed}\}$. Again the inequality `$\geq$' is obvious. To show the other inequality let $C_n\subset B$ be closed sets such that $\mu(B-C_n)\leq n^{-1}$. Notice that if $\mu(B)=\infty$ then $\mu(C_n)=\infty$ and the inequality holds. On the other hand if $\mu(B)<\infty$, then $\mu(C_n)<\infty$ and we have that  $0\leq \mu(B)-\mu(C_n)\leq n^{-1}$. Therefore $\lim_{n\rightarrow \infty} \mu(C_n)=\mu(B)$ which implies
\[\mathcal \mu(B) \leq \sup \{ \mu(C): C\subset B,\, \text{$C$ closed}\}\]
as was to be shown.
 \end{proof}

%
{\flushleft\textcolor{blue}{$\uparrow$------------end long version---------}}\newline
} \fi
%%------------------------------ end long version



{\em Remark:} 
In fact, a measure $\mu$ is called inner-regular if $\mu(A)$  Claim (\ref{ir}) holds.


{\em Remark:} 
If one uses the semiring theory given in Billingsly (Theorem 11.4) one can strengthen this a bit and show that
if $B\in\mathcal B^{\Bbb R^d}$ and $\mathcal L^d(B)<\infty$ then 
\begin{align*}
\mathcal L^d(B)  &= \sup \{ \mathcal L^d(K):  K\subset B,\, \text{$K$ compact}\}.
\end{align*}

\begin{exercise}
Give an example of a $\sigma$-finite measure $\mu$ on $\mathcal B^{\Bbb R}$ and a Borel set $B$ such that 
\[ \mu(B-C)= \infty = \mu(O-B) \]
for every closed subset $C$ of $B$ and every open super set $O$ of $B$.
\end{exercise}

%%---------
%\subsection{Defining $\mathcal L^d$  on $\Bbb R^n$ when $d<n$}
%\textcolor{blue}{This seems too complex for this set of notes...I suggest skipping it al together.}
%I think way this is done is by defining $\mathcal L^d$ on a different $\sigma$-field on $\Bbb R^n$ which generates the lower dimensional sets. In particular, define the field $f\langle \textit{sticks of finite length} \rangle$ within $\Bbb R^n$. Then notice that  $f\langle \textit{sticks of finite length} \rangle$ is composed of finite disjoint unions of 1-d sticks of finite length (I think you need to consider all sticks with open and/or closed endpoints). Finally define $\mathcal L^1$ on $f\langle \textit{sticks of finite length} \rangle$ using Euclidean length. Now extend to $(\Bbb R^n,\sigma\langle \textit{sticks of finite length} \rangle)$. Notice that $\sigma\langle \textit{sticks of finite length} \rangle$ contains what we would consider to be the one-dimensional subsets of $\Bbb R^n$.
%\textcolor{blue}{This might be a good exercise}
%
%Later in the course we will learn how to integrate with respect to measures. Then, integrating with respect to $\mathcal L^1$  on $\Bbb R^n$  will correspond to line integrals.
%

%%%%%%%%%%%%%%%%%%
%%%%%%%%%%%%%%%%%
%\section{The completion $(\Bbb R^d,\bar{\mathcal R}^d,\bar{\mathcal L}^d)$}
%
%One can use the following theorem to establish the completion
%
%
%
%\begin{definition}[{\bf $\mu$-null and $\mu$-neg}]
%%$\phantom{asdf}$
%Let $(\Omega,\mathcal F, \mu)$ be a measure space. Then 
%%then
%\begin{itemize}
%\item 
%A set $A\in \mathcal F$ is said to be \underline{$\mu$-null}  if $\mu(A)=0$. 
%\item 
%A set $A\in 2^\Omega$ is said to be \underline{$\mu$-negligible} if there exists a $\mu$-null set  $B\in \mathcal F$ such that $A\subset B$.
%\end{itemize}
%\end{definition}
%
%
%
%\begin{definition}[{\bf Complete}]
%A measure space $(\Omega^\prime, \mathcal F^\prime, \mu^\prime)$ is said to be complete if all the $\mu^\prime$-negligible sets belong to $\mathcal F^\prime$.
%\end{definition}
%
%\begin{theorem}[{\bf The completion $(\Omega, \bar {\mathcal F},\bar \mu)$}]
%Let $(\Omega,\mathcal F, \mu)$ be a measure space and let $\mathcal N_\mu$ be the collection of $\mu$-negligible sets. 
%%Let $\bar {\mathcal F}:=\sigma\langle \mathcal F, \mathcal N_\mu\rangle$ and $\bar \mu$ 
%Then
%\begin{itemize}
%\item $\bar {\mathcal F}:= \sigma\langle \mathcal F, \mathcal N_\mu\rangle = \{F\cup N: F\in \mathcal F, N\in \mathcal N_\mu  \}$;
%\item The set function $\bar \mu$ on $\bar{\mathcal F}$ defined by $\bar\mu(F\cup N)= \mu(F)$ for $F\in\mathcal F$ and $N\in \mathcal N_\mu$ is the unique extension of $\mu$ to a measure on $(\Omega, \bar{\mathcal F})$;
%\item The measure space $(\Omega, \bar{\mathcal F}, \bar \mu)$ is complete.
%\end{itemize}
%The triple $(\Omega, \bar{\mathcal F}, \bar \mu)$ is called the \underline{completion} of  $(\Omega, \mathcal F, \mu)$.
%\end{theorem}
%
%
%%----------------------------
%\subsection{Sets in $\bar{\mathcal R}^d$ that are not already in $\mathcal R^d$}
%See exercise 15 on page 15 of Chung.
%
%
%%----------------------------
%\subsection{Sets not in $\bar{\mathcal R}^d$}
%Do the example of page 45 in Billingsly taken in combination with exercise 3.18 on page 50.
%
%You might also want to mention the Borel set in $\mathcal R^2$ but is not in  $\mathcal R^2$ when projected down to the $x$-axis (see page 174) of billing sly.
%
%%---------------
%\subsection{Impossibility theorem}
%Here we show that it is impossible to extend $\bar{\mathcal L}^d$ to all sets of $\Bbb R^d$ that is also translation invariant. Page 46 of billingsly.
%
%
%%%%%%%%%%%%%%%%%%%
%%%%%%%%%%%%%%%%%%
%\section{Hausdorff measure $\mathcal H^d$}
%The Hausdorff measure $\mathcal H^d$ is essentially $c(\mathcal L^{d})^*$ , i.e. a scaler constant of the outer measure extension to $\mathcal L^d$ (although we haven't technically defined out measures for anything other than probability measures). 
%
%One way to circumvent this difficulty is to define the Hausdorff measure by stitching together outer measures on $(i,i+1]$ and do an extension of this idea for higher dimension (with a re-scalign normalizing constant). 
%
%\subsection{$\mathcal H^d(A)=0$ implies $\bar{\mathcal L}^d(A)=0$}
%This is general fact about outer measure....
%
%\subsection{Hausdorff dimension} 
%



\end{document}



















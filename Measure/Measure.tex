

% It might be useful to give a historical motivation for this course. Was it Lebesgue who wanted to see if there existed a consistent definition of length for arbitrary subsets of $\Bbb R$. It turns out the answer is no. Any extension of regular lengths on intervals violates some natural measures of length like, shift invariance and additivity, for example. We'll see this later when we define Lebesgue measure. To start, however, lets prove Borel's normal number theorem which extends the notion of length to a strange set, the set of normal numbers in $(0,1]$. We can prove the theorem in a elementary way and the result hints at a deeper theory. Side note, Borel's original proof contained a mistake and it wasn't corrected till Hausdorff in 1914 [check the facts here].


%%%%%%%%%%%%%%%%%%%%%%%%%%%%%%%%%%%%%%%%%%
%
% Section
%
%%%%%%%%%%%%%%%%%%%%%%%%%%%%%
\section{Borel's normal number theorem}
\label{Bsec}




\begin{definition}[{\bf Borel field}]
\label{fb}
Let $\mathcal B_0((0,1])$ denote the class of finite (possibly empty) disjoint unions of intervals of the form  $(a,b]\subset (0,1]$.
\end{definition}

\begin{definition}
\label{l1defP}
Let $P$  be a probability assignment on $\mathcal B_0((0,1])$ such that $P[(a,b]]=b-a$ for all $0\leq a\leq b\leq 1$ and extended to all of $\mathcal B_0((0,1])$ using the identity $P[A\cup B]=P[A]+P[B]$ whenever $A,B$ are disjoint sets in $ \mathcal B_0((0,1])$.
\end{definition}

\begin{theorem}
$P$ is well defined.
\end{theorem}


\begin{definition}
For each $\omega\in(0,1]$, let $d_k(\omega)$ denote the $k^\text{th}$ nonterminating binary digit of $\omega$. Let $z_k(\omega):=2d_k(\omega)-1$ and $s_n(\omega):=\sum_{k=1}^nz_k(\omega)\equiv\text{\rm excess of heads in $n$ tosses}$.
\end{definition}

\begin{theorem}[{\bf WLLN}]
For all $\epsilon>0$,
\begin{equation}
\label{Borel's WLLN}
\lim_{n\rightarrow \infty}P\Bigl[\bigl\{\omega\in (0,1]: \left|{s_n(\omega)}/n \right|\geq\epsilon \bigr\} \Bigr] =0.
\end{equation}
\end{theorem}
\begin{proof}
Notice first that any event or bet based on the values of $z_1(\omega), \ldots, z_n(\omega)$ must be a disjoint union of dyadic intervals of the form $(\frac{k-1}{2^n}, \frac{k}{2^n}]$. Therefore $\bigl\{\omega\in (0,1]: \left|{s_n(\omega)}/n \right| \geq \epsilon \bigr\} \in \mathcal B_0((0,1])$ and the left hand side of (\ref{Borel's WLLN}) is well defined. Also notice
\begin{align*}
%\int_0^1 z_k(\omega)\, d\omega &= 0 \\
\int_0^1 z_k(\omega)z_j(\omega)\, d\omega
& = \begin{cases}
1 & \text{when $k=j$}\\
0 & \text{when $k\neq j$}.
\end{cases}
\end{align*}
This implies $\int_0^1 s^2_n(\omega)\, d\omega = \int_0^1 \sum_{k,j=1}^n z_k(\omega)z_j(\omega) \,d\omega = n$ which gives
\begin{align*}
n  &= \int_0^1 s^2_n(\omega)\, d\omega  \geq \int_{|s_n/n|\geq \epsilon} s^2_n(\omega)\, d\omega \\
& \qquad \geq \int_{|s_n/n|\geq \epsilon} n^2\epsilon^2\, d\omega  \geq n^2\epsilon^2 P\bigl[|s_n/n|\geq \epsilon\bigr]
\end{align*}
Therefore $P\bigl[|s_n/n|\geq \epsilon\bigr]\leq 1/(n\epsilon^2)\rightarrow 0$ as $n\rightarrow \infty$.
\end{proof}

\begin{definition}
The set of {\bf normal numbers} in $(0,1]$ is defined as
\begin{align*}
N
&:=\{\omega\in(0,1]: \lim_{n\rightarrow \infty} {s_n(\omega)}/n =0\}\\
&\phantom{:}=\{\omega\in(0,1]: \lim_{n\rightarrow \infty} \textstyle\frac{1}{n}\textstyle\sum_{k=1}^n d_k(\omega) = \textstyle\frac{1}{2}\}.
\end{align*}
The set of abnormal numbers is defined as $A:=(0,1]-N$.
\end{definition}


\begin{definition}[{\bf Negligible set}]
A subset $B\subset (0,1]$ is said to be {\bf negligible} if for all $\epsilon>0$, there exists $\mathcal B_0((0,1])$-sets $B_1, B_2,\ldots$ such that
\begin{align*}
&B\subset \bigcup_{k=1}^\infty B_k\quad\text{and}\quad\sum_{k=1}^\infty P[B_k]\leq \epsilon.
\end{align*}
\end{definition}

\begin{theorem}[{\bf Borel's normal number theorem, i.e. the SLLN for coin flips}]
\label{thm: Borel's normal number theorem}
The set of abnormal numbers, $A$,  is negligible.
\end{theorem}
\begin{proof}
Let $\epsilon_k\downarrow 0$ as $k\rightarrow \infty$. Then
\begin{align}
\{\omega \colon |\textstyle\frac{s_{k^2}(\omega)}{k^2}| < \epsilon_k&\, \text{ for all large $k$} \} \nonumber\\
&\subset  \{\omega\colon \textstyle\lim_{k}\textstyle\frac{s_{k^2}(\omega)}{k^2} = 0 \}  \nonumber\\
&\subset \underbrace{\{\omega\colon \textstyle\lim_{n}\textstyle\frac{s_{n}(\omega)}{n} = 0 \}}_{= N}  \label{subseq for N}
\end{align}
To see why (\ref{subseq for N}) holds assume $\lim_k s_{k^2}(\omega)/\text{\scriptsize $k^2$} = 0$ for  $\omega$ and notice that
\newcommand{\tempn}{\text{\scriptsize $\lfloor\sqrt{n}\rfloor^2$}}
\begin{align*}
 \left|\frac{s_n}{n} \right| = \frac{|s_n|}{ (\sqrt{n})^{2} }
 \leq \frac{|s_n|}{\tempn} &\leq \frac{s_{\tempn}}{\tempn} + \frac{|s_n - s_\tempn|}{\tempn} \\
 & \leq \frac{s_{\tempn}}{\tempn} + \sum_{k = \tempn +1}^n \frac{|z_k|}{\tempn} \\
 & = {\frac{s_{\tempn}}{\tempn}} + \frac{(n-\tempn)}{\tempn} \longrightarrow 0
 \end{align*}
since $n-\tempn \leq (\text{\scriptsize $\lfloor\sqrt{n}\rfloor$} + 1)^2 - \tempn = 1 + 2\lfloor\sqrt{n} \rfloor $. Now by (\ref{subseq for N}) we have that
\begin{align*}
A = N^c  &\subset \{\omega \colon |\textstyle\frac{s_{k^2}(\omega)}{k^2}| \geq \epsilon_k\, \text{ for infinitely many $k$} \} \\
&\subset \bigcup_{k=j}^\infty  \underbrace{\{\omega \colon  |\textstyle\frac{s_{k^2}(\omega)}{k^2}| \geq \epsilon_k \}}_{=: B_k},\quad\text{for any $j$}
\end{align*}
where $B_k\in B_0^{(0,1])}$. By the proof of the WLLN we have
\[
P[B_k]\leq \frac{1}{k^2 \epsilon^2_k} = \frac{1}{k^{3/2}}
\]
when $\epsilon_k:= k^{-1/4}$. Therefore $\sum_{k=1}^n P[B_k]<\infty$ and hence $\sum_{k=j}^n P[B_k]\rightarrow 0$ as $j\rightarrow \infty$. Hence $A$ is negligible.
\end{proof}


\begin{exercise}
Using just calculus and ideas from this section, show that
\begin{equation}
\label{ex1a}
 M(t):= \int_0^1 e^{ts_n(\omega)}\,d\omega = \Bigl(\frac{e^t+e^{-t}}{2} \Bigr)^n
 \end{equation}
for each $t\in\Bbb R$. By differentiating with respect to $t$, show that $\int_0^1 s_n(\omega) \,d\omega = M^\prime (0)=0$ and $\int_0^1 s_n^2(\omega)\, d\omega =M^{\prime\prime} (0)=n$.
\end{exercise}

\begin{exerciseproof}
The idea is to split the integral $\int_0^1$ into the dyadic intervals of the form $(\frac{k-1}{2^n},\frac{k}{2^n} ]$. On each one of these intervals $s_n(\omega)$ is constant. Moreover, these intervals are in one-to-one correspondence with all $n-$digit binary digits. Therefore the dyadic intervals of order $n$ are then in one-to-one correspondence with $\{(z_1,\ldots, z_n)\colon z_k\in\{ -1,1\} \}$. Since $s_n(\omega) = z_1(\omega)+\cdots+ z_n(\omega) $ we have
\begin{align*}
\int_0^1 e^{ts_n(\omega)}\,d\omega
& = \sum_{k=0}^{2^n}\int_{(\frac{k-1}{2^n},\frac{k}{2^n} ]} e^{ts_n(\omega)}\,d\omega \\
& = \sum_{z_1=1,-1}\cdots  \sum_{z_n=1,-1}\frac{1}{2^n} e^{tz_1} \cdots e^{tz_n} \\
& = \Bigl(\sum_{z_1=1,-1}\frac{e^{tz_1}}{2}\Bigr)\cdots  \Bigl(\sum_{z_n=1,-1}\frac{e^{tz_n}}{2}\Bigr)\\
& = \Bigl(\frac{e^t+e^{-t}}{2} \Bigr)^n .
\end{align*}
\end{exerciseproof}



\begin{exercise}
\label{exp ineq for sn}
 Show that
\[P\bigl[|s_n/n|\geq \epsilon\bigr]\leq 2 e^{-n\epsilon^2/2}  \]
for each $\epsilon>0$.

 Hint: %$s_n(w)\geq n\epsilon$ entails $e^{ts_n(w)}\geq e^{tn\epsilon}$ for each $t>0$.
 Use (\ref{ex1a}) in conjunction with the inequality $(e^x + e^{-x})/2\leq \exp(x^2/2)$ which holds (why?) for all $x\in\Bbb R$.
\end{exercise}


%-------------------------------'
%---------section  ---------------'
%-------------------------------'
\clearpage
\section{Classes of sets}
%-------------------------------'
%---------section  ---------------'
%-------------------------------'

%\subsection{Basic definitions}

\begin{definition}
$\Omega$ denotes the {\bf sample space}. Subsets of $\Omega$ are called {\bf events} and $2^\Omega$ denotes the power set of $\Omega$ (i.e. the class of all subsets of $\Omega$).
\end{definition}


\begin{definition}[{\bf field}] A collection of events $\mathcal F\subset 2^\Omega$  is a {\bf field} if
\begin{enumerate}
\item $\Omega\in \mathcal F$
\item $A\in\mathcal F\Longrightarrow A^c\in \mathcal F$
\item $A,B\in \mathcal F\Longrightarrow A\cup B\in \mathcal F$.
\end{enumerate}
\end{definition}



\begin{definition}[{\bf $\sigma$-field}]
A collection of events $\mathcal F\subset 2^\Omega$  is a {\bf $\sigma$-field} if
\begin{enumerate}
\item $\Omega\in \mathcal F$
\item $A\in\mathcal F\Longrightarrow A^c\in \mathcal F$
\item $A_1, A_2, \ldots \in \mathcal F\Longrightarrow \bigcup_{k=1}^\infty A_k \in \mathcal F$.
\end{enumerate}
\end{definition}




\begin{definition}[{\bf $\lambda$-system}]
A  collection of events $\mathcal F\subset 2^\Omega$ is called a {\bf $\lambda$-system} if
\begin{enumerate}
\item $\Omega\in \mathcal F$
\item $A\in\mathcal F\Longrightarrow A^c\in \mathcal F$
\item  $\underbrace{A_1, A_2, \ldots}_{\text{all disjoint}} \in \mathcal F \Longrightarrow \bigcup_{k=1}^\infty A_k \in \mathcal F$.
\end{enumerate}
\end{definition}

Notice that the only reason we require $\Omega\in \mathcal F$ in the definitions above is to force the class $\mathcal F$ be non-empty. We could just as well have changed the requirement $\Omega\in \mathcal F$ to the statment that there exists some $A\in \mathcal F$. One of the reasons it is traditional to put the assumption $\Omega\in \mathcal F$ is that the definition of a probability measure will require $P(\Omega)=1$. Therefore, it makes things more clear if we explicitly claim that $\Omega\in \mathcal F$, but otherwise its superfluous.


\begin{definition}[{\bf $\pi$-system}]
A  collection of events $\mathcal P\subset 2^\Omega$ is called a {\bf $\pi$-system} if
\begin{enumerate}
\item $A, B\in\mathcal P \Longrightarrow A\cap B\in\mathcal P$.
\end{enumerate}
\end{definition}




\begin{definition}[{\bf $A_n\uparrow A$}]
Let $A_1, A_2, \ldots$  and $A$ be events of $\Omega$. Then we write $\setlimup{n} A_n = A$ (or $A_n\uparrow A$) if
\begin{enumerate}
\item $A_1\subset A_2 \subset \cdots$
\item $A=\bigcup_{k=1}^\infty A_k$.
\end{enumerate}
\end{definition}

\begin{definition}[{\bf $A_n\downarrow A$}]
Let $A_1, A_2, \ldots$  and $A$ be events of $\Omega$. Then we write $\setlimdown{n} A_n = A$ (or $A_n\downarrow A$) if
\begin{enumerate}
\item $A_1\supset A_2 \supset \cdots$
\item $A=\bigcap_{k=1}^\infty A_k$.
\end{enumerate}
\end{definition}



\begin{definition}[{\bf monotone class}]
A collection of events $\mathcal M\subset 2^\Omega$  is a {\bf monotone class} if
\begin{enumerate}
\item $A_1, A_2,\ldots\in\mathcal M$ and $A_n\uparrow A \Longrightarrow A\in\mathcal M$
\item $A_1, A_2,\ldots\in\mathcal M$ and $A_n\downarrow A \Longrightarrow A\in\mathcal M$.
\end{enumerate}
\end{definition}




\begin{theorem}[{\bf $\sigma = \lambda+\pi = \mathscr M + f$ }]
\begin{align}
\text{$\mathcal F$ is a $\sigma$-field}
&\Longleftrightarrow  \text{$\mathcal F$ is a field and a monotone class} \label{eq: m+f}\\
&\Longleftrightarrow  \text{$\mathcal F$ is a $\lambda$-system and a $\pi$-system.} \label{eq: l+p}
 \end{align}
\end{theorem}
\begin{proof}
({\sl show (\ref{eq: l+p})}) Notice the direction $(\Longrightarrow)$ is trivial. To show the other direction suppose $\mathcal F$ is a $\lambda$-system and a $\pi$-system. We need to show $\mathcal F$ is a $\sigma$-field. Notice $\Omega\in \mathcal F$ is trivial by $\lambda$-system properties. Also $A\in \mathcal F\Rightarrow A^c\in \mathcal F$ is trivial by $\lambda$-system properties. To show $A_1, A_2, \ldots \in \mathcal F\Rightarrow \cup_{k=1}^\infty\in \mathcal F$ one uses a common trick for turning a non-disjoint union into a disjoint union.
\begin{align*}
\bigcup_{k=1}^\infty A_k &= \bigcup_{k=1}^\infty \underbrace{A_k - (A_1\cup \cdots \cup A_{k-1})}_{disjoint} \\
 &= \bigcup_{k=1}^\infty A_k \cap A_1^c \cap \cdots \cap A_{k-1}^c
\end{align*}
Now $A_k^c \in \mathcal F$ by $\lambda$-system properties and hence $ A_k \cap A_1^c \cap \cdots \cap A_{k-1}^c\in \mathcal F$ by $\pi$-system properties. Therefore $\cup_{k=1}^\infty A_k$ can be written as a disjoint union of events from $\mathcal F$. Therefore $\cup_{k=1}^\infty A_k\in \mathcal F$ by $\lambda$-system properties as was to be shown.


({\sl show (\ref{eq: m+f})}) Just as in the proof of (\ref{eq: l+p}) the only non-trivial thing to show is that when $\mathcal F$ is a field and a monotone class this implies that $\mathcal F$ is closed under countable union. Indeed if $A_1, A_2, \ldots \in \mathcal F$ then
\[
\bigcup_{k=1}^\infty A_k = \setlimup{n} {\bigcup_{k=1}^n A_k}
\]
where ${\bigcup_{k=1}^n A_k}\in \mathcal F$ by the field properties and therefore $\bigcup_{k=1}^\infty A_k \in \mathcal F$ by the monotone class properties.

\end{proof}


\subsection{Generators}


\begin{theorem}[{\bf field generated by $\mathcal C$}]
Let $\mathcal C\subset 2^\Omega$. Then
\begin{equation}
\nonumber
f\langle\mathcal C \rangle := \bigcap_{\shortstack{\text{\small $\mathcal F$ is a field}  \\
 \text{\small $\mathcal C\subset \mathcal F$} }}\mathcal F
\end{equation}
is a field (which contains $\mathcal C$).
\end{theorem}


\begin{theorem}[{\bf $\sigma$-field generated by $\mathcal C$}]
Let $\mathcal C\subset 2^\Omega$. Then
\begin{equation}
%\label{eq: generating}
\nonumber
\sigma\langle\mathcal C \rangle := \bigcap_{\shortstack{\text{\small $\mathcal F$ is a $\sigma$-field}  \\
 \text{\small$\mathcal C\subset \mathcal F$ }}}\mathcal F
\end{equation}
is a $\sigma$-field (which contains $\mathcal C$).
\end{theorem}

\begin{theorem}[{\bf monotone class generated by $\mathcal C$}]
Let $\mathcal C\subset 2^\Omega$. Then
\begin{equation}
\nonumber
 \mathscr M\langle\mathcal C \rangle := \bigcap_{\shortstack{\text{\small $\mathcal M$ is a monotone class}  \\
 \text{\small$\mathcal C\subset \mathcal M$ }}}\mathcal M
\end{equation}
is a monotone class (which contains $\mathcal C$).
\end{theorem}






\begin{theorem}[{\bf $\lambda$-system generated by $\mathcal C$}]
Let $\mathcal C\subset 2^\Omega$. Then
\begin{equation}
\nonumber
\lambda\langle\mathcal C \rangle := \bigcap_{\shortstack{\text{\small $\mathcal L$ is a $\lambda$-system}  \\
 \text{\small$\mathcal C\subset \mathcal L$ }}}\mathcal L
\end{equation}
is a $\lambda$-system (which contains $\mathcal C$).
\end{theorem}




\begin{theorem}[{\bf Good sets}]
Let $\mathcal C$ and $\mathcal G$ be two collections of subsets of $\Omega$. If
\begin{itemize}
\item $\mathcal C\subset \mathcal G$;
\item $\mathcal G$ is a $\sigma$-field
\end{itemize}
Then  $\sigma\langle\mathcal C\rangle\subset \mathcal G$.
\end{theorem}


%%%%%%%%%%%%%%%%%%%%%%
\begin{theorem}[{\bf Restricted generators}]
\label{restricTHM}
Let $\Omega$ be a sample space and  $\mathcal C$ be a class of subsets of $\Omega$.  If $\Omega_0\subset \Omega$ then
\[
\underbrace{\sigma\bigl\langle  \mathcal C \cap \Omega_0 \bigr\rangle}_{\shortstack{ \text{\small $\sigma$-field} \\ \text{\small on $\Omega_0$}}}= \underbrace{\sigma\bigl\langle \mathcal C \bigr\rangle}_{\shortstack{ \text{\small $\sigma$-field} \\ \text{\small on $\Omega$}}} \cap\, \Omega_0.
\]
\end{theorem}

\begin{proof}
({\sl Show $\sigma \langle  \mathcal C \cap \Omega_0 \rangle \subset \sigma \langle \mathcal C  \rangle \cap\, \Omega_0$}) This easily follows by {\it good sets}  since clearly $\mathcal C\cap \Omega_0\subset \sigma\langle \mathcal C\rangle\cap \,\Omega_0$ and  Exercise \ref{sliceoutF} shows that $\sigma \langle \mathcal C  \rangle \cap\, \Omega_0$ is a $\sigma$-field.

({\sl Show  $\sigma\langle \mathcal C \rangle \cap\, \Omega_0 \subset \sigma \langle  \mathcal C \cap \Omega_0 \rangle $})  Notice that this inclusion is
equivalent to the statement that for every $A\in \sigma\langle\mathcal C\rangle$, $A\cap \Omega_0\in \sigma\langle\mathcal C\cap \Omega_0\rangle.$  To show this  let
\[ \mathcal G:=\{ A\subset \Omega: A\cap \Omega_0 \in \sigma\langle \mathcal C \cap \Omega_0 \rangle \}. \]
It will then be sufficient to show the following four bullets and then use good sets to conclude $\sigma\langle\mathcal C \rangle\subset \mathcal G$.
\\
\textbullet($\mathcal C\subset \mathcal G$)
$A\in \mathcal C \Longrightarrow A\cap \Omega_0\in \mathcal C \cap \Omega_0 \subset \sigma\langle \mathcal C \cap \Omega_0 \rangle.  $
\\
\textbullet($\Omega \in \mathcal G$)
\begin{align*}
\Omega_0\subset \Omega
&\Longrightarrow \Omega \cap \Omega_0 = \Omega_0 \in \sigma\langle \mathcal C\cap \Omega_0\rangle\\
&\qquad\qquad\text{since a  $\sigma$-field on $\Omega_0$ must contain $\Omega_0$} \\
&\Longrightarrow \Omega \in \mathcal G.
\end{align*}
\\
\textbullet($A\in \mathcal G\Longrightarrow A^c \in \mathcal G$) Notice that $A^c$  denotes complementation within $\Omega$. Now
\begin{align*}
A\in\mathcal G&\Longrightarrow A\cap \Omega_0\in \sigma\langle \mathcal C\cap \Omega_0\rangle \\
&\Longrightarrow \underbrace{\Omega_0-A\cap \Omega_0}_\text{\footnotesize complement in $\Omega_0$}\in \sigma\langle \mathcal C\cap \Omega_0\rangle \\
&\Longrightarrow \underbrace{\Omega_0\cap (A^c\cup \Omega_0^c)}_{= A^c\cap\Omega_0}\in \sigma\langle \mathcal C\cap \Omega_0\rangle \\
&\Longrightarrow A^c\in\mathcal G.
\end{align*}
\\
\textbullet($A_1,A_2,\ldots \in \mathcal G \Longrightarrow \bigcup_k A_k\in \mathcal G$)
\begin{align*}
A_1,A_2,\ldots \in \mathcal G&\Longrightarrow A_k\cap \Omega_0 \in \sigma\langle \mathcal C\cap \Omega_0\rangle,\,\forall k \\
&\Longrightarrow  \bigcup_k (A_k\cap \Omega_0) \in\sigma\langle \mathcal C\cap \Omega_0\rangle \\
 &\Longrightarrow  \Bigl(\bigcup_k A_k\Bigr)\cap \Omega_0 \in\sigma\langle \mathcal C\cap \Omega_0\rangle \\
&\Longrightarrow  \bigcup_k A_k\in \mathcal G.
\end{align*}
\end{proof}


\begin{theorem}[{\bf Halmos's monotone class theorem}]  If $\mathcal F_0$ is a field then $\mathscr M\langle \mathcal F_0\rangle = \sigma\langle \mathcal F_0\rangle$.
\end{theorem}


\begin{theorem}[{\bf Dynkin's $\pi - \lambda$ theorem}]  If $\mathcal P$ is a $\pi$-system then $\mathcal \lambda\langle \mathcal P\rangle = \sigma\langle \mathcal P\rangle$.
\end{theorem}
\begin{proof} This proof uses {\it good sets} all over the place.
First notice that  $\lambda\langle \mathcal P\rangle \subset \sigma\langle \mathcal P\rangle$ follows directly from {\it good sets} since $\mathcal P \subset \sigma\langle \mathcal P\rangle$ and clearly $ \sigma\langle \mathcal P\rangle$ is also a $\lambda$-system. Therefore we only need to show $\sigma\langle \mathcal P\rangle \subset \lambda\langle \mathcal P\rangle$.

Each statement below gives a sufficient condition to establish that $\sigma\langle \mathcal P\rangle \subset \lambda\langle \mathcal P\rangle$. They are given in reverse dependency order to make it easier to follow the train of reasoning.

\begin{quote}
\begin{centering}
$\sigma\langle \mathcal P\rangle \subset \lambda\langle \mathcal P\rangle$
 \\
$\Uparrow$
\\
$\lambda\langle\mathcal P\rangle$ is a $\sigma$-field
 \\
$\Uparrow$
\\
$\lambda\langle\mathcal P\rangle$ is a $\pi$-system
 \\
$\Uparrow$
\\
$\forall A, B\in \lambda\langle\mathcal P\rangle$ one has $ A\cap B\in \lambda\langle\mathcal P\rangle$
 \\
$\Uparrow$
\\
$\forall A \in \lambda\langle\mathcal P\rangle$ one has $\lambda \langle \mathcal P\rangle \subset \mathcal G_A$ where $\mathcal G_A:=\bigl\{ B\subset \Omega\colon A\cap B\in \lambda\langle \mathcal P\rangle \bigr\} $
\\
$\Uparrow$
\\
$\forall A \in \lambda\langle\mathcal P\rangle$, $\mathcal P\subset \mathcal G_A$ and $\mathcal G_A$ is a $\lambda$-system. \\
\end{centering}
\end{quote}
The last statement above is what we show. Notice, first, that
\begin{align}
\label{eq: l+d good set}
A\in \mathcal G_B \Longleftrightarrow A\cap B\in \lambda\langle \mathcal P\rangle \Longleftrightarrow B\in \mathcal G_A.
\end{align}
In particular if $A\cap B\in \lambda\langle\mathcal P\rangle$ then one has that both $A\in \mathcal G_B$ and $B\in \mathcal G_A$.


({\sl Case 1: show $\mathcal P\subset \mathcal G_A$ and $\mathcal G_A$ is a $\lambda$-system when $A\in \mathcal P$})
\\
\textbullet($\mathcal P\subset \mathcal G_A$) If $B\in \mathcal P$ then $A\cap B\in \mathcal P$ by the $\pi$-system properties of $\mathcal P$. Therefore $B\in \mathcal G_A$.
\\
\textbullet($\Omega\in \mathcal G_A$) This follows since $A\cap \Omega = A \in \mathcal P$.
\\
\textbullet($B\in \mathcal G_A\Longrightarrow B^c\in \mathcal G_A$)
\begin{align*}
B\in \mathcal G_A
&\Longrightarrow A\cap B\in \lambda\langle \mathcal P\rangle \\
&\Longrightarrow A - A\cap B\in \lambda\langle \mathcal P\rangle,\quad\text{$\lambda$-system properties} \\
&\Longrightarrow A -  B\in \lambda\langle \mathcal P\rangle \\
&\Longrightarrow A \cap  B^c\in \lambda\langle \mathcal P\rangle \\
&\Longrightarrow B^c\in \mathcal G_A.
\end{align*}
\\
\textbullet(disjoint $B_1, B_2, \ldots\in \mathcal G_A\Longrightarrow \cup_k B_k\in \mathcal G_A$)
Notice  $A \cap \bigcup_{k=1}^\infty B_k =  \bigcup_{k=1}^\infty (A \cap B_k)$. The $A\cap B_k$'s are disjoint if the $B_k$'s are too. Since $B_k$'s  are in $\mathcal G_A$, by assumption, we must have $A\cap B_k\in \lambda\langle \mathcal P\rangle$.
Therefore  $ \bigcup_{k=1}^\infty (A \cap B_k)\in \lambda\langle \mathcal P\rangle$ by $\lambda$-system properties.
Therefore $A \cap \bigcup_{k=1}^\infty B_k \in \mathcal G_A$.


({\sl Case 2: show $\mathcal P\subset \mathcal G_A$ and $\mathcal G_A$ is a $\lambda$-system when $A\in \lambda\langle\mathcal P\rangle$})
\\
\textbullet($\mathcal P\subset \mathcal G_A$) The only reason we established Case 1 was to proof this part of Case 2. Indeed,  Case 1 establishes that when $A\in \mathcal P$ we have that $\lambda\langle \mathcal P\rangle \subset \mathcal G_A$ by {\it good sets}. Changing names gives $\lambda\langle \mathcal P\rangle \subset \mathcal G_B$ whenever $B\in \mathcal P$. Now
\begin{align*}
B\in \mathcal P
&\Longrightarrow  \lambda\langle \mathcal P\rangle \subset \mathcal G_B,\quad\text{from Case 1}\\
&\Longrightarrow  A \in \mathcal G_B\\
&\Longrightarrow  B \in \mathcal G_A,\quad\text{by (\ref{eq: l+d good set}).}
\end{align*}
\\
\textbullet($\Omega\in \mathcal G_A$) Same as in Case 1.
\\
\textbullet($B\in \mathcal G_A\Longrightarrow B^c\in \mathcal G_A$) Same as in Case 1.
\\
\textbullet(disjoint $B_1, B_2, \ldots\in \mathcal G_A\Longrightarrow \cup_k B_k\in \mathcal G_A$) Same proof as in Case 1.


\end{proof}



\begin{theorem}[{\bf Good sets, take 2}]
Let $\mathcal P$ and $\mathcal G$ be two collections of subsets of $\Omega$. If
\begin{itemize}
\item $\mathcal P\subset \mathcal G$;
\item $\mathcal P$ is a $\pi$-system;
\item $\mathcal G$ is a $\lambda$-system
\end{itemize}
Then  $\sigma\langle\mathcal P\rangle\subset \mathcal G$.
\end{theorem}





%%%%%%%%%%%%%
\begin{exercise} \label{sliceoutF}
Suppose $\mathcal F$ is a $\sigma$-field on $\Omega$ and let $\Omega_0$ be any subset of $\Omega$ (not necessarily in $\mathcal F$). Prove that $\mathcal F\cap \Omega_0 := \{ F\cap \Omega_0: F\in \mathcal F\}$ is a $\sigma$-field on $\Omega_0$. \end{exercise}

\begin{exerciseproof}

\flushleft\textbullet({\sl Show $\Omega_0\in \mathcal F\cap \Omega_0$})
Notice that $\Omega \in \mathcal F$. Therefore $\Omega\cap \Omega_0\in \mathcal F\cap \Omega_0$. Now since $\Omega_0\subset \Omega$ implies $\Omega\cap\Omega_0= \Omega_0$ we have that $\Omega_0\in \mathcal F\cap \Omega_0$ as was to be shown.
\\
\textbullet({\sl Show $A\in \mathcal F\cap \Omega_0\Rightarrow (\Omega_0 - A)\in \mathcal F\cap \Omega_0$}) Let $A\in \mathcal F\cap \Omega_0$. Then $A = B\cap \Omega_0$ for some $B\in\mathcal F$. Therefore letting $A^c$ denote complementation within the larger space $\Omega$ we have
\begin{align*}
\Omega_0-A & = \Omega_0 \cap A^c\\
& =\Omega_0 \cap (B^c \cup \Omega_0^c)\\
& =\underbrace{\Omega_0 \cap B^c.}_{\in\mathcal F\cap \Omega_0}
\end{align*}
\\
\textbullet({\sl Show $A_1,A_2,\ldots \in \mathcal F\cap \Omega_0\Rightarrow \cup_{k=1}^\infty A_k\in \mathcal F\cap \Omega_0$}) Notice that each $A_k = B_k\cap \Omega_0$ for some $B_k\in\mathcal F$. Therefore
\begin{align*}
\bigcup_{k=1}^\infty A_k &=\bigcup_{k=1}^\infty (B_k\cap \Omega_0)\\
&=\Omega_0\cap \underbrace{\bigcup_{k=1}^\infty B_k}_{\mathcal F}
\end{align*}
Therefore $\bigcup_{k=1}^\infty A_k \in \mathcal F\cap \Omega_0$.
\end{exerciseproof}





\begin{exercise}
Prove Halmos's monotone class theorem. (Hint: To show $\sigma\langle \mathcal F_0 \rangle \subset \mathscr M\langle \mathcal F_0\rangle$ notice that it will be sufficient to show that $\mathscr M\langle \mathcal F_0\rangle$ is a field (why?); then to show  that $\mathscr M\langle \mathcal F_0\rangle$ is a field start by showing it is closed under complementation, then under intersection.)
\end{exercise}




\begin{exercise}
 \label{h2}
For any non-empty class $\mathcal A\subset 2^\Omega$, if
\begin{align*}
\mathcal C &:= \textit{the collection of $\mathcal A$ sets and their complements}\\
\mathcal I &:= \textit{the collection of finite intersections of $\mathcal C$ sets} \\
\mathcal U &:=\textit{the  collection of finite unions of $\mathcal I$ sets}.
\end{align*}
then $f\langle\mathcal A\rangle = \mathcal U$. Hint: first show $\mathcal U$ is closed under intersections, then complements.
\end{exercise}
\begin{exerciseproof}
We show $\mathcal U$  is a field and apply {\it good sets} to conclude $f\langle\mathcal A\rangle = \mathcal U$.


\flushleft\textbullet({\sl Show $\Omega \in \mathcal U$}) This is clear since $\mathcal C\subset \mathcal U$ and unions of two complimentary sets in  $\mathcal C$ sets gives $\Omega$.
\\
\textbullet({\sl Show $A,B\in \mathcal U\Longrightarrow A\cup B\in \mathcal U$}) This is trivial by definition of $\mathcal U$.
\\
\textbullet({\sl Show $A,B\in \mathcal U\Longrightarrow A\cap B\in \mathcal U$}) The general form of $\mathcal U$ sets can be written as follows (where $I_k\in\mathcal I$)
\begin{align*}
\underbrace{\bigcup_{k=1}^n I_k}_{\in \mathcal U}  \cap \underbrace{\bigcup_{j=n+1}^m I_{j}}_{\in \mathcal U} = \underbrace{\bigcup_{k=1}^n  \bigcup_{j=n+1}^m \underbrace{[I_{j}\cap I_{k}]}_{\in \mathcal I} }_{\in \mathcal U}
\end{align*}
since $\mathcal I$ is closed under intersection.
\\
\textbullet({\sl Show $A\in \mathcal U\Longrightarrow A^c\in \mathcal U$})
If $A\in \mathcal U$ then $A^c$ has the form
\[
\Bigl(\bigcup_{k=1}^n I_k \Bigr)^c = \bigcap_{k=1}^n I_k^c.
\]
where $I_k\in \mathcal I$ has the form $\cap_{j=1}^m C_j$ with $C_j\in \mathcal C$. Now notice that $I_k^c$ has the form $\cup_{j=1}^m C_j^c$ which is clearly in $\mathcal U$ since  $C_j^c\in\mathcal U$ and $\mathcal U$ is closed under union (by definition).
\end{exerciseproof}



\begin{definition}[{\bf Semi-ring with unit}]
A  collection of events $\mathcal A\subset 2^\Omega$ is called a {\bf semi-ring with unit} if
\begin{enumerate}
\item $\Omega\in \mathcal A$
\item $A, B\in\mathcal A\Longrightarrow A\cap B\in \mathcal A$
\item  If $A\in \mathcal A $ then $A^c$ is a finite disjoint union of $\mathcal A$-sets
\end{enumerate}
\end{definition}


\begin{exercise}
\label{ex: semi-ring}
Suppose $\mathcal A\subset 2^\Omega$ is a semi-ring with unit. Let $\mathscr D$ denote the class of finite disjoint unions of $\mathcal A$-sets. Show  $f\langle \mathcal A\rangle=\mathscr D$.
Hint: first show $\mathscr D$ is closed  intersections, then complements.
\end{exercise}
\begin{exerciseproof}

\textbullet({\sl Show $\mathcal A\subset \mathscr D$})
Trivial by definition of $\mathscr D$.
\\
\textbullet({\sl Show $\Omega\in \mathscr D$}) This follows since  $\Omega\in \mathcal A$.
\\
\textbullet({\sl Show $\mathscr D$ is closed under intersection})
Suppose $D, Q\in \mathscr D$. Then
\[
D\cap Q = \bigcup_{k=1}^n \bigcup_{j=1}^m A_k \cap  A_j^\prime
\]
where $A_k, A_j^\prime \in\mathcal A$, $A_1,\ldots, A_n$ are disjoint and $A_1^\prime,\ldots, A_n^\prime$  are disjoint. Therefore $A_k \cap  A_j^\prime$ are disjoint and are in $\mathcal A$ by property 2 of the semi-right-with-unit. Therefore $D\cap Q\in\mathscr D$ and therefore $\mathscr D$ is closed under intersection.
\\
\textbullet({\sl Show $\mathscr D$ is closed under complementation})
Let $D\in \mathscr D$. Clearly $D^c$ has the form $\cap_{i=1}^n A_i^c$ where $A_i\in \mathcal A$ are disjoint. But $A_i^c\in \mathscr D$ by property 3 of a semi-right-with-unit. Therefor $D^c= \cap_{i=1}^n A_i^c$ is in $\mathscr D$ since we already showed that $\mathscr D$ is closed under intersection.

The above bullets shows that $\mathscr D$ is a field which contains $\mathcal A$. Therefore $f\langle \mathcal A\rangle \subset \mathscr D$. To finish, simply notice that $\mathscr D\subset f\langle \mathcal A\rangle  $ by closure properties of a field.

\end{exerciseproof}




%%%%%%%%%%%%%
\begin{exercise} \label{stofB}
Show that  $\mathcal B_0((0,1])$ from Definition \ref{fb} is a field and coincides with $f\langle (a,b]: 0\leq a\leq b\leq 1 \rangle $.
\end{exercise}

\begin{exerciseproof}
This should follow directly from exercise \ref{ex: semi-ring} and the fact that $\{ (a,b]: 0\leq a\leq b\leq 1 \}$ is a semi-ring with unit.
\end{exerciseproof}


\begin{exercise}
\label{ex1}
Let $\Omega = \Bbb R$.
Show that $f\langle (-\infty,a]: -\infty< a< \infty\rangle$  is the the set of finite (possibly empty) disjoint unions of intervals of the form $(-\infty, b]$, $(a,\infty)$ and $(a,b]$ for finite $a < b$. (Hint:  change the generators a bit to apply exercise \ref{ex: semi-ring}.)
\end{exercise}

\begin{exerciseproof}
  I think the easiest way to prove this is to use exercise \ref{ex: semi-ring}. In particular
\begin{align*}
f\langle (-\infty,a]:& -\infty< a< \infty\rangle \\
&= f\langle \underbrace{\varnothing, \Bbb R, (-\infty, b], (a,\infty), (a,b]\colon a < b}_{\text{semi-ring-with unit}} \rangle
\end{align*}

\end{exerciseproof}


\begin{exercise} \label{countablly generated}
Let $\mathcal A\subset2^\Omega$ be a countable collection of $\Omega$ sets.
Show that  $f\langle\mathcal A\rangle$ is a  countable collection of $\Omega$ sets.
\end{exercise}

\begin{exerciseproof}
Use  exercise \ref{h2}.
\end{exerciseproof}



\begin{exercise}
Let $\mathcal L$ be a collection of subsets of $\Omega$. Show that $\mathcal L$ is a $\lambda$-system if and only if $\mathcal L$ satisfies the following three conditions
\begin{enumerate}
\item $\Omega \in \mathcal L$
\item If $A-B\in \mathcal L$ whenever $B\subset A$ and $A, B\in \mathcal L$
\item $A_1, A_2,\ldots\in\mathcal L$ and $A_n\uparrow A \Longrightarrow A\in\mathcal L$.
\end{enumerate}
\end{exercise}

% %%%%%%%%%%%%%%%%%%
% \begin{definition}[{\bf Borel field}] The \underline{Borel field} on $(0,1]$, denoted $\mathcal B_0((0,1])$, is defined as
% \[
%  f\bigl \langle (a,b] : 0\leq a < b\leq 1 \bigr\rangle
% \]
% (Note: by exercise \ref{stofB}  this does not conflict with definition \ref{fb}).
% \end{definition}


% %%%%%%%%%%%%%%%%%%%%%%%%%%%
% \begin{definition}[{\bf Borel $\sigma$-field}] The \underline{Borel $\sigma$-field} on $(0,1]$, denoted $\mathcal B((0,1])$, is defined as
% \[
%  \sigma\bigl \langle (a,b] : 0\leq a < b\leq 1 \bigr\rangle.
% \]
% \end{definition}




\subsection{Borel $\sigma$-fields}
Borel $\sigma$-fields are used throught the whole theory of measure and integration. In this section we go into detail treatment of these fields. The main story is that Borel $\sigma$-fields have many equivenlent generators. Different generators are useful for proving different things. For example the Borel $\sigma$-field on $(0,1]^d$ as generated by the field of finite disjoint unions of rectangles is useful for constructing Lebesque measure. To specidfy uniqueness of a measure on $\Bbb R^d$ with a particular property it is often useful to consider the Broel field on $\Bbb R^d$ to be $\sigma\bigl\langle (-\infty,c_1]\times \cdots \times (-\infty, c_d]: -\infty < c_k< \infty \bigr\rangle$ the generators of which form a $\pi$-system.


\begin{definition}[{\bf Metric space Borel $\sigma$-field: $\mathcal B(\Omega)$}]
Suppose $\Omega$ forms a metric space with some metric $d:\Omega\times\Omega\rightarrow [0,\infty]$. A set $A\subset \Omega$ is said to be {\bf open} if for each $x\in A$, there exists an $\epsilon>0$ such that the {open ball} $\{y\in\Omega: d(x,y)<\epsilon \}$ is contained in $A$. The {\bf Borel $\sigma$-field of $\Omega$} (with respect to metric $d$), denoted $\mathcal B(\Omega)$, is defined as the $\sigma$-field generated by the open sets.
\end{definition}

The above definition immediately allows us to define the Borel $\sigma$-fields  $\mathcal B(\Bbb R^d)$ and $\mathcal B((0,1]^d)$. To define $\mathcal B(\bar{\Bbb R}^d)$ , where $\bar{\Bbb R}:=[-\infty, \infty]$ we use the metric given by $d(x,y):= |\tau(x) - \tau(y)|$ where
\begin{equation}
\label{eq: generate extended metric}
\tau(x):=\begin{cases} \frac{x}{1+|x|} &\text{ when $|x|<\infty$};\\
 1 &\text{when $x=\infty$};\\
  -1&\text{when $x=-\infty$}.
 \end{cases}
 \end{equation}



\begin{theorem}[{\bf Borel restrictions}]
Let $\Omega$ be a metric space and $\Omega_o\subset \Omega$.  Then the Borel $\sigma$-field $\mathcal B(\Omega_o)$, which is constructed using the induced metric on $\Omega$, satisfies
\begin{align*}
\mathcal B(\Omega_o) &=  \mathcal B(\Omega)\cap \Omega_o
\end{align*}
If, in addition, $\Omega_o\in \mathcal B(\Omega)$ then $\mathcal B(\Omega_o)=  \{ B\in \mathcal B(\Omega): B\subset \Omega_o \}$.
\end{theorem}



\begin{proof}
({\sl Show $ \mathcal B(\Omega_o) = \mathcal B(\Omega)\cap \Omega_o$})
Let
\begin{align*}
\mathcal G &:= \text{open subsets of $\Omega$}\\
\mathcal G_o &:= \text{open subsets of $\Omega_o$}
\end{align*}
Notice that Theorem 2.30 in Rudin (Principles in Mathematical Analysis) shows that
\[ \mathcal G_o  = \mathcal G\cap \Omega_o.\]
This implies
\begin{align*}
\mathcal B(\Omega_o)  = \sigma \langle \mathcal G_o\rangle &=  \sigma\langle \mathcal G \cap\Omega_o \rangle\\
& = \sigma\langle \mathcal G \rangle\cap \Omega_o,\,\,\text{by Theorem \ref{restricTHM}}\\
& = \mathcal B(\Omega)\cap \Omega_o.
\end{align*}


({\sl Show $\mathcal B(\Omega)\cap \Omega_o =  \{ B\in \mathcal B(\Omega): B\subset \Omega_o \}$ whenever $\Omega_o\in \mathcal B(\Omega)$}). To see `$\supset$' suppose $B\subset \Omega_o$ and $B\in \mathcal B(\Omega)$. Then $B = B\cap \Omega_o \in \mathcal B(\Omega)\cap \Omega_o$. To  see `$\subset$' let $B\in  \mathcal B(\Omega)\cap \Omega_o $ so that $B= \tilde B\cap \Omega_o$ where $\tilde B\in \mathcal B(\Omega)$. Since $\Omega_o\subset \Omega$ we have $B\in \mathcal B(\Omega)$ and $B\subset  \Omega_o $.
\end{proof}

% %%%%%%%%%%%%%%%%%%
% \begin{definition}[{\bf The Borel field on $(0,1]^d$}] The Borel field on $(0,1]^d$, denoted $\mathcal B_0((0,1]^d)$, is defined as the field generated by the rectangles in $(0,1]^d$ as follows
% \[
% \mathcal B_0((0,1]^d):= f\bigl \langle (a_1,b_1]\times\cdots\times (a_d,b_d] : 0\leq a_k< b_k\leq 1 \bigr\rangle.
% \]
% \end{definition}


% %%%%%%%%%%%%%%%%%%%%%%
% \begin{theorem}[{\bf Structure of the Borel field on $(0,1]^d$}]
% \label{bfield}
% Any set in   $\mathcal B_0((0,1]^d)$ is a  finite (possibly empty) disjoint union of rectangles from $\{(a_1,b_1]\times\cdots\times (a_d,b_d] : 0\leq a_k< b_k\leq1  \}$.
% \end{theorem}

% %%%%%%%%%%%%%%%%%%%%%%%%%%%
% \begin{definition}[{\bf The Borel $\sigma$-field on $(0,1]^d$}]
%  The Borel $\sigma$-field on $(0,1]^d$, denoted $\mathcal B((0,1]^d)$, is defined as the field generated by the rectangles in $(0,1]^d$ as follows
% \[
% \mathcal B((0,1]^d):= \sigma\bigl \langle (a_1,b_1]\times\cdots\times (a_d,b_d] : 0\leq a_k<b_k\leq 1 \bigr\rangle.
% \]
% \end{definition}


% %%%%%%%%%%%
% \begin{definition}[{\bf The Borel $\sigma$-field on $\Bbb R^d$}]
% The Borel $\sigma$-field of $\Bbb R^d$, denoted $\mathcal B(\Bbb R^d)$, is defined as the $\sigma$-field generated by the class of all finite rectangles in $\Bbb R^d$ as follows
% \[ \mathcal B(\Bbb R^d):=\sigma\bigl\langle (a_1,b_1]\times \cdots \times (a_d,b_d]: -\infty< a_k<b_k<\infty \bigr\rangle. \]
% \end{definition}




\begin{theorem}[{\bf Non-exhaustive list of useful Borel generators}]\label{eqvgens}
\begin{align*}
\mathcal B(\Bbb R^d)
& = \sigma\bigl\langle (-\infty,c_1]\times \cdots \times (-\infty, c_d]: -\infty < c_k< \infty \bigr\rangle    \\
& = \sigma\bigl\langle \text{open balls of $\Bbb R^d$}  \bigr\rangle  \\
& = \sigma\bigl\langle \text{open subsets of $\Bbb R^d$}  \bigr\rangle  \\
& = \sigma\bigl\langle \text{closed subsets of $\Bbb R^d$}  \bigr\rangle\\
& = \sigma\bigl\langle \text{compact subsets of $\Bbb R^d$}  \bigr\rangle \\
& = \sigma\bigl\langle \text{rectangles in $\Bbb R^d$}  \bigr\rangle \\
& = \sigma\bigl\langle \text{cylinders $\Bbb R^d$}  \bigr\rangle
\end{align*}
\begin{align*}
\mathcal B((0,1])
&=\sigma\langle \mathcal B_0((0,1])\rangle \hphantom{asdfdddddddasdfasdfasdfasdf}\\
&= \sigma\bigl\langle (a,b]: 0\leq a \leq b \leq 1 \bigr\rangle \\
  & =\sigma\bigl\langle (a,b): 0< a < b <1  \bigr\rangle \\
  & =\sigma\bigl\langle [a,b]: 0< a < b <1  \bigr\rangle \\
    & =\sigma\bigl\langle (0,a]: 0< a  <1  \bigr\rangle \\
& = \sigma\bigl\langle \text{open subsets of $(0,1]$}  \bigr\rangle  \\
& = \sigma\bigl\langle \text{closed subsets of $(0,1]$}  \bigr\rangle
\end{align*}
\begin{align*}
\mathcal B((0,1]^d)
& = \mathcal B(\Bbb R^d)\cap (0,1]^d \\
& = \{B\in\mathcal B(\Bbb R^d): B\subset (0,1]^d  \} \\
&=  \sigma\bigl \langle (a_1,b_1]\times\cdots\times (a_d,b_d] : 0\leq a_k<b_k\leq 1 \bigr\rangle \\
& = \sigma\bigl\langle  \mathcal B_0((0,1]^d)  \bigr\rangle.
\end{align*}
where $\mathcal B_0((0,1]^d):= f\bigl \langle (a_1,b_1]\times\cdots\times (a_d,b_d] : 0\leq a_k< b_k\leq 1 \bigr\rangle$ is the Borel field on $(0,1]^d$ which equals the finite (possibly empty) disjoint union of rectangles from $\{(a_1,b_1]\times\cdots\times (a_d,b_d] : 0\leq a_k< b_k\leq1  \}$

\end{theorem}

I would venture to say that one of the most important results above is that $\mathcal B((0,1]^d) = \sigma\bigl\langle  \mathcal B_0((0,1]^d)  \bigr\rangle$ where $\mathcal B_0((0,1]^d)$ is the field of finite (possibly empty) disjoint union of rectangles. This characterization allows one to construct probabilities on $\mathcal B_0((0,1]^d)$, then use the Carath\'eodory Extension Theorem to extend this to a full probability model on $\mathcal B((0,1]^d)$.

Notice that most of the equalities in Theorem \ref{eqvgens} are shown using the good sets principle. In particular, to show that $\sigma\langle \mathcal A_1\rangle=\sigma\langle\mathcal A_2 \rangle$ one simply needs to establish that $ \mathcal A_1 \subset \sigma\langle \mathcal A_2\rangle$ (which implies that $ \sigma\langle \mathcal A_1\rangle \subset \sigma\langle \mathcal A_2\rangle$ by ``good sets") and $ \mathcal A_2 \subset \sigma\langle \mathcal A_1\rangle$ (which implies that $ \sigma\langle \mathcal A_2\rangle \subset \sigma\langle \mathcal A_1\rangle$ by ``good sets").

\begin{proof}
I will only show one of these equalities. The rest follow by similar arguments.
To show
\[ \sigma\bigl\langle (a,b]: 0< a < b < 1 \bigr\rangle =\sigma\bigl\langle (a,b): 0< a < b <1  \bigr\rangle\]
it will be sufficient to show the following two statements for any arbitrary $0<a_0<b_0<1$.
\\
\textbullet({\sl Show $(a_0,b_0]\in \sigma\langle (a,b):0<a<b<1\rangle$})
This is follows from the identity
\[(a_0,b_0] = \bigcap_{n=1}^\infty (a_0,b_0+n^{-1} ).  \]
\\
\textbullet({\sl Show $(a_0,b_0)\in \sigma\langle (a,b]:0<a<b<1\rangle$})
This is follows from the identity
\[(a_0,b_0) = \bigcup_{n=1}^\infty (a_0,b_0-n^{-1}  ].  \]
\end{proof}

The sets in the Borel $\sigma$-field are extremely rich. In fact, it is hard to show that there are sets which are not in $\mathcal B(\Bbb R)$. The easiest way to find such a set is to use properties of Lebseque measure which we will construct later in the notes. Therefore, we postpone a discussion of such sets until	 we have Lebseque measure at our disposal. For the reminder of this section we give some examples of sets which {\em are} in the Borel $\sigma$-fields on Euclidean space.

\begin{example}
The set of normal and abnormal numbers are in $\mathcal B((0,1])$.
\end{example}


\begin{example}
All countable, co-countable (i.e. complements of countable sets), and perfect subsets of $(0,1]$ are in $\mathcal B((0,1])$. In particular, the collection of  irrational numbers in $(0,1]$ is a Borel set.
\end{example}

\begin{exercise}
Show the Cantor set is an uncountable set in $\mathcal B((0,1])$ (a nice way to see that it is uncountable is to work with a base-3 digit characterization of the Cantor set).
\end{exercise}


\begin{exercise}
Let $\Omega$ be a metric space with distance function $d$. $\Omega$ is said to be {\bf separable} if there exists a countable $\Omega_0\subset \Omega$ which is dense in $\Omega$ (i.e., every point of $\Omega$ is a limit of some sequence of points of $\Omega_0$).
\begin{enumerate}
\item Show that $\sigma\langle \text{open balls in $\Omega$}\rangle\subset \mathcal B(\Omega)$.
\item Show that  $\sigma\langle \text{open balls in $\Omega$}\rangle= \mathcal B(\Omega)$ if $\Omega$ is separable.
\item Show that $\Omega = \bar {\Bbb R}$ is separable with the metric defined with (\ref{eq: generate extended metric}) and conclude that $\sigma\langle \text{open balls in $\bar{\Bbb R}$}\rangle= \mathcal B(\bar{\Bbb R})$.
\end{enumerate}
\end{exercise}

\begin{exerciseproof}
({\sl Show 1.})
This is obvious since the open balls are also open sets (which generate $\mathcal B(\Omega)$).

({\sl Show 2.})
It will be sufficient to show that the open sets can be constructed from a countable number of set operations on the open balls. Let $\Omega_0\subset \Omega$ be a dense subset.
Let $G$ be an open subset of $\Bbb R^d$. Now for each element $y\in G$ there exists an open ball $B_y$ centered at $y$ such that $B_y\subset G$. By the seperatbility there exists a $y_0\in \Omega_0$ close enough to $y$ such that there exists another open ball, call it $B_{y_0}$, for which  $y\in B_{y_0}\subset B_y$. Then clearly
\[
G = \bigcup_{y\in G} B_{y_0}.
\]
 Notice  there are only countably many elements in $\Omega_0$. Therefore the union $ \bigcup_{y\in G} B_{y_0}$ must be a countable union  and therefore $G$ can be constructed  from a countable number of set operations on the open balls, as was to be shown.


({\sl Show 3.}) Take the separable subset to be $\Bbb Q\cup \{ \infty\} \cup \{ -\infty\}$ where $\Bbb Q$ denotes the finite rationals. Notice there is an implicit application of continuity of $\tau$ here.
\end{exerciseproof}

\begin{exercise}
$\vphantom{aaa}$
\begin{enumerate}
\item Show that $\mathcal B(\bar{\Bbb R})$ is generated by the sets of the form $[-\infty, a]$ for $-\infty<a<\infty$, and also by sets of the form $[-\infty, a)$ for  $-\infty<a<\infty$.
\item Show that $\mathcal B(\bar{\Bbb R})$ is not generated by sets of the form $(-\infty, a)$ for $-\infty<a<\infty$. (Hint:  find a $\sigma$-field which contains the intervals $(-\infty, a)$ but which is strictly smaller than  $\mathcal B(\bar{\Bbb R})$ ).
\end{enumerate}
\end{exercise}
\begin{exerciseproof}
({\sl Show 1.})
Since ${\bar{\Bbb R}}$ is separable we can use the fact that $\mathcal B(\bar{\Bbb R})$  is generated by the open balls.
Therefore it is sufficient to show that $\mathcal F_1 = \mathcal F_2 = \mathcal F_3$ where
\begin{align*}
\mathcal F_1  & = \sigma\langle \text{open balls in $\bar{\Bbb R}$} \rangle\\
\mathcal F_{2} &= \sigma\langle [-\infty, a]\colon -\infty<a<\infty \rangle\\
\mathcal F_{3} &= \sigma\langle [-\infty, a)\colon -\infty<a<\infty   \rangle.
\end{align*}
Since each $\mathcal F_i$ is a $\sigma$-field we just have to show that one set of generators can construct the other set of generators, then apply {\it good sets}.
If $-\infty<a<\infty$ then
\begin{align*}
[-\infty, a] &= \bigcap_{n=1}^\infty [-\infty, a+\textstyle\frac{1}{n})  \\
[-\infty, a) &= \bigcup_{n=1}^\infty [-\infty, a-\textstyle\frac{1}{n}]
\end{align*}
which implies $\mathcal F_1 = \mathcal F_2$. Now notice that the open balls in $\bar{\Bbb R}$ have the form
$[-\infty, \infty],[-\infty, a),\, (b, a),\, (b, \infty] $
where  $-\infty<a<b<\infty$. Since these open balls contain the generators of $\mathcal F_3$ we immediately have $\mathcal F_3\subset \mathcal F_1$. The other inclusion easily follows by using the sets $[-\infty, a)$ to generate $(b, a),\, (b, \infty] $, thus showing that {\it open balls}$\,\subset \mathcal F_3$.

({\sl Show 2.}) Define
\[
\mathcal B_{-\infty, \infty}:= \mathcal B(\Bbb R) \cup \bigl\{B\cup \{-\infty,\infty \}\colon B\in \mathcal B(\Bbb R) \bigr\}
\]
and notice that it is easy/standard to show  $\mathcal B_{-\infty, \infty}$ is a $\sigma$-field. Since $\mathcal B_{-\infty, \infty}$ contains the sets $(-\infty, a)$ we have
\[ \sigma\langle  (-\infty, a)\colon -\infty<a<\infty \rangle\subset \mathcal B_{-\infty, \infty}\]
by {\it good sets}. Moreover
\[\mathcal B_{-\infty, \infty} \subsetneq  \mathcal B(\bar{\Bbb R}) \]
since $\{\infty \}\in  \mathcal B(\bar{\Bbb R}) $ but clearly $\{\infty \}\not\in  \mathcal B_{-\infty, \infty}  $.
\end{exerciseproof}




%
%%%%%%%%%%%%%%%%%%%
%\begin{exercise}
%Let $\Omega:=(0,1]$ and  let $\mathcal B(0,1]:= \sigma\langle (a,b]: 0< a <  b < 1 \rangle$. Show that
%\begin{align}
%\mathcal B(0,1] &= \sigma\bigl\langle (0,a]: 0< a < 1 \bigr\rangle \\
%& = \sigma\bigl\langle \text{open subsets of $(0,1]$}  \bigr\rangle
%\end{align}
%\end{exercise}
%

%\begin{exercise}  If $\Omega=[0,1]$ then
%\[\sigma\langle (0,a]: 0<a<1 \rangle \neq \sigma\langle [0,a]: 0<a<1 \rangle.\]
%However, if
% If $\Omega=(0,1]$ or  $\Omega=[0,1)$  or  $\Omega=(0,1)$ then
%\[\sigma\langle (0,a]: 0<a<1 \rangle = \sigma\langle [0,a]: 0<a<1 \rangle.\]
%This is just to show that you need to be a bit careful with the generators.
%\end{exercise}
%
%



\clearpage
%-------------------------------'
%---------section  ---------------'
%-------------------------------'
\section{Measures}
%-------------------------------'
%---------section  ---------------'
%-------------------------------'


\begin{definition}[{\bf finitely additive probability}]
If $\mathcal F_0$ is a field on $\Omega$, then $P:\mathcal F_0\rightarrow [0,1]$ is said to be a {\bf finitely additive probability on $\mathcal F_0$} if
\begin{enumerate}
\item $P[\Omega]=1$
\item $P[A\cup B]=P[A]+P[B]$ \\ for all disjoint $A, B\in\mathcal F_0$.
\end{enumerate}
\end{definition}




\begin{definition}[{\bf probability measure}]
If $\mathcal F_0$ is a field on $\Omega$, then $P:\mathcal F_0\rightarrow [0,1]$ is said to be a {\bf probability measure on $\mathcal F_0$} if
\begin{enumerate}
\item $P(\Omega)=1$
\item $P\bigl( \bigcup_{k=1}^\infty A_k \bigr)=\sum_{k=1}^\infty P(A_k)$ \\ for all disjoint $A_1, A_2,\ldots \in\mathcal F_0$ such that $\bigcup_{k=1}^\infty A_k \in \mathcal F_0$.
\end{enumerate}
\end{definition}



\begin{definition}[{\bf Measure}]
If $\mathcal F_0$ is a field of $\Omega$-sets, then $\mu:\mathcal F_0\rightarrow [0,\infty]$ is a {measure} if
\begin{enumerate}
\item $\mu(\varnothing)=0$
\item $\mu\bigl( \bigcup_{k=1}^\infty A_k \bigr)=\sum_{k=1}^\infty \mu(A_k)$ \\ for all disjoint $A_1, A_2,\ldots \in\mathcal F_0$ such that $\bigcup_{k=1}^\infty A_k \in \mathcal F_0$.
\end{enumerate}
\end{definition}



\begin{definition}[{\bf Measurable space}]
If $\mathcal F$ is a $\sigma$-field on $\Omega$ then the pair $(\Omega, \mathcal F)$  is called a {measurable space}.
\end{definition}


\begin{definition}[{\bf Measure/Probability space}]
If $\mu$ is a measure on the measurable space $(\Omega, \mathcal F)$ then the triple $(\Omega, \mathcal F, \mu)$ is called a {measure space}.
If, in addition, $\mu$ is a probability measure then the triple $(\Omega, \mathcal F, \mu)$ is called a {probability space}.
\end{definition}


\begin{definition}[{\bf Finite and $\sigma$-finite}]
Let $(\Omega, \mathcal F,\mu)$ be a measure space.
\begin{itemize}
\item If $\mu(\Omega)<\infty$ then $\mu$ is said to be a {finite} measure;
\item If $\mu(\Omega)=\infty$ then $\mu$ is said to be an {infinite} measure;
\item If there exists $\mathcal F$-sets $A_1,A_2,\ldots$ such that $\Omega = \cup_{k=1}^\infty A_k$ and $\mu(A_k)<\infty$ then $\mu$ is said to be a {$\sigma$-finite} measure;
\item If $\mathscr A\subset \mathcal F$ such that there exists $\mathscr A$-sets $A_1,A_2,\ldots$ such that $\Omega = \cup_{k=1}^\infty A_k$ and $\mu(A_k)<\infty$ then $\mu$ is said to be {$\sigma$-finite on $\mathscr A$}.
\end{itemize}
\end{definition}


Note that every probability measure is  $\sigma$-finite. Most of the basic rules of probability follow from  finitely additive properties.

\begin{theorem}[{\bf Basic properties of probability measures}]
Suppose $P$ is a probability measure on field $\mathcal F_0$ of $\Omega$ subsets. Then each of the following statements hold for all $\mathcal F_0$-sets $A, A_1, \ldots, B, B_1, \ldots$
\begin{enumerate}
\item $P[A^c] = 1-P[A]$;
\item $P[\varnothing]=0$;
\item\label{bpfa one} If $A\subset B$ then $P[B-A]=P[B]-P[A]$;
\item\label{bpfa two} {\bf (Increasing)} If $A\subset B$ then  $P[A]\leq P[B]$;
\item\label{bpfa three} {\bf (Inclusion-exclusion)} $P[A\cup B] = P[A]+P[B]-P[A\cap B]$;
\item\label{bpfa four} {\bf (Finite additivity)}  If $A_k$'s are disjoint then
\\$P\bigl[ \bigcup_{k=1}^n A_k \bigr] = \sum_{k=1}^n P[A_k]$;
\item\label{bpfa five} {\bf (Finite sub-additivity)}  $P\bigl[ \bigcup_{k=1}^n A_k \bigr] \leq \sum_{k=1}^n P[A_k]$;
\item\label{bpfa six} {\bf (Approximation)} If $A_k\subset B_k$ then
\\$P\bigl[ \bigcup_{k=1}^n B_k \bigr] - P\bigl[ \bigcup_{k=1}^n A_k \bigr] \leq \sum_{k=1}^n P[B_k - A_k]$ and
$P\bigl[ \bigcap_{k=1}^n B_k \bigr] - P\bigl[ \bigcap_{k=1}^n A_k \bigr] \leq \sum_{k=1}^n P[B_k - A_k]$;
\item\label{bpfa seven}{(\bf Continuous from below)} \\ If  $A_n\uparrow A\in \mathcal F_0$ then $P[A_n]\uparrow P[A]$;
\item\label{bpfa eight}{(\bf Continuous from above)} \\ If $A_n\downarrow A\in \mathcal F_0$ then $P[A_n]\downarrow P[A]$;
\item\label{bpfa nine}{(\bf Countable sub-additivity)}\\  If $\cup_{k=1}^\infty A_k \in \mathcal F_0$  then $P\bigl[\sum_{k=1}^\infty A_k\bigr] \leq \sum_{k=1}^\infty P[A_k]$;
\end{enumerate}
\end{theorem}

\begin{proof}
\textbullet({\sl Show item \ref{bpfa one}}) If $A\subset B$ then $B = A \cup (B-A)$ is a disjoint union. Therefore $P[B] = P[A] + P[B-A]$ which implies $P[B-A]=P[B]-P[A]$.
\\
\textbullet({\sl Show item \ref{bpfa two}}) Use $0\leq P[B-A]=P[B]-P[A]$.
\\
\textbullet({\sl Show item \ref{bpfa three}}) Use the fact that $A\cup B$ can be written as a disjoint union ${A \cup (B-A\cap B)}$ to get that
\begin{align*}
P[A\cup B]&= P[A] + P[B-A\cap B] \\
&= P[A] + P[B] - P[A\cap B].
\end{align*}
\\
\textbullet({\sl Show item \ref{bpfa four}}) Use induction.
\\
\textbullet({\sl Show item \ref{bpfa five}}) Use induction and inclusion exclusion.
\\
\textbullet({\sl Show item \ref{bpfa six}})
For the first equation use the fact that $\cup_k B_k - \cup_k A_k$ and $\cap_k B_k - \cap_k A_k$  are covered by $\cup_k(B_k - A_k)$ and then apply sub-additivity.
\\
\textbullet({\sl Show item \ref{bpfa seven}})
Assume $A_1, A_2, \ldots \in \mathcal F_0$ and $A_n\uparrow A\in\mathcal F_0$. Then
 \begin{align*}
 P[A_n] &= P\big[ \textstyle\bigcup_{k=1}^n A_k \big],\qquad\text{since $A_1\subset A_2\subset \cdots$}\\
 &= P\big[ \textstyle\bigcup_{k=1}^n \underbrace{A_k - A_{k-1}}_{\text{disjoint}} \big] \\
 &= \textstyle\sum_{k=1}^nP\big[ A_k - A_{k-1} \big] \\
 &\uparrow \textstyle\sum_{k=1}^\infty P\big[ A_k - A_{k-1}\big] \\
 &=P\big[ \textstyle\bigcup_{k=1}^\infty  A_k - A_{k-1} \big],\,\,\text{by item 1.}\\
 &=P[ A ]
 \end{align*}
 \\
\textbullet({\sl Show item \ref{bpfa eight}})
Use the fact that $A_n\downarrow A \Longleftrightarrow A_n^c \uparrow A^c$.
\\
\textbullet({\sl Show item \ref{bpfa nine}})
 \begin{align*}
  P\Bigl[\textstyle\bigcup_{k=1}^\infty A_k \Bigr]
  & =  P\Bigl[\setlimup{n}\textstyle\bigcup_{k=1}^n A_k \Bigr] \\
  & =  \setlimup{n} P\Bigl[\textstyle\bigcup_{k=1}^n A_k \Bigr],\,\text{continuity from below}\\
  & \leq  \setlimup{n} \textstyle\sum_{k=1}^n P\bigl[A_k \bigr],\,\text{finite sub-additivity}\\
  & =  \textstyle\sum_{k=1}^\infty P\bigl[A_k \bigr].
 \end{align*}
\end{proof}



\begin{theorem}[{\bf Basic properties of measures}]
Suppose $\mu$ is a probability measure on field $\mathcal F_0$ of $\Omega$ subsets. Then each of the following statements hold for all $\mathcal F_0$-sets $A, A_1, \ldots, B, B_1, \ldots$
\begin{enumerate}
\item $\mu[A^c] = \mu[\Omega]-\mu[A]$ if \textcolor{red}{$\mu[\Omega]<\infty$};
\item\label{bpfa one} If $A\subset B$ and \textcolor{red}{$\mu[A]<\infty$} then $\mu[B-A]=\mu[B]-\mu[A]$;
\item\label{bpfa two} {\bf (Increasing)} If $A\subset B$ then  $\mu[A]\leq \mu[B]$;
\item\label{bpfa three} {\bf (Inclusion-exclusion)} $\mu[A\cup B] = \mu[A]+ \mu[B - A\cap B]$ and $\mu[A\cup B] = \mu[A]+ \mu[B] - \mu[A\cap B]$ if \textcolor{red}{$\mu[A\cap B]<\infty$};
\item\label{bpfa four} {\bf (Finite additivity)}  If $A_k$'s are disjoint then
$\mu\bigl[ \bigcup_{k=1}^n A_k \bigr] = \sum_{k=1}^n \mu[A_k]$;
\item\label{bpfa five} {\bf (Finite sub-additivity)}
$\mu\bigl[ \bigcup_{k=1}^n A_k \bigr] \leq \sum_{k=1}^n \mu[A_k]$.
\item{(\bf Continuous from below)} \\ If  $A_n\uparrow A\in \mathcal F_0$ then $\mu[A_n]\uparrow \mu[A]$;
\item{(\bf Continuous from above)} \\ If $A_n\downarrow A\in \mathcal F_0$ and \textcolor{red}{there exists an $n$ such that $\mu[A_n]<\infty$} then $\mu[A_n]\downarrow \mu[A]$.
\item{(\bf Countable sub-additivity)} If $\cup_{k=1}^\infty A_k \in \mathcal F_0$  then $\mu\bigl[\sum_{k=1}^\infty A_k\bigr] \leq \sum_{k=1}^\infty \mu[A_k]$;
\end{enumerate}
\end{theorem}




%
%
% \begin{theorem}[{\bf Basic measure facts}]
% Let $(\Omega, \mathcal F, \mu)$ be a measure space. Then
% \begin{enumerate}
% \item If $A_1, A_2,\ldots, A_n$ are disjoint $\mathcal F$-sets then \\$\mu\bigl(\sum_{k=1}^n A_k\bigr) = \sum_{k=1}^n \mu(A_k)$.
% \item If $A\subset B$ are $\mathcal F$-sets then \\$\mu(A)\leq \mu(B)$.
% \item If $A\subset B$ are $\mathcal F$-sets and {$\mu(A)<\infty$} then \\$\mu(B-A)=\mu(B)-\mu(A)$.
% \end{enumerate}
% \end{theorem}
%



The following theorem tells us that a consequences of the countable additivity which do not follow from finite additivity.


\begin{theorem}[{\bf Countable additivity equivalence}]
\label{thm: cae}
Let $P:\mathcal F_0\rightarrow [0,1]$ be a finitely additive probability on a field $\mathcal F_0$. Then the following statements are equivalent:
\begin{enumerate}
\item $P$ is a probability measure;
\item $P$ is continuous from below;
\item $P$ is continuous from above;
%({\bf Continuous from below}) If whenever $A_1, A_2, \ldots \in \mathcal F_0$ and $A_n\uparrow A\in\mathcal F_0$ then $P(A_n)\uparrow P(A)$;
%\item ({\bf Continuous from above})  If whenever $A_1, A_2, \ldots \in \mathcal F_0$ and $A_n\downarrow A\in\mathcal F_0$ then $P(A_n)\downarrow P(A)$;
\item ({\bf Continuous from above at $\varnothing$}) If whenever $A_1, A_2, \ldots \in \mathcal F_0$ and $A_n\downarrow  \varnothing$ then $P(A_n)\downarrow 0$.
\end{enumerate}
\end{theorem}
\begin{proof}

By a previous theorem already have {\sl 1. $\Longrightarrow $ 2. $\Longrightarrow $ 3. $\Longrightarrow $ 4. }



 ({\sl 4. $\Longrightarrow $ 3.}) Suppose $P$ is continuous from above at $\varnothing$. Now suppose  $A_1, A_2, \ldots \in \mathcal F_0$ and $A_n\downarrow A\in\mathcal F_0$. Now to show item 3 notice
 \begin{align*}
 A_n\downarrow A
 &\Longrightarrow A_n-A \downarrow \varnothing \\
 &\Longrightarrow P[A_n]-P[A] = P[A_n-A]   \downarrow 0,\,\,\text{by item 4.} \\
 &\Longrightarrow P[A_n] \downarrow P[A]  \\
 \end{align*}


 ({\sl 2. $\Longleftrightarrow $ 3.}) Use the fact that $A_n\uparrow A \Longleftrightarrow A_n^c \downarrow A^c$ along with the identity $P[A] = 1- P[A^c]$.


({\sl 2. $\Longrightarrow $ 1.}) The only thing to show is countable additivity over disjoint sets which stay in the field. In paritcular suppose $A_1, A_2, \ldots $ are disjoint $\mathcal F_0$-sets  such that $\textstyle\bigcup_{k=1}^\infty A_k \in \mathcal F_0$. Then
\begin{align*}
P\Bigl[\textstyle\bigcup_{k=1}^\infty A_k \Bigr]= P\Bigl[\setlimup{n}\textstyle\bigcup_{k=1}^n A_k \Bigr] = \setlimup{n}P\Bigl[\textstyle\bigcup_{k=1}^n A_k \Bigr]
 \end{align*}
 where the last equality follows by assuming item 2.


\end{proof}




%%%%%%%%%%%%%%%%%
\begin{theorem}[{\bf Uniqueness for probability measures}]
\label{ui}
Let $\mathcal P$ be a collection of subsets of $\Omega$.
If $P$ and $Q$ are two probability measures on $(\Omega, \sigma\langle \mathcal P\rangle)$ such that
\begin{enumerate}
\item $P$ and $Q$ agree on $\mathcal P$;
\item $\mathcal P$ is a $\pi$-system,
\end{enumerate}
then $P$ and $Q$ agree on all of $ \sigma\langle \mathcal P\rangle$.
\end{theorem}
\begin{proof}
This is our first use of Dynkin's $\pi-\lambda$ theorem which allows us to extend the good sets principle. In particular, define the good sets as follows:
\begin{equation}
\mathcal G:=\{A\subset \Omega\colon Q[A] = P[A] \}.
\end{equation}
Dynkin's $\pi-\lambda$ theorem says that $\sigma\langle \mathcal P\rangle = \lambda\langle \mathcal P\rangle$ since $\mathcal P$ is a $\pi$-system. Therefore to show $\sigma\langle \mathcal P\rangle = \lambda\langle \mathcal P\rangle\subset \mathcal G$ we just show that $\mathcal G$ is a $\lambda$-system and invoke {\it good sets}.

\textbullet({\sl $\Omega \in \mathcal G$}) This is trivial since $Q[\Omega]=1$ and $P[\Omega]=1$  by properties probability measures.
\\
\textbullet({\sl $A\in \mathcal  G\Longrightarrow A^c\in \mathcal G$})
\begin{align*}
A\in \mathcal G
&\Longrightarrow Q[A] = P[A]  \\
&\Longrightarrow 1-Q[A^c] = 1-P[A^c] \\
&\Longrightarrow Q[A^c] = P[A^c] \\
&\Longrightarrow A^c\in \mathcal G.
\end{align*}
\\
\textbullet({\sl disjoint $A_1, A_2\in \mathcal  G\Longrightarrow \bigcup_{k=1}^\infty A_k\in \mathcal G$})
\begin{align*}
\underbrace{A_1,A_2,\ldots}_{\text{disjoint}}\in \mathcal G
&\Longrightarrow  Q[A_k] = P[A_k],\,\forall k  \\
&\Longrightarrow \underbrace{\sum_{k=1}^\infty Q[A_k]}_{=Q[\cup_{k=1}^\infty A_k]} = \underbrace{\sum_{k=1}^\infty P[A_k]}_{=P[\cup_{k=1}^\infty A_k]}  \\
&\Longrightarrow \bigcup_{k=1}^\infty \in \mathcal G.
\end{align*}
Notice that this last statement could not be proved if the $A_k$'s were not disjoint. This illustrates the necessity of Dynkin's $\pi-\lambda$ theorem.

\end{proof}





%%%%%%%%%%%%%%%%%
\begin{theorem}[{\bf Uniqueness for measures}]
\label{uui}
If $\mu_1$ and $\mu_2$ are measures on $(\Omega, \sigma\langle \mathcal P\rangle)$ such that
\begin{enumerate}
\item $\mu_1$ and $\mu_2$ agree on $\mathcal P$;
\item $\mathcal P$ is a $\pi$-system;
\item \textcolor{red}{$\mu_1$ and $\mu_2$ are $\sigma$-finite on $\mathcal P$},
\end{enumerate}
then $\mu_1$ and $\mu_2$ agree on all of $ \sigma\langle \mathcal P\rangle$.
\end{theorem}





\begin{definition}[{\bf $\mu$-null and $\mu$-neg}]
%$\phantom{asdf}$
Let $(\Omega,\mathcal F, \mu)$ be a measure space. Then
%then
\begin{itemize}
\item
A set $A\in \mathcal F$ is said to be {$\mu$-null}  if $\mu(A)=0$.
\item
A set $A\in 2^\Omega$ is said to be {$\mu$-negligible} if there exists a $\mu$-null set  $B\in \mathcal F$ such that $A\subset B$.
\end{itemize}
\end{definition}





\begin{definition}[{\bf Complete}]
A measure space $(\Omega, \mathcal F, \mu)$ is said to be complete if all the $\mu$-negligible sets belong to $\mathcal F$.
\end{definition}



\begin{definition}[{\bf Complete}]
A probability space $(\Omega, \mathcal F, P)$ is said to be {\bf complete} if all the $P$-negligible sets belong to $\mathcal F$.
\end{definition}



\begin{theorem}[{\bf The completion $(\Omega, \bar {\mathcal F},\bar \mu)$}]
Let $(\Omega,\mathcal F, \mu)$ be a measure space and let $\mathcal N_\mu$ be the collection of $\mu$-negligible sets.
%Let $\bar {\mathcal F}:=\sigma\langle \mathcal F, \mathcal N_\mu\rangle$ and $\bar \mu$
Then
\begin{itemize}
\item $\bar {\mathcal F}:= \sigma\langle \mathcal F, \mathcal N_\mu\rangle = \{F\cup N: F\in \mathcal F, N\in \mathcal N_\mu  \}$;
\item The set function $\bar \mu$ on $\bar{\mathcal F}$ defined by $\bar\mu(F\cup N)= \mu(F)$ for $F\in\mathcal F$ and $N\in \mathcal N_\mu$ is the unique extension of $\mu$ to a measure on $(\Omega, \bar{\mathcal F})$;
\item The measure space $(\Omega, \bar{\mathcal F}, \bar \mu)$ is complete.
\end{itemize}
The triple $(\Omega, \bar{\mathcal F}, \bar \mu)$ is called the {completion} of  $(\Omega, \mathcal F, \mu)$.
\end{theorem}





\begin{exercise}
Suppose that $\mu_1$ and $\mu_2$ are measures on  $\sigma\langle \mathcal F_0\rangle$ generated by a class $\mathcal F_0$. Suppose also that the inequality
\begin{equation}
\label{fourteen}
\mu_1(A)\leq \mu_2(A)
\end{equation}
holds for all $A$ in $\mathcal F_0$. (a) Show that if $\mathcal F_0$ is a field and $\mu_1$ and $\mu_2$ are $\sigma$-finite on $\mathcal F_0$, then (\ref{fourteen}) holds for all $A\in\sigma\langle \mathcal F_0\rangle$. (b) Show by examples that (\ref{fourteen}) can fail for some $A\in\sigma\langle \mathcal F_0\rangle$ if: $\mathcal F_0$ is a field but $\mu_1$ and $\mu_2$ are only $\sigma$-finite  overall, not $\sigma$-finite on $\mathcal F_0$; or if $\mu_1$ and $\mu_2$ are $\sigma$-finite on $\mathcal F_0$, but $\mathcal F_0$ is only a $\pi$-system. Hint: for (a) first treat the case where $\mu_2$ is finite.
\end{exercise}




%%%%%%%%%%%%%%%%%%%
The following exercises shows that it doesn't matter what order we sum an infinite number of positive terms. This is useful for rigorously showing the measure axioms for counting measure on any $\Omega$. Notice that if there are negative terms, order does matter.
\begin{exercise}
Let $I$ be an infinite set and let $f$ be a function from $I$ to $[0,\infty]$. The sum over $f$ over $I$ is defined as
\[ \sum_{i\in I} f(i):=\sup\Bigl\{ \sum_{i\in H} f(i): \text{ $H\subset I$, $H$ is finite}  \Bigr\}. \]
Show that: (a) for any partition $I = \bigcup_{k\in K} I_k$ of $I$ into nonempty disjoint subsets $I_k$,
\[  \sum_{i\in I} f(i) =  \sum_{k\in K}\Bigl(\sum_{i\in I_k} f(i)\Bigr) \]
and (b) if $I$ is countable, then
\[  \sum_{i\in I} f(i) = \lim_{n\rightarrow \infty} \sum_{m=1}^n f(i_m) \]
for any enumeration $i_1,i_2,\ldots$ of the points in $I$.
\end{exercise}





\begin{exercise}
Let $\Omega = \Bbb R$ and $\mathcal B_0(\Bbb R):= f\langle (-\infty,a]: -\infty < a <\infty  \rangle$ be the Borel field of $\Bbb R$. Let $P$ be a finitely additive probability on $\mathcal B_0(\Bbb R)$. Show that $P$ is a probability measure on  $\mathcal B_0(\Bbb R)$ if and only if the function defined by $F(x):= P((-\infty, x])$ is non-decreasing, right-continuous and satisfies $\lim_{x\rightarrow -\infty} F(x)=0$ and  $\lim_{x\rightarrow \infty} F(x)=1$.
%(a) Show that the formula
%\[ F(x):= P((-\infty,x]) \]
% establishes  a one-to-one correspondence between the finitely additive probabilities on $\mathcal B_0(\Bbb R)$ and the set of nondecreasing functions $F:{\Bbb R}\rightarrow \Bbb R$ such that $F(0)=0$ and $F(1)=1$. (b) Show that there is a one-to-one correspondence between the probability measures on $\mathcal B_0(\Bbb R)$ and the set of nondecreasing functions $F: \Bbb R \rightarrow \Bbb R$ such that $F(0)=0$, $F(1)=1$ and $F$ is right continuous.
\end{exercise}
%


%-------------------------------'
%---------section  ---------------'
%-------------------------------'
\subsection{Carath\'eodory Extension Theorem}
%-------------------------------'
%---------section  ---------------'
%-------------------------------'


\begin{sectionassumption}
\label{outer}
For the remainder of this section $P_0$ denotes a probability measure on $\mathcal F_0$, where $\mathcal F_0$ is a field on $\Omega$. Also let $\mathcal F^\uparrow, \mathcal F^\downarrow,  \bar{\mathcal F}, P^\uparrow, P^\downarrow, P^*, P_*, \bar P$ be defined as follows
\begin{itemize}
\item  $\mathcal F^\uparrow := \{\setlimup{k}  A_k : A_k\in \mathcal F_0  \}$
\item  $\mathcal F^\downarrow := \{\lim_k\!\!\downarrow A_k : A_k\in \mathcal F_0  \}$
\item $P^\uparrow(\lim_k\!\!\uparrow A_k) := \lim_k P_0(A_k)$  when $\lim_k\!\!\uparrow A_k\in \mathcal F^\uparrow$
\item $P^\downarrow(\lim_k\!\!\downarrow A_k) := \lim_k P_0(A_k)$ when  $\lim_k\!\!\downarrow A_k\in \mathcal F^\downarrow$
\item  $P^*(A):=\inf\{P^\uparrow(A^\uparrow): A\subset A^\uparrow\in \mathcal F^\uparrow  \}$ when $A\in2^\Omega$
\item $P_*(A):=\sup\{P^\downarrow(A^\downarrow): A\supset A^\downarrow\in \mathcal F^\downarrow  \}$ when $A\in2^\Omega$
\item  $\bar{\mathcal F}:=\{ A\in 2^\Omega : P^*(A) = P_*(A)\}$
\item $\bar P(A):=P^*(A)=P_*(A)$ when $A\in \bar{\mathcal F}$
%\item $\mathcal F :=\sigma\langle \mathcal F_0 \rangle$. Note: we will show $\mathcal F\subset \overline{\mathcal F}$.
%\item $P(A):= \overline P(A)$ for all $A\in \mathcal F$.
\end{itemize}
\end{sectionassumption}



\begin{theorem}
\label{thm: well defined Pup, P*}
 $P^\uparrow$, $P^\downarrow$, $P^*$ and $P_*$ are all well defined. Moreover,  $(2^\Omega, P^*)$ is an extension of $(\mathcal F^\uparrow, P^\uparrow)$ which is an extension of $(\mathcal F_0, P_0)$ (and similarly for $(2^\Omega, P_*)$ and $(\mathcal F^\downarrow, P^\downarrow)$).
 \end{theorem}
\begin{proof}
({\sl Show $P^\uparrow$ is well defined})
Notice that if $A_n \uparrow A$ then  $\lim_n P_0[A_n]$ exists by monotonicity and boundedness of $P_0[A_n]$. Therefore we just need to show that  $\lim_n P_0[A_n] = \lim_n P_0[B_n]$ whenever $\setlimup{n} A_n =  \setlimup{n} B_n$.
It will be sufficient to show that for any $A_n, B_n\in \mathcal F_0$ we have
\begin{equation}
\label{eq: increasing Pup}
\setlimup{n} A_n \subset   \setlimup{n} B_n \Longrightarrow \lim_n P_0[A_n] \leq \lim_n P_0[B_n].
\end{equation}
Notice that if $\setlimup{n} A_n \subset   \setlimup{n} B_n$ then
$A_n = A_n \cap \bigl(\setlimup{m} B_m\bigr) =  \setlimup{m} (A_n\cap B_m).
$
Therefore
\begin{align*}
P_0[A_n]  & = P_0\bigl[\setlimup{m} \underbrace{(A_n\cap B_m)}_{\in \mathcal F_0}\bigr] \\
& = \setlimup{m} P_0 [(A_n\cap B_m)],\quad\text{since $P_0$ is a prob measure} \\
& \leq  \setlimup{m} P_0[B_m].
\end{align*}
Now taking limits of both sides in $n$ gives $\setlimup{n}P_0[A_n]\leq  \setlimup{m} P_0\bigl[B_m\bigr]$ as was to be shown. Notice that (\ref{eq: increasing Pup}) implies  increasingness of $P^\uparrow$. In particular
\begin{equation}
\label{eq: increasing Pup II}
\text{$A^\uparrow, B^\uparrow \in \mathcal F^\uparrow$ and  $A^\uparrow\subset B^\uparrow$}
 \Longrightarrow P^\uparrow[A^\uparrow]\leq P^\uparrow[B^\uparrow].
\end{equation}


({\sl Show $(\mathcal F^\uparrow, P^\uparrow)$ extends $(\mathcal F_0, P_0)$})
We need to show that $\mathcal F_0\subset \mathcal F^\uparrow$ and $P^\uparrow = P_0$ on $\mathcal F_0$.
Clearly $\mathcal F_0\subset \mathcal F^\uparrow$ holds since  any $A\in \mathcal F_0$ can be trivially written as $A = \setlimup{n} A$. The second statement follows since whenever $A\in \mathcal F_0$ we have that
\begin{equation}
P^\uparrow[A] := \setlimup{n}P_0[A] = P_0[\setlimup{n}A] =   P_0[A]
\end{equation}
where the second equality follows since we are assuming $P_0$ is a probability measure.



({\sl Show $P^*$ is well defined}) Trivial.


({\sl Show $(2^\Omega, P^*)$ extends $(\mathcal F^\uparrow, P^\uparrow)$}) Trivially we have $\mathcal F^\uparrow \subset 2^\Omega$. Also notice that  if $A \in \mathcal F^\uparrow$ then
\begin{align*}
P^\uparrow(A) &\leq \underbrace{\inf\{P^\uparrow(A^\uparrow): A\subset A^\uparrow\in \mathcal F^\uparrow  \}}_{ = P^*[A]} \leq P^\uparrow(A).
\end{align*}
where the first inequality is given by (\ref{eq: increasing Pup II}) and the second inequality follows since $A\subset A\in \mathcal F^\uparrow$ is one of the covers in the infimum.

The proofs for $(\mathcal F^\downarrow, P^\downarrow)$ and $(2^\Omega, P_*)$ follow in a similar manner (after noticing $A_n\uparrow A \Longleftrightarrow A_n^c \downarrow A^c$).

\end{proof}


\begin{theorem}[{\bf 5 facts about $P^*$ and $P_*$}]
\label{f5}
For all sets $A, B,C, A_1, \ldots\in 2^\Omega$
\begin{enumerate}
\item\label{eff1}  $P^* (A) + P_*(A^c) = 1$.
\item\label{eff2}  If $A\subset B\subset C$ then $P_* (A)\leq P_*(B)\leq P^*(B)\leq P^*(C)$.
\item\label{eff3}  $P^*(A\cup B)\leq P^*(A) + P^*(B) - P^*(A\cap B)$.
\item\label{eff4}  $P_*(A\cup B)\geq P_*(A) + P_*(B) - P_*(A\cap B)$
%\item\label{eff5}  $P^*(B\cup C) - P^*(A\cup C) \leq P^*(B)-P^*(A)$.
%\item If $A_k\subset B_k$ then $P^*(\cup_{k=1}^n B_k) - P^*(\cup_{k=1}^n
%A_k)\leq \sum_{k=1}^n \bigl[P^*(B_k) - P^*(A_k)\bigr]$. Approximating unions term-by-term.
\item\label{eff6} If $ A_n \uparrow A$ then $P^*(A_n)\uparrow P^*(A)$.
%\item\label{eff7} If  $A_m\subset B_m$ for each $m$. Then
%\[ P^*(\cup_{m=1}^n B_m)- P^*(\cup_{m=1}^n A_m)\leq \sum_{m=1}^n P^*( B_m)- P^*( A_m). \]
\end{enumerate}
\end{theorem}
\begin{proof}
These are a tedious and not very insightful so we will skip the proof in this class.
\end{proof}


\begin{theorem}[{\bf Carath\'eodory extension theorem}]
\label{thm: Caratheodory}
The probability measure $P_0$ on $\mathcal F_0$ has a unique extension to a probability measure $P$ on $\sigma\langle \mathcal F_0\rangle=:\mathcal F$.
\end{theorem}
\begin{proof}
Notice that the uniqueness follows from Theorem \ref{ui} since $\mathcal F_0$ is already a $\pi$-system. Therefore all we need to show is that $\bar{\mathcal F}$ is a $\sigma$-field containing $\mathcal F_0$ and $\bar P$ is a probability measure on $\bar{\mathcal F}$.

({\sl Show $\mathcal F_0\subset \bar{\mathcal F}$ }) In particular we need to show  $A\in \mathcal F_0\Longrightarrow P^*(A) = P_*(A)$. This follows directly by the fact that $(2^\Omega,P^*)$ and $(2^\Omega,P_*)$ are extensions of $(\mathcal F_0, P_0)$ by Theorem  \ref{thm: well defined Pup, P*}.

({\sl Show $\bar{\mathcal F}$ is a field })\\
\textbullet($\Omega\in \bar{\mathcal F}$) Just showed $\mathcal F_0\subset \bar{\mathcal F}$ and  $\Omega\in \mathcal F_0$.
\\
\textbullet($A\in \bar{\mathcal F}\Longrightarrow A^c \in \bar{\mathcal F}$)
Suppose $A\in \bar{\mathcal F}$. Then
\begin{align}
P^*(A^c) &= 1 - P_*(A),\quad\text{by Theorem \ref{f5}.\ref{eff1}} \nonumber\\
	&=1 - P^*(A),\quad\text{since $A\in \bar{\mathcal F}$} \nonumber\\
	&=P_*(A^c),\quad\text{by Theorem \ref{f5}.\ref{eff1}}. \label{eq: comp for P*}
\end{align}
Therefore  $A^c\in \bar{\mathcal F}$.
\\
\textbullet($A, B\in \bar{\mathcal F}\Longrightarrow A\cup B \in \bar{\mathcal F}$)
Suppose $A, B\in \bar{\mathcal F}$. Then
\begin{align}
P^*(A\cup B) &\leq P^*(A)+P^*(B)-P^*(A\cap B)  ,\quad\text{by Theorem \ref{f5}.\ref{eff3}} \nonumber\\
	&=P_*(A)+P_*(B)-P^*(A\cap B) ,\quad\text{since $A,B\in \bar{\mathcal F}$} \nonumber\\
	&\leq P_*(A)+P_*(B)-P_*(A\cap B) ,\quad\text{by Theorem \ref{f5}.\ref{eff2}} \nonumber\\
	&\leq P_*(A\cup B),\quad\text{by Theorem \ref{f5}.\ref{eff4}}\nonumber\\
	&\leq P^*(A\cup B),\quad\text{by Theorem \ref{f5}.\ref{eff2}}. \label{eq: inc-ex for P*}
\end{align}
For one thing, this implies $P^*(A\cup B) = P_*(A\cup B)$ so that $A\cup B\in \bar{\mathcal F}$ as was to be shown.



({\sl Show $\bar{\mathcal F}$ is a monotone class }) Since $\bar{\mathcal F}$ is a field we simply show that $\bar{\mathcal F}$ is closed under monotonically increasing and decreasing limits. In particular, let $A_n\in \bar{\mathcal F}$ such that $A_n\uparrow A$. Then
\begin{align*}
P^*(A) &= \lim_{n} P^*(A_n),\quad\text{by Theorem \ref{f5}.\ref{eff6} } \\
&= \lim_{n} P_*(A_n),\quad\text{since $A_n\in \bar{\mathcal F}$} \\
&\leq  \lim_{n} P_*(A),\quad\text{by Theorem \ref{f5}.\ref{eff2} } \\
&=  P_*(A) \\
&\leq  P^*(A),\quad\text{by Theorem \ref{f5}.\ref{eff2} } \label{eq: inc-ex for P*}\\
\end{align*}
To show closure under decreasing limits just use the fact that $A_n\uparrow A\Longleftrightarrow A_n^c \downarrow A^c$ and equation (\ref{eq: comp for P*}).





({\sl Show $\bar P$ is a measure on $\bar{\mathcal F}$}) By Theorem \ref{thm: cae} it will be sufficient to show that $\bar P$ is a FAP and $\bar P$ is continuous from below. \\
\textbullet({\sl $\bar P$ is a FAP})
By extension facts $\bar P(\Omega) = P^*(\Omega) = P_0(\Omega) = 1$. (\ref{eq: comp for P*}) shows that $\bar P(\varnothing) = P^*(\varnothing) =1- P^*(\Omega) = 0$. Also, (\ref{eq: inc-ex for P*}) establishes inclusion exclusion for $P^*$ on $\bar{\mathcal F}$. Therefore whenever $A, B\in \bar{\mathcal F}$ and $A\cap B=\varnothing$ we get $\bar P(A\cup B)=\bar P(A)+\bar P(B)$. Therefor $\bar P$ is a FAP.
\\
\textbullet({\sl $\bar P$ is continuous from below}) Trivial from Theorem \ref{f5}.\ref{eff6}.
\end{proof}

\begin{theorem}
 $(\mathcal F, P)$ is an extension of both  $(\mathcal F^\uparrow, P^\uparrow)$ and   $(\mathcal F^\downarrow, P^\downarrow)$.
\end{theorem}
\begin{proof}
Clearly both $\mathcal F^\uparrow\subset \mathcal F$ and   $\mathcal F^\downarrow\subset \mathcal F$ by closure properties of $\mathcal F$ (in particular that any $\sigma$-field is also a monotone class). Let $A^\uparrow \in  \mathcal F^\uparrow$. Then
\begin{align*}
P^\uparrow(A^\uparrow) &= P^*(A^\uparrow),\quad\text{since $P^*$ extends $P^\uparrow$ by Thm \ref{thm: well defined Pup, P*}} \\
&= \bar P(A^\uparrow),\quad\text{since $A^\uparrow\in \bar{\mathcal F}$} \\
&= P(A^\uparrow),\quad\text{since $A^\uparrow\in \mathcal F$}.
\end{align*}
Therefore $P$ extends $P^\uparrow$. A similar proof establishes the desired result for $P^\downarrow$.
\end{proof}

\begin{sectionassumption}
\label{outer2}
For the remainder of this section let $P$ denote the probability measure  on $\sigma\langle \mathcal F_0\rangle$ which is the unique extension of  $P_0$ on $\mathcal F_0$. Also let $\mathcal F:=\sigma\langle \mathcal F_0\rangle$.
\end{sectionassumption}

\begin{theorem}[{\bf Easier formula for $P^*$}]
\label{thm: Easier formula for $P^*$}
%Let $\mathcal F_0$ be a field of $\Omega$-sets and $P_0$ be a probability measure on $\mathcal F_0$. Let $P$ denote the unique extension of $P_0$ to $(\Omega, \sigma\langle \mathcal F_0\rangle)$.
%Then f
For all $A\subset\Omega$
\begin{enumerate}
\item\label{item1: Easier formula for $P^*$} $P^*(A)=\inf\{ P(B): A\subset B\in   \mathcal F \}$;
\item\label{item2: Easier formula for $P^*$} $P_*(A)=\sup\{ P(B): A\supset B\in  \mathcal F \}$.
\end{enumerate}
Moreover, the above infimum and supremum are attained.
%where $P^*$ and $P_*$ are the outer and inner measures corresponding to $P_0$.
\end{theorem}
\begin{proof}
\begin{align*}
P^*(A) &:= \inf\{P^\uparrow(A^\uparrow): A\subset A^\uparrow\in   \mathcal F^\uparrow   \} \\
&=   \inf\{P^*(A^\uparrow): A\subset A^\uparrow\in   \mathcal F^\uparrow   \},\quad\text{$P^*$ extends $P^\uparrow$} \\
&\geq  \inf\{\underbrace{P^*(B)}_{=P(B)}: A\subset B\in   \mathcal F  \},\quad\text{inf over larger set} \\
&\geq  P^*(A)
\end{align*}
where the last inequality follows since $P^*(B)\geq P^*(A)$ (by Theorem \ref{f5}.\ref{eff2}). A similar proof establishes the result for $P_*$.

To see why the infimum is attained let $A\subset B_n\in \mathcal F$ such that $P(B_n)\rightarrow P^*(A)$. Now
\begin{equation}
\label{eq:easier formula}
P\Bigl( \bigcap_{n=1}^\infty B_n\Bigr) = \setlimdown{N} P\Bigl( \bigcap_{n=1}^N B_n\Bigr) \leq \lim_N P( B_N) = P^*(A).
\end{equation}
Therefore
\begin{align*}
P\Bigl( \bigcap_{n=1}^\infty B_n\Bigr)
&\leq P^*(A),\quad\text{by (\ref{eq:easier formula})}\\
&= \inf\{ P(B): A\subset B\in   \mathcal F \} \\
&\leq  P\Bigl( \bigcap_{n=1}^\infty B_n\Bigr)
\end{align*}
where the  last inequality follows from the fact that  $A\subset \bigcap_{n=1}^\infty B_n\in \mathcal F$ .
% Since $A\subset \bigcap_{n=1}^\infty B_n\in \mathcal F$ we have
% \begin{equation}\label{eq:easier form I}\inf\{ P(B): A\subset B\in   \mathcal F \} \leq  P\Bigl( \bigcap_{n=1}^\infty B_n\Bigr). \end{equation}
% Moreover equation (\ref{eq:easier formula}) establishes that
% \begin{equation}\label{eq:easier form II}P\Bigl( \bigcap_{n=1}^\infty B_n\Bigr) \leq \inf\{ P(B): A\subset B\in   \mathcal F \}. \end{equation}
% Both (\ref{eq:easier form I}) and (\ref{eq:easier form II}) establish that the infimum is attained.
Therefore the infimum is attained as was to be shown.
A similar proof is used for the supremum.
\end{proof}

Although the above infimum and supremum are attained in Theorem \label{thm: Easier formula for $P^*$} notice that the following  infimum and supremum are {\bf not} necessarily attained:
\begin{align*}
P^*(A)&:=\inf\{P^\uparrow(A^\uparrow): A\subset A^\uparrow\in \mathcal F^\uparrow  \}\\
P_*(A)&:=\sup\{P^\downarrow(A^\downarrow): A\supset A^\downarrow\in \mathcal F^\downarrow  \}.
\end{align*}
The following theorem is as close as we can get working with $\mathcal F^\uparrow$ and $\mathcal F^\downarrow$.

\begin{theorem}[{\bf Approximating $P$ with $\mathcal F^\uparrow$}]
%Let $\mathcal F_0$ be a field of $\Omega$-sets and $P_0$ be a probability measure on $\mathcal F_0$. Let $P$ denote the unique extension of $P_0$ to $(\Omega, \sigma\langle \mathcal F_0\rangle)$. Let  $\mathcal F^\uparrow$ and $\mathcal F^\downarrow$  be as in definition \ref{outer}. Then
For all $A\in \mathcal F$ there exists $\mathcal F^\downarrow$-sets $A^\downarrow_n$ and $\mathcal F^\uparrow$-sets $A_{n}^\uparrow$ such that
\begin{itemize}
\item $\bigcup_{n=1}^\infty A^\downarrow_{n}\subset A \subset \bigcap_{n=1}^\infty A^\uparrow_{n}$;
\item $P\Bigl(\bigcup_{n=1}^\infty A^\downarrow_{n}\Bigr) = P(A) = P\Bigl( \bigcap_{n=1}^\infty A^\uparrow_{n}\Bigr)$.
\end{itemize}
\end{theorem}
\begin{proof}
Let $A\subset A_n^\uparrow\in \mathcal F^\uparrow$ such that
\begin{align*}
\lim_n P(A_n^\uparrow) &= \inf\{P^\uparrow(A^\uparrow): A\subset A^\uparrow \in \mathcal F^\uparrow\}\\
 &= \inf\{P(A^\uparrow): A\subset A^\uparrow \in \mathcal F^\uparrow\}.
\end{align*}
By a similar proof as in Theorem \ref{thm: Easier formula for $P^*$} we get $A\subset \bigcap_{n=1}^\infty A_n^\uparrow \in \mathcal F$ and
\[
P\Bigl(\bigcap_{n=1}^\infty A_n^\uparrow\Bigr)\leq \underbrace{\inf\{P(A^\uparrow): A\subset A^\uparrow \in \mathcal F^\uparrow\} }_{\text{$=P(A)$ when $A\in \mathcal F$}}\leq P\Bigl( \bigcap_{n=1}^\infty A_n^\uparrow
\Bigr).\]
The proof for $\mathcal F^\downarrow$ is similar.
\end{proof}


\begin{theorem}[{\bf Approximating $P$ with $\mathcal F_0$}]
\label{approximating P with F0}
%Let $\mathcal F_0$ be a field of $\Omega$-sets and $P_0$ be a probability measure on $\mathcal F_0$. Let $P$ denote the unique extension of $P_0$ to $(\Omega, \sigma\langle \mathcal F_0\rangle)$. Then f
For all $A\in \mathcal F$ and all $\epsilon>0$ there exists $A^o\in\mathcal F_0$ such that
\begin{itemize}
\item $P(A\,\triangle\, A^o)\leq \epsilon$.
\end{itemize}
\end{theorem}
\begin{proof}
If $A\in \mathcal F$ then $P(A) = P^*(A) =  \inf\{P(A^\uparrow)\colon A\subset A^\uparrow \in \mathcal F^\uparrow  \}$, since $P^*$ extends $P$. Therefore one can find $A^\uparrow \in \mathcal F^\uparrow$ such that $A\subset A^\uparrow$ and
\begin{equation}
\label{eq:approx with F0 I}
P(A^\uparrow - A) =  P(A^\uparrow) -  P(A)   \leq \epsilon / 2.
\end{equation}
Note that  $P(A^\uparrow) = P(\setlimup{n}A_n^o) = \setlimup{n}P(A_n^o) $ where $A_n^o\in \mathcal F_0$ and $A_n^o\subset A^\uparrow$. Therefore we can find $A_n^o$ such that
\begin{equation}
\label{eq:approx with F0 II}
P(A^\uparrow -  A_n^o) =  P(A^\uparrow) -  P(A_n^o)   \leq \epsilon / 2.
\end{equation}
Now
\begin{align*}
P(A\,\triangle\, A_n^o)
&\leq P(A\cap (A_n^o)^c) + P(A^c \cap A_n^o)\\
&\leq P(A^\uparrow \cap (A_n^o)^c) + P(A^c \cap A^\uparrow)\\
&= P(A^\uparrow - A_n^o) + P(A^\uparrow - A)\\
&\leq \epsilon,\,\text{by (\ref{eq:approx with F0 I}) and (\ref{eq:approx with F0 II}) }
\end{align*}

\end{proof}



\begin{theorem}[{\bf Use $P^*$ to find $P$-neg sets}]
Let $A\subset \Omega$
%Let $\mathcal F_0$ be a field on $\Omega$ and $P_0$ be a probability measure on $\mathcal F_0$. Let $P$ denote the unique extension of $P_0$ to $(\Omega, \mathcal F)$ where $\mathcal F:=\sigma\langle \mathcal F_0\rangle$. Then
\begin{align}
\text{$A$ is $P$-negligible }&\Longleftrightarrow P^*(A)=0\label{$P$-negligible}\\
&\Longrightarrow A\in \bar{\mathcal F} \label{$P$-null}.
\end{align}
\end{theorem}
\begin{proof}
({\sl Show (\ref{$P$-null})}) This follows since $0\leq P_*(A)\leq P^*(A)$ for all $A\subset \Omega$ by Theorem \ref{f5}.\ref{eff2}.

({\sl Show $\Longrightarrow$ of (\ref{$P$-negligible})}) This follows since
\begin{equation}
\label{eq: negnull 1}
 P^*(A) = \inf\{\underbrace{P(B)}_{\text{one of these is $0$}}\colon A\subset B\in \mathcal F  \}.
 \end{equation}

({\sl Show $\Longleftarrow$ of (\ref{$P$-negligible})})
This follows since the infimum in (\ref{eq: negnull 1}) is attained so that there exists some $B\in \mathcal F$ such that $A\subset B $ and
\[  \underbrace{0 = P^*(A)}_{\text{ by assumption}}=P(B). \]

\end{proof}

The nice  thing about the above theorem is that you can show both $\bar P(A) = 0$ and $A\in \bar{\mathcal F}$ just by establishing $P^*(A)=0$, which you can technically analyze without knowing $A$ is in $\mathcal F$ or $\bar{\mathcal F}$.



\begin{theorem}[{\bf The structure of $(\Omega, \bar {\mathcal F},\bar P)$}]
%Let $\mathcal F_0$ be a field of $\Omega$-sets and $P_0$ be a probability measure on $\mathcal F_0$. Let $P$ denote the unique extension of $P_0$ to $(\Omega, \sigma\langle \mathcal F_0\rangle)$. Let  $\bar P$ and $\bar {\mathcal F}$ be as in definition \ref{outer} and
Let $\mathcal N_P\subset 2^\Omega$ denote the $P$-negligible sets. Then
\begin{itemize}
\item $\bar {\mathcal F}= \sigma\langle \mathcal F, \mathcal N_P\rangle = \{F\cup N: F\in \mathcal F, N\in \mathcal N_P  \}$;
\item $\bar P[F\cup N]=P[F]$ for all $F\in \mathcal F$ and $N\in \mathcal N_P$.
\end{itemize}
\end{theorem}
\begin{proof}
Start by letting
\[
\widetilde{\mathcal F} :=\{F\cup N: F\in \mathcal F, N\in \mathcal N_P  \}.
\]

({\sl Show $\bar{\mathcal F}\subset \widetilde{\mathcal F}$})
Let $C\in \bar{\mathcal F}$ and we try to write $C$ in the form $F\cup N$ where $F\in \mathcal F$ and $N$ is $P$-negligible. Since  $C\in \bar{\mathcal F}$ we have that
\[ P_*[C] = \bar P[C] = P^*[C]. \]
Since the infimum and supermum in Theorem \ref{thm: Easier formula for $P^*$} are attained,
 there exists $C^*\in\mathcal F$ and $C_*\in \mathcal F$ such that $C_* \subset C\subset C^*$ and
\[ P[C_*] = \bar P[C] = P[C^*]. \]
Now we have that
\[
C = C_* \cup (C-C_*).
\]
If we can show that $(C-C_*)$ is $P$-negligible we are done (in particular $C\in \widetilde{\mathcal F}$). To see why notice that $C-C_*  \subset C^*-C_* \in \mathcal F$ which then implies
\begin{align*}
 P[C-C_*]\leq P[C^*-C_*] = P[C^*]-P[C_*] = 0.
\end{align*}

({\sl Show $\widetilde{\mathcal F}\subset \bar{\mathcal F}$ and $\bar P[F\cup N] = P[F]$}) Let $F\cup N\in \widetilde{\mathcal F}$. It will be sufficient to show $P^*[F\cup N] = P_*[F\cup N] = P[F]$. To see why
\begin{align*}
P[F] &= P_*[F],\quad\text{since $F\in \mathcal F$} \\
&\leq P_*[F\cup N],\quad\text{by Theorem \ref{f5}.\ref{eff2}} \\
&\leq P^*[F\cup N],\quad\text{by Theorem \ref{f5}.\ref{eff2}} \\
&\leq P^*[F\cup B],\quad\text{where $N\subset B\in \mathcal F$, $P[B]=0$} \\
&\leq P^*[F] + \underbrace{P^*[B]}_{=P[B]=0} - \underbrace{P^*[F\cap B]}_{\leq P^*[B]=0},\quad\text{by Theorem \ref{f5}.\ref{eff3}} \\
&= P[F].
\end{align*}

({\sl Show $\widetilde{\mathcal F}\subset \sigma\langle \mathcal F, \mathcal N_P  \rangle$}) This is obvious since $F\cup N \in  \sigma\langle \mathcal F, \mathcal N_P  \rangle$ for any $F\in \mathcal F$ and $N\in \mathcal N_P$.

({\sl Show $\sigma\langle \mathcal F,  \mathcal N_P  \rangle\subset \widetilde{\mathcal F}$ }) This follows by {\sl good sets}. Indeed  $\widetilde{\mathcal F}$ is a $\sigma$-field since it equals the $\sigma$-field $\bar{\mathcal F}$. Also clearly $\mathcal F\subset \bar{\mathcal F} = \widetilde{\mathcal F}$. To finish we note that $\mathcal N_P\subset \widetilde{\mathcal F}$ since $N = \varnothing \cup N\in \widetilde{\mathcal  F}$ for any $N\in \mathcal N_P$.
\end{proof}




%%%%%%%%%%%%%%
\begin{theorem}[{\bf Regularity}]
\label{oir}
Let $\mu$ be any measure on $(\Bbb R^d,\mathcal B(\Bbb R^d))$ which assigns finite measure to bounded sets in $\mathcal B(\Bbb R^d)$.
For any  $B\in\mathcal B(\Bbb R^d)$ and $\epsilon>0$ there exists a closed set $C$ and an open set $O$ such that $C\subset B\subset O$ and
\[\mathcal \mu(O-C)<\epsilon.\]
\end{theorem}

%%%%%%%%%%%%%%%%%%%%%%%%
\begin{corollary}
\label{ir}
Let $\mu$ be any measure on $(\Bbb R^d,\mathcal B(\Bbb R^d))$ which assigns finite measure to bounded sets in $\mathcal B(\Bbb R^d)$. Then
\begin{align*}
\mathcal \mu(B)  &= \sup \{ \mathcal \mu(C):  C\subset B,\, \text{$C$ closed}\}\\
  &=\, \inf \{ \mu(O):  B\subset O,\, \text{$O$ open}\}
\end{align*}
%If, in addition,  $\mathcal \mu(B)<\infty$ then
%\[\mathcal \mu(B) = \sup \{ \mathcal \mu(K):  K\subset B,\, \text{$K$ compact}\} .\]
\end{corollary}




\begin{exercise}
Prove Theorem \ref{oir} for $d=1$
\end{exercise}




\begin{exercise}
(a) Prove Corollary \ref{ir} for $d=1$.
(b)
Give an example of a $\sigma$-finite measure $\mu$ on $\mathcal B(\Bbb R)$ and a Borel set $B$ such that
\[ \mu(B-C)= \infty = \mu(O-B) \]
for every closed subset $C$ of $B$ and every open superset $O$ of $B$.
\end{exercise}







%%%%%%%%
\begin{exercise}
\label{l1}
Suppose  $\mathcal F_0$ is a field, $\mu$ is a measure on $\sigma\langle \mathcal F_0\rangle$ and $\mu$ is $\sigma$-finite on $\mathcal F_0$.
\begin{enumerate}
\item
Suppose $B\in \sigma\langle\mathcal F_0\rangle$ and $\epsilon>0$. Show that there exists a disjoint sequence of $\mathcal F_0$-sets $A_1, A_2,\ldots$ such that $B\subset \cup_{n=1}^\infty A_n$ and $ \mu\bigl( \cup_{n=1}^\infty A_n -
 B\bigr)\leq \epsilon$.
 \item Suppose $B\in \sigma\langle\mathcal F_0\rangle$, $\mu(B)<\infty$ and $\epsilon>0$. Show there exists an $\mathcal F_0$-set $A$ such that $\mu(A \,\triangle\, B)\leq \epsilon$.
\item Show by example that the conclusion to 2 may fail if $B$ has infinite measure.
 \end{enumerate}
\end{exercise}


\begin{exerciseproof}

\textbullet({\sl Show 1:})
Since $\mu$ is $\sigma$-finite on $\mathcal F_0$ and $\mathcal F_0$ is a field, one can find disjoint $\mathcal F_0$-sets $F_1, F_2,\ldots$ such that $\Omega = \cup_{n=1}^\infty F_n$ and $\mu(F_n)<\infty$. We start by supposing $\mu(F_n)>0$ for all $n$ and show at the end of the proof to remove this assumption.


Fix $B\in \mathcal F$.
Define  $ \mu_n(\cdot):= \frac{\mu(\cdot\cap F_n)}{\mu(F_n)} $ on $\mathcal F$. Notice that $\mu_n$ is a probability measure on $\mathcal F$ so that
\[ \mu_n(B)=\inf\{ \mu_n(D): B\subset D \in \mathcal F^{\uparrow}\}\leq 1. \]
Therefore one can find a $D_n\in \mathcal F^{\uparrow}$, covering $B$, such that
$B\subset D_n$ and
\begin{align}
\label{aaa}
\underbrace{\mu_n(D_n)-\mu_n(B)}_{=\mu_n(D_n - B)} &\leq \frac{\epsilon}{2^n \mu(F_n)}.
\end{align}
 Now define  $D :=\bigcup_{n=1}^\infty D_n\cap F_n$
and notice that
\begin{align*}
 B\subset D_n, \,\forall n&\Rightarrow (B\cap F_n) \subset (D_n\cap F_n), \,\forall n \\
 &\Rightarrow \bigcup_{n=1}^\infty  (B\cap  F_n) \subset  \bigcup_{n=1}^\infty (D_n\cap F_n) \\
 &\Rightarrow B\subset D.
 \end{align*}
Notice also that each  $D_n\in\mathcal F^\uparrow$ can be written as a disjoint union of $\mathcal F_0$-sets (exercise). Let $D_n = \cup_{m=1}^\infty A_{n,m}$ be such a decomposition (i.e. the $A_{n,m}$'s are  disjoint across $m$ and $A_{n,m}\in\mathcal F_0$). Then
\[D =  \bigcup_{(n,m)\in \Bbb N^+\times \Bbb N^+}  A_{n,m}\cap F_n\]
where the sets $A_{n,m}\cap F_n$ are disjoint (the $F_n$'s are disjoint for different $n$'s and  the $A_{n,m}$'s are disjoint for different $m$'s) and are $\mathcal F_0$-sets. Now
\begin{align}
\mu(D - B) &= \mu(\bigcup_{n=1}^\infty D_n\cap F_n \cap B^c)\nonumber\\
 &= \sum_{n=1}^\infty \mu(D_n\cap F_n \cap B^c)\nonumber\\
 &=  \sum_{n=1}^\infty \mu(F_n)\mu_n(D_n \cap B^c)\nonumber\\
  &=  \sum_{n=1}^\infty \mu(F_n)\underbrace{\mu_n(D_n - B)}_{\leq \epsilon/(2^n\mu(F_n))}\nonumber\\
 &\leq \epsilon.\label{if}
\end{align}
Therefore the class $\{ A_{n,m}\cap F_n\}_{(n,m)\in \Bbb N^+\times \Bbb N^+ }$ gives a countable, disjoint $\mathcal F_0$-set covering of $B$ such that $\mu(\bigcup_{(n,m)\in \Bbb N^+\times \Bbb N^+}  A_{n,m}\cap F_n - B )\leq \epsilon$.

It's easy to extend to the case when some of the $\mu(F_n)=0$ by defining $\mu_n (\cdot):=0$ and $D_n:=F_n$ for these $n$. Then (\ref{if}) still follows.

\textbullet({\sl Show 2:})
Since $\mu(B)<\infty$ we have that
\begin{equation}
 \mu(D)-\mu(B)=\mu(D-B)\leq \epsilon.
\end{equation}
Therefore $\mu(D)<\infty$
Moreover $D=\setlimup{n}D_n$, where $D_n\in \mathcal F_0$ are the $n$ partial unions in the defintion of $D$.
Therefore $\mu(D_n)\uparrow \mu(D)<\infty$ and hence we can fine a
 a $D_n$ such that $D_n\subset D$ and
\begin{equation}
\mu(D) - \mu(D_n)=\mu(D - D_n)\leq \epsilon
\end{equation}
Now
\begin{align*}
\mu(B\,\triangle\, D_n)
&\leq \mu(B\cap D_n^c) + \mu(B^c \cap D_n)\\
&\leq \mu(D \cap D_n^c) + \mu(B^c \cap D)\\
&= \mu(D - D_n) + \mu(D - A)\\
&\leq 2\epsilon.
\end{align*}
\end{exerciseproof}








\begin{exercise}
(a) Show that $(\Omega, \bar{\mathcal F}, \bar P)$ is the smallest complete extension of $(\Omega, \mathcal F, P)$---that is, if $(\Omega, \mathcal F^\prime, P^\prime)$ is probability space which is a complete extension of $(\Omega, \mathcal F, P)$, then $(\Omega, \mathcal F^\prime, P^\prime)$ is also a complete extension of $(\Omega, \bar{\mathcal F}, \bar P)$. (b) Show by example that $(\Omega, \bar{\mathcal F}, \bar P)$ can have infinitely many different complete extensions (Hint: use a sample space consisting of two points).
\end{exercise}






%-------------------------------'
%---------section  ---------------'
%-------------------------------'
\subsection{Application: Lebesque Measure}
%-------------------------------'
%---------section  ---------------'
%-------------------------------'






%%%%%%
\begin{theorem}
\label{thm: Borels P is a measure}
 The mapping $P:\mathcal B_0((0,1])\rightarrow [0,1]$ defined in Definition \ref{l1defP} is a probability measure.
 \end{theorem}
\begin{proof}
We will use Theorem \ref{thm: cae} and show that $P$ is continuous from above at $\varnothing.$ Let $A_n \downarrow \varnothing$ where $A_n\in \mathcal B_0((0,1])$ (in particular we have that $\bigcap_{k=1}^\infty A_k =\varnothing$). We show $P[A_n]\downarrow 0$.

Notice first that for all $n\geq N$ we have
\[ P[A_n]\leq P[A_N] =  P\bigl[ \textstyle\bigcap_{k=1}^N A_k\bigr] \]
since the $A_k$'s are decreasing. It will then be sufficient to show that for all $\epsilon>0$ there exists  $N_\epsilon$ such that
\begin{equation}
\label{eq: show pm for bo}
P\bigl[ \textstyle\bigcap_{k=1}^{N_\epsilon} A_k\bigr]\leq \epsilon.
\end{equation}

The following argument doesn't quite work but it will motivate the solution. Let $\epsilon > 0$ and for each $A_k$  find a sequence of closed sets $F_k$ such that $F_k\subset A_k$ and $P[A_k-F_k]\leq \epsilon/2^k$ (If $A_k$ has the form $\bigcup_i (a_i,b_i]$ take $F_k:=\bigcup_i [a_i+\tau,b_i]$ for small enough $\tau$). Since $\bigcap_{k=1}^\infty A_k=\varnothing$ one has that $\bigcap_{k=1}^\infty F_k=\varnothing$.
By a compactness argument\footnote{For, if not, then there exists $x_n\in \bigcap_{k=1}^n F_k$ for each $n$. Notice
\begin{equation}
\label{foot:main eq}
\text{$\bigcap_{k=1}^n F_k\subset \bigcap_{k=1}^m F_k$ when $m\leq n$}
\end{equation}
For one thing, equation (\ref{foot:main eq}) implies that $x_n\in F_1$. Therefore by compactness  there exists a  sub-sequential limit $x=\lim_k x_{n_k} \in F_1$.  Again by (\ref{foot:main eq}) and the assumption  $x_{n_k}\in \bigcap_{k=1}^{n_k} F_k$ one has that for sufficiently large $k$ all  $x_{n_k}$ are eventually within $F_m$. Therefore $x\in F_m$ for each $m$. This contradicts the assumption $\cap_{k=1}^\infty F_k=\varnothing$.  } there exists an $N_\epsilon$ such that $\bigcap_{k=1}^{N_\epsilon} F_k=\varnothing$.
 Therefore
\[
P\bigl[ \textstyle\bigcap_{k=1}^{N_\epsilon} A_k\bigr] - \underbrace{P\bigl[ \underbrace{\textstyle\bigcap_{k=1}^{N_\epsilon} F_k}_{=\varnothing}\bigr]}_{=0}\leq  \sum_{k=1}^{N_\epsilon} P\bigl[A_k-F_k \bigr] \leq \sum_{k=1}^{N_\epsilon} \frac{1}{2^k}\leq \epsilon.
\]
This would establish (\ref{eq: show pm for bo}) if it weren't for the problem that $P$ isn't defined on the $F_k$'s.

It is clear how to fix this. For each $A_k$ find  closed sets $F_k$ and $\mathcal B_0((0,1])$-sets $A_k^o$ such that $A_k\supset F_k \supset A_k^o$ and $P[A_k-A_k^o]\leq \epsilon / 2^k$. For each $\epsilon>0$ we still have the property that there exists $N_\epsilon$ such that  $\bigcap_{k=1}^{N_\epsilon}F_k = \varnothing$ which  implies $\bigcap_{k=1}^{N_\epsilon}A^o_k = \varnothing$  and now
\[
P\bigl[ \textstyle\bigcap_{k=1}^{N_\epsilon} A_k\bigr] - \underbrace{P\bigl[ \underbrace{\textstyle\bigcap_{k=1}^{N_\epsilon} A^o_k}_{=\varnothing}\bigr]}_{=0}\leq  \sum_{k=1}^{N_\epsilon} P\bigl[A_k-A^o_k \bigr] \leq \sum_{k=1}^{N_\epsilon} \frac{1}{2^k}\leq \epsilon.
\]
\end{proof}



\begin{theorem}[{\bf Application to Borel's normal numbers}]
Let $P:\mathcal B_0((0,1])\rightarrow [0,1]$ be as in Definition \ref{l1defP}. Then
\begin{enumerate}
\item\label{BB item 1} $P$ has a unique extension to a probably measure  on $\mathcal B((0,1])$ (still denoted $P$ for the rest of this theorem);
\item\label{BB item 2} $P$ is the only measure on  $\mathcal B((0,1])$ which satisfies $P[(0,x]]=x$ for all $x\in (0,1]$;
\item\label{BB item 3} $N\in \mathcal B((0,1])$ and $P[N]=1$ where $N$ is the set of normal numbers in $(0,1]$;
\item\label{BB item 4} $\mathcal B_0((0,1]) \subsetneq \mathcal B((0,1]) \subsetneq  \overline{\mathcal B((0,1])}\subsetneq 2^\Omega$. Sets in $\mathcal B((0,1])$ are called {\bf Borel measurable}. Sets in $\overline{\mathcal B((0,1])}$ are called {\bf Lebesgue measurable}.
\end{enumerate}
\end{theorem}

\begin{proof}
({\sl Show item \ref{BB item 1}}) First note the Carath\'eodory Extension Theorem  along with Theorem \ref{thm: Borels P is a measure} shows there exists a unique extension $P:\mathcal B_0((0,1])\rightarrow [0,1]$ to $P:\mathcal B((0,1])\rightarrow [0,1]$ since $\mathcal B((0,1])=\sigma\bigl\langle \mathcal B_0((0,1]) \bigr\rangle$.

({\sl Show item \ref{BB item 2}}) This follows from the uniqueness theorem for measures since $\mathcal B((0,1]) = \sigma\langle  (0,x]\colon x\in (0,1]\rangle$ and $\{(0,x]\colon x\in (0,1] \}$ is a $\pi$-system.


({\sl Show item \ref{BB item 3}})
Notice first that $N$ and $N^c$ are both in $\mathcal B((0,1])$.
Now, in Theorem \ref{thm: Borel's normal number theorem} we showed that $N^c$ is negligible. In particular, for any $\epsilon >0$, there exists $B_n\in \mathcal \mathcal B_0((0,1])$ such that $N^c\subset \bigcup_{n=1}^\infty B_n$ where  $\sum_{n=1}^\infty P[B_n]\leq \epsilon$. Since  $\bigcup_{n=1}^\infty B_n\in \mathcal B((0,1]) $ we have
\[P\Bigl(\bigcup_{n=1}^\infty B_n\Bigr)\leq \sum_{n=1}^\infty P[B_n]\leq \epsilon. \]
Therefore
\[P(N^c)= \inf\{P(B)\colon N^c\subset B\in \mathcal F  \}\leq \epsilon  \]
for all $\epsilon$. Therefore $P(N^c)=0$ and $P(N)=1$.


({\sl Show item \ref{BB item 4}}) We've already mentioned that $N\in \mathcal B((0,1])$ but $N\notin \mathcal B_0((0,1])$. Exercise 15 of page 15 in Chung (``A Course in Probability Theory'') shows that $\mathcal B((0,1]) \subsetneq  \overline{\mathcal B((0,1])}$. Billingsley (``Probability and Measure'') page 46 shows that it is impossible to extend $P$ to a probability measure on $2^\Omega$ which establishes that $ \overline{\mathcal B((0,1])}\subsetneq 2^\Omega$.

\end{proof}





For any $\bs i=(i_1,\ldots, i_d)\in\Bbb Z^d$ let $(\bs i,\bs i+1]$ be the unit cube in $\Bbb R^d$ translated up by $\bs i$ so that
\[(\bs i,\bs i+1] \equiv (i_1, i_1+1]\times \cdots \times (i_d,i_d + 1].  \]
Notice that these sets give a checker board decomposition, $\Bbb R^d=\bigcup_{\bs i\in\Bbb Z^d} (\bs i,\bs i+1] $, so that $\Bbb R^d$ is expressed as a countable disjoint union of the translated unit cubes. Let $\mathcal B_{0}^{(\bs i,\bs i+1] }$ denote the field of finite disjoint unions of rectangles in $(\bs i,\bs i+1] $ and let $\mathcal B((\bs i,\bs i+1])\equiv \sigma \langle \mathcal B_{0}^{(\bs i,\bs i+1] }\rangle $ denote the Borel $\sigma$-field of  $(\bs i,\bs i+1]$. Finally let $P_{\bs i}$ denote the unique uniform probability measure on $\mathcal B((\bs i,\bs i+1])$ which assigns Euclidean volume  to the rectangles in  $(\bs i,\bs i+1] $, i.e.
\[ P_{\bs i}\bigl( (a_1,b_1]\times \cdots \times (a_d,b_d]\bigr)=\prod_{k=1}^d (b_k - a_k) \]
whenever $(a_1,b_1]\times \cdots \times (a_d,b_d]\subset (\bs i,\bs i+1]$. The construction of $P_{\bs i}$ is done in exactly the same way as the uniform probability measure was constructed on $(0,1]$ in the beginning of the class. Lets recall how this is done. One first shows that for any $A\in \mathcal \mathcal B_{0}^{(\bs i,\bs i+1] }$ one can define $P_{\bs i}(A)$ to be the sum  of the disjoint rectangle volumes which make up $A$ (this is not trivial since there are different decompositions of $A$ into disjoint rectangles, but one can use a result similar to Theorem 1.3 of Billingsley to prove that $P_{\bs i}$ is well defined).  Secondly, one shows that $P_{\bs i}$ is a probability measure on $((\bs i, \bs i +1],\mathcal B_{0}^{(\bs i,\bs i+1] })$. The hard part of this step is to  show the countable additivity. For $(0,1]$ we used the equivalent condition that  $P_{\bs i}$ is continuous from above at $\varnothing$. This argument carries over to $((\bs i, \bs i +1],\mathcal B_{0}^{(\bs i,\bs i+1] }, P_{\bs i})$. Finally one invokes the Carath\'eodory Extension theorem to get a uniform probability measure $((\bs i, \bs i +1],\mathcal B((\bs i,\bs i+1]), P_{\bs i})$ (uniqueness follows by the fact that rectangles, including the empty ones, form a $\pi$-system).

Now, using the uniform probability measures $((\bs i, \bs i +1],\mathcal B((\bs i,\bs i+1]), P_{\bs i})$ we can define Lebesque measure $\mathcal L^d$ on sets $A\in\mathcal B((\bs i,\bs i+1])$ by stitching these $P_{\bs i}$ together as follows
\begin{equation}
\label{Lm}
 \mathcal L^d(A):= \sum_{\bs i\in\Bbb Z^d} P_{\bs i}\bigl( (\bs i, \bs i+1]\cap A\bigr).
 \end{equation}
Notice that each $ (\bs i, \bs i+1]\cap A$ is in the Borel $\sigma$-field $\mathcal B((\bs i,\bs i+1])$ by Claim \ref{restricTHM} so that
$P_{\bs i}\bigl( (\bs i, \bs i+1]\cap A\bigr)$ is defined.
Lets see that $\mathcal L^d$ is indeed a measure on $(\Bbb R^d,\mathcal B(\Bbb R^d))$.


\begin{theorem}
 $\mathcal L^d$  is a measure on $(\Bbb R^d,\mathcal B(\Bbb R^d))$.
\end{theorem}
\begin{proof}
We show the following three axioms (i), (ii) and (iii):
\begin{enumerate}
\item[(i)] $\mathcal L^d(A)\in [0,\infty]$: Trivial.
 \item[(ii)]  $\mathcal L^d(\varnothing)=0$: This is also easy since $P_{\bs i}\bigl( (\bs i, \bs i+1]\cap \varnothing\bigr)=0$.
\item[(iii)] Countable additivity: Suppose $A_1,A_2,\ldots\in\mathcal B(\Bbb R^d)$ are disjoint. Then
\begin{align}
\mathcal L^d\Bigl (\bigcup_{k=1}^\infty A_k\Bigr)&=\sum_{\bs i\in\Bbb Z^d} P_{\bs i}\Bigl((\bs i, \bs i+1] \cap \bigcup_{k=1}^\infty A_k \Bigr) \nonumber\\
&=\sum_{\bs i\in\Bbb Z^d} P_{\bs i}\Bigl( \bigcup_{k=1}^\infty (\bs i, \bs i+1] \cap A_k \Bigr) \nonumber\\
&=\sum_{\bs i\in\Bbb Z^d} \sum_{k=1}^\infty P_{\bs i} \Bigl((\bs i, \bs i+1] \cap A_k \Bigr)\label{se} \\
&=\sum_{k=1}^\infty  \sum_{\bs i\in\Bbb Z^d} P_{\bs i} \Bigl((\bs i, \bs i+1] \cap A_k \Bigr)\label{se2} \\
&=\sum_{k=1}^\infty  \mathcal L^d\bigl (A_k \bigr) \nonumber
\end{align}
where (\ref{se}) follows since $P_{\bs i}$ is countably additive and the $ (\bs i, \bs i+1]\cap A_k $'s are disjoint; and (\ref{se2}) follows from general results about positive iterated sums.
\end{enumerate}
\end{proof}



 \begin{theorem}
 \label{ui2}
  $\mathcal L^d$ is the only measure on $(\Bbb R^d,\mathcal B((0,1]^d))$ which assigns standard Euclidean volume to the finite rectangles as follows
 \begin{equation}
\mathcal L^d \bigl( (a_1,b_1]\times \cdots \times (a_d,b_d]\bigr)=\prod_{k=1}^d (b_k - a_k)
\end{equation}
for $-\infty < a_k < b_k <\infty$.
\end{theorem}
\begin{proof}
Define $\mathcal P$ to be the $\pi$-system composed of the finite rectangles $\{ (a_1,b_1]\times \cdots \times (a_d,b_d]: -\infty < a_k < b_k <\infty\}$ and the empty set $\varnothing$. One can easily establish that $\mathcal B(\Bbb R^d)=\sigma\langle\mathcal P\rangle$.
 Also notice that $\mathcal L^d$ is $\sigma$-finite on $\mathcal P$ since  $\mathcal L^d\bigl((\bs i,\bs i+1])=1$, $\Bbb R^d = \cup_{\bs i\in\Bbb Z^d} (\bs i,\bs i+1]$ and each $(\bs i,\bs i+1]\in \mathcal P$.
 Therefore Theorem \ref{uui} establishes the following claim
\end{proof}




\begin{theorem}
For any $A\in\mathcal  B^{\Bbb R^d}$ and $x\in \Bbb R^d$, the set $A+x:= \{ a+x: a\in A\}$ is in $\mathcal B(\Bbb R^d)$ and
\begin{equation}
 \label{translate}
 \mathcal L^d(A+x) =  \mathcal L^d(A)
 \end{equation}
\end{theorem}
\begin{proof}
To show $A+x \in\mathcal B(\Bbb R^d)$ use the good sets principle. Fix $x\in \Bbb R^d$ and set   $\mathcal G_x:=\{ A\in \mathcal B(\Bbb R^d):  A+x \in \mathcal B(\Bbb R^d)\}$.
It is easy to see that $\mathcal G_x$ is a $\sigma$-field since complementation and union is preserved under translation by $x$. For example,
\begin{align*}
A\in\mathcal G_x &\Rightarrow  A\in \mathcal B(\Bbb R^d) \text{  and } A+x \in \mathcal B(\Bbb R^d) \\
&\Rightarrow  A^c\in \mathcal B(\Bbb R^d) \text{  and } (A+x)^c \in \mathcal B(\Bbb R^d) \\
&\Rightarrow  A^c\in \mathcal B(\Bbb R^d) \text{  and } A^c+x \in \mathcal B(\Bbb R^d) \\
&\Rightarrow A^c\in\mathcal G_x.
\end{align*}
The other axioms are established in a similar fashion.
Moreover, clearly all the finite rectangles are in $\mathcal G_x$. Therefore good sets implies $\mathcal B(\Bbb R^d) \subset \mathcal G_x$  which implies $A\in \mathcal B(\Bbb R^d)\rightarrow A+x \in \mathcal B(\Bbb R^d)$, as was to be shown.

Now to show (\ref{translate}) one can simply use the same arguments used in the Theorem \ref{ui2} on the uniqueness of $\mathcal L^d$. In particular, fix $x$ and define $\mu_x(A):= \mathcal L^d(A+x)$. It is easy to show that $\mu_x$ is a measure on $(\Bbb R^d, \mathcal B(\Bbb R^d))$. Moreover, since the volume of any rectangle in $\Bbb R^d$ is invariant under translation by $x$, the measures $\mu_x$ and $\mathcal L^d$ both agree on the $\pi$-system of finite, possibly empty, rectangles in $\Bbb R^d$. Since they are also both $\sigma$-finite on these rectangles one must have, by Theorem \ref{ui2}, $\mathcal L^d(A) =\mu_x(A):= \mathcal L^d(A+x) $ for all $A\in\mathcal B(\Bbb R^d)$, as was to be shown.
\end{proof}

%%%%%%%%%%%%%%%
\begin{theorem}
If $T:\Bbb R^d\rightarrow \Bbb R^d$ is linear and nonsingular, then $A\in\mathcal B(\Bbb R^d)$ implies that $T\!A:=\{ T(a): a\in A\}\in \mathcal B(\Bbb R^d)$ and
\[ \mathcal L^d(T\!A):=|\det T |\mathcal L^d(A). \]
\end{theorem}

%%
\begin{theorem}
\label{cc}
Let $(\Omega, \mathcal F,\mu)$ be a $\sigma$-finite measure space. Then $\mathcal F$ cannot contain an uncountable, disjoint collection of sets of positive $\mu$-measure
\end{theorem}
\begin{proof}
Let $\{B_i: i\in \mathcal I\}$ a disjoint collection of sets of such that $\mu(B_i)>0$ for each $i\in\mathcal I$.
We show $\mathcal I$ must be countable.


Since $\mu$ is $\sigma$-finite there exists $A_1,A_2,\ldots \in\mathcal F$ such that $\mu(A_k)<\infty$ and $\Omega = \cup_k A_k$.  We show the following three facts.
\begin{itemize}
\item $\{ i\in \mathcal I: \mu(A_k\cap B_i)>\epsilon\}$ is finite for all $k$:
Let $\epsilon>0$ and suppose by contradiction one can find a countably infinite set  $\mathcal I_c\subset \mathcal I$ such that $\mu(A_k\cap B_i)>\epsilon $ for all $i\in \mathcal I_c$ and for this set of indices one has
\begin{align*}
\mu(A_k)&\geq \mu(A_k\cap (\cup_{i\in\mathcal I_c}B_i ))= \sum_{i\in\mathcal I_c}\mu(A_k\cap B_i) > \sum_{i\in\mathcal I_c}\epsilon =\infty
\end{align*}
which gives a contradiction.
\item $\{ i\in \mathcal I: \mu(A_k\cap B_i)>0\}$ is countable for all $k$:
This follows from the  identity
\[\{ i\in \mathcal I: \mu(A_k\cap B_i)>0\}=\bigcup_\text{\small rational
$\epsilon$} \underbrace{\{ i\in \mathcal I: \mu(A\cap B_i)>\epsilon\}.} _\text{\small finite by (i)}   \]
\item $\mathcal I = \bigcup_k \{ i\in \mathcal I: \mu(A_k\cap B_i)>0\}$:
 To show $\mathcal I \cup \bigcup_k \{ i\in \mathcal I: \mu(A_k\cap B_i)>0\}$ notice that  if $i\in\mathcal I$ then $\mu(B_i)>0$. Now  $\Omega =\cup_k A_k$ so there must exist a $k$ such that $\mu(A_k\cap B_i)>0$.  Therefore $i\in  \bigcup_k \{ i\in \mathcal I: \mu(A_k\cap B_i)>0\}$. The other inclusion is obvious.
\end{itemize}

 To finish the proof simply notice that the last two bullets imply $\mathcal I$ is countable.
\end{proof}


%%%%%%%
\begin{corollary}If $k<d$ then
$\mathcal L^d(A)=0$ for any $k$-dimensional hyperplane $A\subset \Bbb R^d$ where $k<d$.
\end{corollary}
\begin{proof}
Let $A$ be a $k$-dimensional hyperplane where $k<d$. Let $x$ be a point in $\Bbb R^d$ which is not  contained in $A$.
Then $\{A+xt: t\in\Bbb R\}$ is an uncountable class of disjoint subsets of $\mathcal B(\Bbb R^d)$. Since $\mathcal L^d$ is translation invariance $\mathcal L^d(A) = \mathcal L^d(A+xt)$ for each $t\in\Bbb R$. Now by Theorem \ref{cc}, $\mathcal L^d(A)=0$, for otherwise there would exists a uncountable, disjoint collection of sets of positive $\mathcal L^d$-measure.
\end{proof}






\begin{definition}[{\bf Borel versus Lebesque measurable sets}]
Let $\bigl(\Omega,\overline{\mathcal B(\Bbb R^d)},\overline{\mathcal L^d}\bigr)$ be the completion of $(\Omega, \mathcal B(\Bbb R^d),\mathcal L^d)$. If $A\in \mathcal B(\Bbb R^d)$ then $A$ is said to be {Borel measurable}. If $A\in \overline{\mathcal B(\Bbb R^d)}$ then $A$ is said to be {Lebesque measurable}.
\end{definition}

\begin{theorem}[{\bf This is why we need $\sigma$-fields}]
$\phantom{asdf}$
\begin{itemize}
\item $\mathcal B(\Bbb R)\subsetneq \overline{\mathcal B(\Bbb R)} \subsetneq 2^\Bbb R$.
\item It is impossible to put a measure on $2^\Bbb R$ which is translation invariant and which assigns normal length to finite intervals. Put another way---there is no Lebesque measure on all of $2^\Bbb R$. Or another way---it is impossible to consistently assign a length to all subsets of $\Bbb R$.
\end{itemize}
\end{theorem}
